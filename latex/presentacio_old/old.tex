%----------------------------------------------------------------------------------------
%	PACKAGES AND THEMES
%----------------------------------------------------------------------------------------
\documentclass[aspectratio=169,xcolor=dvipsnames]{beamer}
% \usefolder{theme}
\usetheme{SimplePlus}
\usepackage{amsfonts, amsmath, amssymb,latexsym,amsthm, mathrsfs, enumerate, tikz-cd}
\usepackage{hyperref}
\usepackage[all]{xy}
\usepackage{algorithm}
\usepackage[noend]{algpseudocode}
\usepackage{hyperref}
\usepackage{graphicx} % Allows including images
\usepackage{booktabs} % Allows the use of \toprule, \midrule and \bottomrule in tables
\usepackage[latin1]{inputenc}
\usepackage{epsfig}
\SelectTips{cm}{}

\parskip=5pt
\parindent=15pt
\usepackage{graphicx}
\usepackage{listings}
\numberwithin{equation}{section}
\newtheorem{teo}{Teorema}[section]
\newtheorem*{teo*}{Teorema}
\newtheorem{prop}[teo]{Proposici\'on}
\newtheorem{corol}[teo]{Corolario}
\newtheorem{lema}[teo]{Lema}
\newtheorem{nota}[teo]{Notaci\'on}

\theoremstyle{definition}
\newtheorem{defi}[teo]{Definici\'on}
\newtheorem{prob}[teo]{Problema}
\newtheorem*{sol}{Soluci\'on}
\newtheorem{ex}[teo]{Ejemplo}
\newtheorem{exs}[teo]{Ejemplos}
\newtheorem{obs}[teo]{Observaci\'on}
\newtheorem{obss}[teo]{Observaciones}
\newcommand{\Set}{\mathop{\rm Set}}
\newcommand{\Grp}{\mathop{\rm Grp}}
\newcommand{\Top}{\mathop{\rm Top}}
\newcommand{\Oper}{\mathop{\rm Oper}}

\newenvironment{algoritmo}[1][htb]{%
    \floatname{algorithm}{Algoritmo}% Update algorithm name
   \begin{algorithm}[#1]%
  }{\end{algorithm}}
\renewcommand{\algorithmicrequire}{\textbf{Input:}}
\renewcommand{\algorithmicensure}{\textbf{Output:}}
\newcommand{\Break}{\textbf{break}}
\algblockdefx[ForElse]{ForElse}{EndForElse}{\textbf{else}}{}
\algtext*{EndForElse}
%----------------------------------------------------------------------------------------
%	TITLE PAGE
%----------------------------------------------------------------------------------------

% \textbf{\LARGE GRADO EN MATEM\'{A}TICAS }
\title[short title]{El producto tensorial de conjuntos dendroidales} % The short title appears at the bottom of every slide, the full title is only on the title page
\subtitle{Trabajo final de grado}

\author[Roger Brasco] {Roger Brasc\'o Garc\'es}

\institute[NTU] % Your institution as it will appear on the bottom of every slide, may be shorthand to save space
{
    Departamento de Matem\'aticas e Inform\'atica \\
    Universidad de Barcelona % Your institution for the title page
}
\date{9 de Febrero de 2022} % Date, can be changed to a custom date

\titlegraphic{\hfill\includegraphics[height=1cm]{statics/matematiquesinformatica-pos-rgb.png}}

%----------------------------------------------------------------------------------------
%	PRESENTATION SLIDES
%----------------------------------------------------------------------------------------

\begin{document}

\begin{frame}
    % Print the title page as the first slide
    \titlepage
\end{frame}

\begin{frame}{Introducci\'on}
    % Throughout your presentation, if you choose to use \section{} and \subsection{} commands, these will automatically be printed on this slide as an overview of your presentation
    \tableofcontents
\end{frame}

%------------------------------------------------
\section{Nociones previas}
%------------------------------------------------

\begin{frame}{Categor\'ias}
    \begin{block}{Block}
        Una \emph{categor\'{\i}a} $\mathcal{C}$ consiste en:
        % \begin{itemize}
        %     \item \emph{Objetos}: ${\rm Ob}(\mathcal{C})$.
        %     % \item Para cada par de objectos $A, B\in{\rm Ob}(\mathcal{C})$ un conjunto $\mathcal{C}(A,B)$ de \emph{morfismos} o \emph{flechas} de $A$ a $B$.
        %     % \item Para cada tres objectos $A, B, C\in{\rm Ob}(\mathcal{C})$ una \emph{funci\'on de composici\'on}
        %           $$
        %               \mathcal{C}(B,C)\times \mathcal{C}(A,B)\stackrel{\circ}{\longrightarrow} \mathcal{C}(A,C)
        %           $$
        %           que env\'{\i}a el par $(g,f)$ a $g\circ f$.
        %     \item Para cada objeto $A$, un elemento ${\rm id}_A\in\mathcal{C}(A,A)$ que llamaremos la \emph{identidad} en $A$.
        % \end{itemize}
    \end{block}
    Adem\'as, esta estructura cumple los siguientes axiomas:
        \begin{itemize}
            \item \emph{Asociatividad}. La funci\'on de composici\'on es asociativa, esto es, dados $f\in\mathcal{C}(A,B)$, $g\in\mathcal{C}(B,C)$ y $h\in\mathcal{C}(C,D)$, se cumple que $(h\circ g)\circ f=h\circ(g\circ f)$.
            \item \emph{Unidad}. La identidad es un elemento neutro para la composici\'on, es decir, para toda $f\in\mathcal{C}(A,B)$ tenemos que $f\circ {\rm id}_A=f={\rm id}_B\circ f$.
        \end{itemize}

\end{frame}

\begin{frame}{Funtores}

\end{frame}

\begin{frame}{Op\'eradas}

\end{frame}

%------------------------------------------------
\section{\'Arboles como op\'eradas coloreadas}
%------------------------------------------------

\begin{frame}{Formalismo de \'arboles}

\end{frame}

\begin{frame}{Categor\'ia \pmb{$\Omega_p$}}

\end{frame}
%------------------------------------------------

% \begin{frame}{References}
%     % Beamer does not support BibTeX so references must be inserted manually as below
%     \footnotesize{
%         \begin{thebibliography}{99}
%             \bibitem[Smith, 2012]{p1} John Smith (2012)
%             \newblock Title of the publication
%             \newblock \emph{Journal Name} 12(3), 45 -- 678.
%         \end{thebibliography}
%     }
% \end{frame}

%------------------------------------------------
\section{Producto Tensorial}
%------------------------------------------------

\begin{frame}{Producto Tensorial de Boardman--Vogt}

\end{frame}

\begin{frame}{Producto Tensorial de Conjuntos Dendroidales}

\end{frame}

%------------------------------------------------
\section{Conjunto de Shuffles}
%------------------------------------------------

\begin{frame}{Shuffles}
    H
\end{frame}

\begin{frame}{Estructura de orden parcial}
    H
\end{frame}

\begin{frame}{Generar Shuffles en Python}

\end{frame}
%------------------------------------------------
\section{Conclusiones}
%------------------------------------------------

\begin{frame}{Conclusiones}
    H
\end{frame}

%------------------------------------------------

\begin{frame}
    \Huge{\centerline{\textbf{Gracias por vuestra atenci\'on}}}
\end{frame}

%----------------------------------------------------------------------------------------

\end{document}