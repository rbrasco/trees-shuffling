\chapter{Tensor product of dendroidal sets}
\label{lecture04}
Like any category of presheaves, the category of dendroidal sets is cartesian closed. In this lecture, we will discuss another monoidal
structure, which is also closed, and seems more relevant than the cartesian structure. It is closely related to the tensor product of
operads introduced by Boardman and Vogt, and it makes the embedding of simplicial sets into dendroidal sets into a strong monoidal functor.
\section{The Boardman--Vogt tensor product}
\label{bvtensor}
The category of small categories $\Cat$ is a cartesian closed category for which the internal hom $\Hom_{\Cat}(\C, \D)$ between two categories $\C$ and $\D$ is defined as the category whose objects are functors from $\C$ to $\D$ and whose morphisms are natural transformations between them. In this section, we show that the category of coloured operads is a closed symmetric monoidal category with the so-called Boardman--Vogt tensor product \cite[Definition 2.14]{BV73}.

We recall the definition of the Boardman--Vogt tensor product for coloured operads.

\begin{defn}
Let $P$ be a symmetric $C$-coloured operad and let $Q$ be a symmetric $D$-coloured operad. The \emph{Boardman--Vogt tensor product} $P\otimes_{BV}Q$ is a $(C\times D)$-coloured operad defined in terms of generators and relations in the following way. For each $d\in D$ and each operation $p\in P(c_1,\ldots, c_n;c)$ there is a generator
$$
p\otimes d\in P\otimes_{BV}Q ((c_1, d),\ldots, (c_n,d); (c,d)).
$$
Similarly, for each $c\in C$ and each $q\in Q(d_1,\ldots, d_m;d)$ there is a generator
$$
c\otimes q\in P\otimes_{BV}Q ((c,d_1), \ldots, (c, d_m); (c,d)).
$$
These generators are subject to the following relations:
\begin{itemize}
\item[(i)] $(p\otimes d)\circ ((p_1\otimes d),\ldots, (p_n\otimes d))=(p\circ (p_1,\ldots, p_n))\otimes d$.
\item[(ii)] $\sigma^*(p\otimes d)=(\sigma^*p)\otimes d$ for every $\sigma\in \Sigma_n$.
\item[(iii)] $(c\otimes q)\circ ((c\otimes q_1),\ldots, (c\otimes q_m))=c\otimes (q\circ (q_1,\ldots, q_m))$.
\item[(iv)] $\sigma^*(c\otimes q)=c\otimes (\sigma^*q)$ for every $\sigma\in \Sigma_m$.
\item[(v)] $\sigma_{n,m}^*((p\otimes d)\circ((c_1\otimes q),\ldots, (c_n\otimes q)))=
(c\otimes q)\circ ((p\otimes d_1),\ldots, (p\otimes d_m))$, where
$\sigma_{n,m}\in \Sigma_{nm}$ is the permutation described as
follows. Consider $\Sigma_{nm}$ as the set of bijections of
the set $\{0,1,\ldots, nm-1\}$. Each element of this set can
be written uniquely in the form $kn+j$ where $0\le k<m$ and $0\le
j<n$ as well as in the form $km+j$ where $0\le k<n$ and $0\le j<m$.
The permutation $\sigma_{n,m}$ is then defined by
$\sigma_{n,m}(kn+j)=jm+k$.
\end{itemize}
\end{defn}

Observe that relations (i) and (ii) imply that for every $d\in D$ the map $P\longrightarrow P\otimes_{BV}Q$ given by $p\longmapsto p\otimes d$ is a map of operads. Similarly, relations (iii) and (iv) ensure that for every $c\in C$ the map $Q\longrightarrow P\otimes_{BV}Q$ given by $q\longmapsto c\otimes q$ is a map of operads.

\begin{exmp} We illustrate relation (v), called the \emph{interchange relation}, with the following examples.
Suppose that $n=2$ and $m=3$. The left-hand operation of relation~(v), before applying $\sigma^*_{2,3}$, can be represented by the tree
$$
\xymatrix{
*{}\ar@{-}[dr]_{(c_1,d_1)} & *{}\ar@{-}[d]|(0.3){(c_1,d_2)}  & *{}\ar@{-}[dl]^{(c_1,d_3)}  &  & *{}\ar@{-}[dr]_{(c_2,d_1)}  & *{}\ar@{-}[d]|(0.3){(c_2,d_2)}  & *=0{}\ar@{-}[dl]^{(c_2,d_3)} \\
\ar@{}[r]|{\,\,c_1\otimes q} & *=0{\bullet}\ar@{-}[drr]_{(c_1,d)}  &  &  &  & *=0{\bullet}\ar@{-}[dll]^{(c_2,d)} & \ar@{}[l]|{c_2\otimes q}\\
 &  &  & *=<3.1pt>{\circ} \ar@{-}[d]^{(c,d)} & \ar@{}[l]|{p\otimes d} \\
 &  &  & *{}
}
$$
The right-hand operation can be represented by the tree
$$
\xymatrix @C=20pt{
*{}\ar@{-}[dr]_{(c_1, d_1)} &  & *{}\ar@{-}[dl]^{(c_2, d_1)} &  & *{}\ar@{-}[dr]_{(c_1, d_2)} &  & {}\ar@{-}[dl]^{(c_2, d_2)} &  & *{}\ar@{-}[dr]_{(c_1, d_3)} &  & *{}\ar@{-}[dl]^{(c_2, d_3)}\\
\ar@{}[r]|{\,\, p\otimes d_1} & *=<3.1pt>{\circ}\ar@{-}[drrrr]_{(c, d_1)}  &  &  &  \ar@{}[r]|{\,\,p\otimes d_2}& *=<3.1pt>{\circ}\ar@{-}[d]_{(c, d_2)} &  &  &  & *=<3.1pt>{\circ}\ar@{-}[dllll]^{(c, d_3)} & \ar@{}[l]|{p\otimes d_3}\\
 &  &  &  &  & *=0{\bullet}\ar@{-}[d]_{(c, d)} & \ar@{}[l]|(0.3){c\otimes q}\\
 &  &  &  &  & *{}
}
$$
And the permutation $\sigma_{2,3}$ corresponds to the permutation $(2\,\; 4\,\; 5\,\; 3)$ of $\Sigma_6$. We represent the vertices coming from operations in $P$ by a white dot $\circ$ and the vertices coming from operations in $Q$ by a black dot $\bullet$.
\end{exmp}

The Boardman--Vogt tensor product preserves colimits in each variable separately. In fact, there is a corresponding internal hom making the category of coloured operads closed monoidal.

\begin{thm}
The category $\Oper$ with the Boardman--Vogt tensor product $\otimes_{BV}$ is a closed symmetric monoidal category.
\end{thm}
\begin{proof}
The unit for the tensor product is the initial operad $I$ on one colour, i.e., $I(*,*)=\{*\}$ and the empty set otherwise. It follows from the definition that this tensor product is associative, commutative and unital. This proves that $\Oper$ is symmetric monoidal.

We define the internal hom for coloured operads as follows. Let $P$ be a $C$\nobreakdash-coloured operad and $Q$ a $D$-coloured operad. Then $\Hom_{\Oper}(P,Q)$ is the operad whose colours are the maps of operads $P\longrightarrow Q$, and if $\alpha_1,\ldots, \alpha_n,\beta$ are $n+1$ such maps, then the elements of
$$
\Hom\nolimits_{\Oper}(P,Q)(\alpha_1,\ldots, \alpha_n;\beta)
$$
are maps $f$ assigning to each colour $c\in C$ an element $f_c\in Q(\alpha_1 c, \dots, \alpha_n c; \beta c)$.
The maps $f_c$ should be natural with respect to all the operations in $P$. For example, if $p\in P(c_1, c_2; c)$, then
$$
\beta(p)(f_{c_1}, f_{c_2})\in Q(\alpha_1 c_1,\dots, \alpha_n c_1, \alpha_1 c_2,\ldots, \alpha_n c_2; \beta c)
$$
is the image under a suitable permutation of
$$
f_c(\alpha_1(p), \ldots, \alpha_n(p))\in Q(\alpha_1 c_1,\alpha_1 c_2, \dots, \alpha_n c_1,\alpha_n c_2; \beta c).
$$
We need to construct a bijection
$$
\Oper(P\otimes_{BV} Q, R)\cong \Oper(P, \Hom\nolimits_{\Oper}(Q, R))
$$
natural in $P$, $Q$ and $R$.

Let $\varphi\colon P\otimes_{BV} Q\longrightarrow R$ be a map of coloured operads. For each $c\in C$ we have a map of operads $\varphi_c$ defined by the composition
$$
Q\longrightarrow P\otimes_{BV} Q\stackrel{\varphi}{\longrightarrow} R
$$
where the first map sends $q$ to $c\otimes q$. This defines a map from the colours of $P$ to the colours of $\Hom_{\Oper}(Q, R)$. Now, if we have an operation $p\in P(c_1, \dots, c_n;c)$, we define
$$
f_d=\varphi(p\otimes d)\in R(\varphi_{c_1} d, \dots, \varphi_{c_n}d; \varphi_c d)
$$
for every $d\in D$.

Conversely,  let $\psi\colon P\longrightarrow \Hom_{\Oper}(Q,R)$ be a map of coloured operads. To construct a map $\overline{\psi}\colon P\otimes_{BV} Q\longrightarrow R$, we need to define it on the colours and the generators of $P\otimes_{BV} Q$. If $(c,d)\in C\times D$, then $\overline{\psi}(c,d)=\psi(c)(d)$. For a generator of the form $c\otimes q$, where $q\in Q(d_1,\ldots, d_n;d)$, we define
$$
\overline{\psi}(c\otimes q)=\psi(c)(q).
$$
For a generator of the form $p\otimes d$, where $p\in P(c_1,\ldots, c_n;c)$, we define
$$
\overline{\psi}(p\otimes d)=\psi(p)_d.
$$
It is now easy to check that $\overline{\psi}$ thus defined is compatible with the relations of the Boardman--Vogt tensor product.
\end{proof}

\begin{rem}
Note that in the definition of the Boardman--Vogt tensor product it is crucial that the coloured operads involved are symmetric. However, the tensor product
still makes sense without the symmetries when one of the operads involved has only unary operations.
\label{remBV}
\end{rem}

\section{Tensor product of dendroidal sets}
The category of dendroidal sets is a category of presheaves, hence cartesian closed. The cartesian product of dendroidal sets extends the cartesian product of simplicial sets, i.e.,
$$
i_!(X\times Y)\cong i_!(X)\times i_!(Y)
$$
for every two simplicial sets $X$ and $Y$. (Note, however, that $i_!$ does not preserve the terminal object.)

As mentioned before, there is another closed monoidal structure on $\dSets$, strongly related with the Boardman--Vogt
tensor product of coloured operads. For any two trees $T$ and $S$ in $\Omega$, the tensor product of the representables
$\Omega[T]$ and $\Omega[S]$ is defined as
$$
\Omega[T]\otimes \Omega[S]= N_d(\Omega(T)\otimes_{BV}\Omega(S)),
$$
where $N_d$ is the dendroidal nerve functor (see Example~\ref{dnerve}), $\Omega(T)$ and $\Omega(S)$ are the coloured operads associated to the trees $T$ and $S$ respectively (see Section~\ref{nonplanartrees}), and $\otimes_{BV}$ is the Boardman--Vogt tensor product.

This defines a tensor product in the whole category of dendroidal sets, since, being a category of presheaves, every object is a canonical colimit of representables and $\otimes$ preserves colimits in each variable.

\begin{defn}
Let $X$ and $Y$ be two dendroidal sets and let $X=\varinjlim \Omega[T]$ and $Y=\varinjlim\Omega[S]$ be their canonical
expressions as colimits of representables. Then the \emph{tensor product} $X\otimes Y$ is defined as
$$
X\otimes Y=\varinjlim \Omega[T]\otimes \varinjlim \Omega[S] = \varinjlim N_d(\Omega(T)\otimes_{BV} \Omega(S)).
$$
\end{defn}

It follows from general category theory \cite{Kel82} that this tensor product is automatically closed, and that the
set of $T$-dendrices of the internal hom is defined by
$$
\Hom\nolimits_{\dSets}(X, Y)_T=\dSets(\Omega[T]\otimes X, Y),
$$
for every two dendroidal sets $X$ and $Y$ and every $T$ in $\Omega$. The dendroidal structure of $\Hom_{\dSets}(X, Y)$ is given in the obvious way.

\begin{thm}
The category of dendroidal sets admits a closed symmetric mo\-noi\-dal structure. This monoidal structure
is uniquely determined (up to isomorphism) by the property that there is a natural isomorphism
$$
\Omega[T]\otimes \Omega[S]\cong N_d((\Omega(T)\otimes_{BV}\Omega(S))
$$
for any two objects $T$ and $S$ of $\Omega$. The unit of the tensor product is the representable dendroidal set $\Omega[\eta]=i_!(\Delta[0])=U$. $\hfill\qed$
\label{dsetsismonoidal}
\end{thm}

The following are some basic properties of the tensor product of dendroidal sets in relation to the Boardman--Vogt tensor product of coloured operads and to the cartesian product of simplicial sets.

\begin{prop}
The following properties hold:
\begin{itemize}
\item[{\rm (i)}] For any two simplicial sets $X$ and $Y$, there is a natural isomorphism
$$
i_!(X)\otimes i_!(Y)\cong i_!(X\times Y).
$$
\item[{\rm (ii)}] For any two dendroidal sets $X$ and $Y$, there is a natural isomorphism
$$
\tau_d(X\otimes Y)\cong \tau_d(X)\otimes_{BV} \tau_d (Y).
$$
\item[{\rm (iii)}] For any two coloured operads $P$ and $Q$, there is a natural isomorphism
$$
\tau_d(N_d(P)\otimes N_d(Q))\cong P\otimes_{BV} Q.
$$
\end{itemize}
\label{proptensor}
\end{prop}
\begin{proof}
To prove (i), it is enough to check that it holds for the representables in~$\sSets$. Note first that, if we view $[n]$ and $[m]$ in $\Delta$ as categories, then by using the linear order one has that
$$
j_!([n]\times [m])\cong j_!([n])\otimes_{BV} j_!([m]).
$$
Therefore, there is a chain of natural isomorphisms
\begin{multline}\notag
 i_!(\Delta[n]\times \Delta[m])\cong i_!(N([n])\times N([m]))\cong i_!(N([n]\times [m])) \\ \notag
\cong N_d j_!([n]\times [m]) \cong N_d(j_!([n])\otimes_{BV} j_!([m]))\cong N_d(\Omega(L_n)\otimes_{BV}\Omega(L_m))\\ \notag
\cong \Omega[L_n]\otimes \Omega[L_m] \cong i_!(\Delta[n])\otimes i_!(\Delta[m]),
\end{multline}
where $L_n$ and $L_m$ denote the linear tree with $n$ and $m$ vertices and $n+1$ and $m+1$ edges respectively.

Again, to prove (ii) it suffices to do it for representables in $\dSets$. But this is clear by using the natural isomorphism $\tau_d N_d \cong {\rm id}$. More precisely,
\begin{multline}
\tau_d(\Omega[T]\otimes \Omega[S])\cong \tau_d N_d(\Omega(T)\otimes_{BV}\Omega(S)) \\ \notag
\cong \Omega(T)\otimes_{BV} \Omega(S)\cong \tau_d(\Omega[T])\otimes_{BV}\tau_d(\Omega[S]).
\end{multline}
Part (iii) follows from part (ii) by using again that $\tau_d N_d \cong {\rm id}$ and replacing $X$ by $N_d(P)$ and $Y$ by $N_d(Q)$.
\end{proof}

\begin{rem}
There is no tensor product in the category of planar dendroidal sets coming from the Boardman--Vogt tensor product,
since the latter is defined only for symmetric operads. However, as we have seen in Remark~\ref{remBV}, the
Boardman--Vogt tensor product still makes sense for non-symmetric operads when at least one of them has only unary operations. This means that, although we cannot define $X\otimes Y$ for planar dendroidal sets $X$ and $Y$ in general, we can define $u_!(K)\otimes Y$ where $K$ is any simplicial set and $Y$ is any planar dendroidal set. In fact, $\pdSets$ is a simplicial category with tensors and cotensors.
\label{tensornonsym}
\end{rem}

\begin{thm}
The category $\pdSets$ of planar dendroidal sets is enriched, tensored and cotensored over simplicial sets.
\label{pdsetsenriched}
\end{thm}
\begin{proof}
Given two planar dendroidal sets $X$ and $Y$, the simplicial enrichment $\Hom(X,Y)$ is defined by
$$
\Hom(X,Y)_n=\pdSets(u_!(\Delta[n])\otimes X, Y)
$$
where $u_!\colon \sSets\longrightarrow \pdSets$ is the left adjoint to the functor $u^*$ induced by the inclusion $u\colon \Delta\longrightarrow \Omega_p$. If $K$ is any simplicial set, we define a tensor
$$
K\otimes Y=u_!(K)\otimes Y
$$
and a cotensor
$$
(Y^K)_T=\pdSets(\Omega_p[T]\otimes u_!(K), Y)
$$
for every planar dendroidal set $Y$.
\end{proof}

Thus, the Boardman--Vogt tensor product makes $\sSets$ into a cartesian closed category, $\pdSets$ into a simplicial category with tensors and cotensors, and $\dSets$ into a closed symmetric monoidal category. In fact, if we consider the cartesian structures on $\Cat$ and $\sSets$, the Boardman--Vogt tensor product on $\Oper$ and the tensor product of dendroidal sets, then in the commutative diagram
$$
\xymatrix{
\sSets \ar@<.75ex>[r]^{i_!}\ar@<-.75ex>[d]_{\tau}  & \dSets \ar@<.75ex>[l]^{i^*}\ar@<-.75ex>[d]_{\tau_d}  \\
\Cat  \ar@<-.75ex>[u]_{N}  \ar@<.75ex>[r]^{j_!} & \Oper \ar@<.75ex>[l]^{j^*}  \ar@<-.75ex>[u]_{N_d}
}
$$
the functors $i_!$, $N$, $\tau$, $j_!$ and $\tau_d$ are strong monoidal. However, $i^*$ and $j^*$ are not strong monoidal functors.
For example, if we denote by $T_1$, $T_2$ and $T_3$ the following trees:
$$
\xy<0.08cm, 0cm>:
(0,0)*=0{}="1";
(0,10)*=0{\bullet}="2";
(-10,20)*=0{}="3";
(10,20)*=0{}="4";
(-10,0)*=0{T_1};
"1";"2" **\dir{-};
"2";"3" **\dir{-};
"2";"4" **\dir{-};
(50,0)*=0{}="1";
(50,10)*=0{\bullet}="2";
(40,0)*=0{T_2};
"1";"2" **\dir{-};
(100,0)*=0{}="1";
(100,10)*=0{\bullet}="2";
(90,20)*=0{\bullet}="3";
(110,20)*=0{\bullet}="4";
(90,0)*=0{T_3};
"1";"2" **\dir{-};
"2";"3" **\dir{-};
"2";"4" **\dir{-};
\endxy
$$
then $i^*(\Omega[T_1])=\emptyset$ and $i^*(\Omega[T_2])=\Delta[0]$, but $\Omega[T_1]\otimes\Omega[T_2]=\Omega[T_3]$ and
$$
i^*(\Omega[T_3])=\Delta[1]\cup_{\Delta[0]}\Delta[1].
$$
\begin{rem}
If $\E$ is a complete and cocomplete monoidal category, then the category $\E^{\Omega^{\rm op}}$ of dendroidal objects in $\E$ also has a Boardman--Vogt
type tensor product. For any two objects $X$ and $Y$ in $\E^{\Omega^{\rm op}}$, their tensor product is defined by the following formula:
$$
(X\otimes Y)_T=\varinjlim_{\Omega[T]\rightarrow \Omega[R]\otimes\Omega[S]} X_R\otimes_{\E} Y_S,
$$
where $\otimes_{\E}$ is the tensor product of $\E$. If $\E$ is closed, then so is $\E^{\Omega^{\rm op}}$ (see \cite[Appendix A]{MW07} and
\cite[\S 7]{BM08}).
\label{dobjectsismonoidal}
\end{rem}

\section{Shuffles of trees}

In this section, we describe the tensor product $\Omega[S]\otimes\Omega[T]$ for any two trees in~$\Omega$, in order to give a better understanding of the tensor product of dendroidal sets. Suppose first that $S=L_n$ and $T=L_m$ are linear trees. Then, by Proposition~\ref{proptensor}(i),
$$
\Omega[L_n]\otimes\Omega[L_m]=i_!(\Delta[n])\otimes i_!(\Delta[m])\cong i_!(\Delta[n]\times \Delta[m]).
$$
The non-degenerate simplices of a product of two representables in simplicial sets are computed by means of
\emph{shuffles}. An $(n,m)$-shuffle is a path of maximal length in the partially ordered set $[n]\times [m]$.
The non-degenerate $(n+m)$-simplices of $\Delta[n]\times \Delta[m]$ correspond to $(n+m)$-shuffles. In fact,
$$
\Delta[n]\times\Delta[m]=\bigcup_{(n,m)}\Delta[n+m],
$$
where the union is taken over all possible $(n,m)$-shuffles.
\begin{exmp}
Let $n=2$ and $m=1$. There are three $(2,1)$-shuffles in $[2]\times [1]$, namely $(00, 01, 02, 12)$,
$(00, 01, 11, 12)$ and $(00, 10, 11, 12)$. If we picture $\Delta[2]\times \Delta[1]$ as the following prism
$$
\xy<0.06cm, 0cm>:
(0,0)*=0{}="00";
(30,0)*=0{}="01";
(0,20)*=0{}="10";
(30,20)*=0{}="11";
(50,20)*=0{}="02";
(50,40)*=0{}="12";
"00";"01" **\dir{-};
"00";"10" **\dir{-};
"00";"02" **\dir{--};
"01";"11" **\dir{-};
"01";"02" **\dir{-};
"02";"12" **\dir{-};
"10";"11" **\dir{-};
"10";"12" **\dir{-};
"11";"12" **\dir{-};
(-5,0)*{\scriptstyle{00}};
(35,0)*{\scriptstyle{01}};
(-5,20)*{\scriptstyle{10}};
(35,20)*{\scriptstyle{11}};
(55,40)*{\scriptstyle{12}};
(55,20)*{\scriptstyle{02}};
\endxy
$$
we can see that each $(2,1)$-shuffle corresponds to a tetrahedron, and that they give a decomposition of $\Delta[2]\times\Delta[1]$ as the union of three copies of $\Delta[3]$:
$$
\xy<0.08cm, 0cm>:
(0,0)*{
\xy<0.05cm, 0cm>:
(0,0)*=0{}="00";
(30,0)*=0{}="01";
(0,20)*=0{}="10";
(30,20)*=0{}="11";
(50,20)*=0{}="02";
(50,40)*=0{}="12";
"00";"01" **\dir{-};
"01";"02" **\dir{-};
"02";"12" **\dir{-};
"00";"12" **\dir{-};
"01";"12" **\dir{-};
"00";"02" **\dir{--};
(15,-8)*{(00,01,02,12)}
\endxy
};
(45,0)*{
\xy<0.05cm, 0cm>:
(0,0)*=0{}="00";
(30,0)*=0{}="01";
(0,20)*=0{}="10";
(30,20)*=0{}="11";
(50,20)*=0{}="02";
(50,40)*=0{}="12";
"00";"01" **\dir{-};
"01";"11" **\dir{-};
"11";"12" **\dir{-};
"01";"12" **\dir{-};
"00";"12" **\dir{-};
"00";"11" **\dir{-};
(15,-8)*{(00,01,11,12)}
\endxy
};
(90,0)*{
\xy<0.05cm, 0cm>:
(0,0)*=0{}="00";
(30,0)*=0{}="01";
(0,20)*=0{}="10";
(30,20)*=0{}="11";
(50,20)*=0{}="02";
(50,40)*=0{}="12";
"00";"10" **\dir{-};
"10";"11" **\dir{-};
"11";"12" **\dir{-};
"00";"11" **\dir{-};
"10";"12" **\dir{-};
"00";"12" **\dir{--};
(15,-8)*{(00,10,11,12)}
\endxy
};
\endxy
$$
\end{exmp}

To give an explicit description of the tensor product of any two representables in $\Omega$, we need to introduce shuffles of trees. Recall that we denote by $E(T)$ the set of edges of a given tree $T$.

\begin{defn}
Let $T$ and $S$ be two objects of $\Omega$. A \emph{shuffle} of $S$ and $T$ is a tree $R$ whose set of edges is a subset of $E(T)\times E(S)$. The root of $R$ is $(a,x)$ where $a$ is the root of $S$ and $x$ is the root of $T$, and its leaves are labelled by all pairs of the form $(l_S, l_T)$, where $l_S$ is a leaf of $S$ and $l_T$ is a leaf of $T$. Its vertices are either or the form
$$
\xy<0.08cm, 0cm>:
(0,0)*=0{}="1";
(0,10)*\cir<2pt>{}="2";
(-10,20)*=0{}="3";
(10,20)*=0{}="4";
(0,15)*=0{\cdots};
(2.5,10)*=0{\scriptstyle u};
(-10,12.5)*=0{\scriptstyle (a_1,x)};
(10,12.5)*=0{\scriptstyle (a_n,x)};
(5,5)*=0{\scriptstyle (b,x)};
"1";"2" **\dir{-};
"2";"3" **\dir{-};
"2";"4" **\dir{-};
(25,10)*=0{\mbox{or}};
(50,0)*=0{}="5";
(50,10)*=0{\bullet}="6";
(40,20)*=0{}="7";
(60,20)*=0{}="8";
(50,15)*=0{\cdots};
(52.5,10)*=0{\scriptstyle v};
(40,12.5)*=0{\scriptstyle (a,x_1)};
(60,12.5)*=0{\scriptstyle (a,x_m)};
(55,5)*=0{\scriptstyle (a,y)};
"5";"6" **\dir{-};
"6";"7" **\dir{-};
"6";"8" **\dir{-};
\endxy
$$
where $u$ is a vertex of $S$ with input edges $a_1,\ldots, a_n$ and output $b$, and $v$ is a vertex of $T$ with input
edges $x_1,\ldots, x_m$ and output~$y$. We will refer to the first type of vertices as \emph{white vertices} and to the
second type of vertices as \emph{black vertices}. To make this distinction clear, we picture them as $\circ$ and
$\bullet$ respectively.
\end{defn}

Note that there is a bijection between the shuffles of two linear trees $L_n$ and $L_m$ and the $(n,m)$-shuffles of $[n]\times [m]$.

\begin{exmp}
Let $S$ and $T$ be the trees
$$
\xy<0.08cm, 0cm>:
(0,0)*=0{}="1";
(0,10)*\cir<2pt>{}="2";
(-10,20)*=0{}="3";
(10,20)*\cir<2pt>{}="4";
(0,30)*=0{}="5";
(10,30)*=0{}="6";
(20,30)*=0{}="7";
(-8,15)*=0{\scriptstyle b};
(8,15)*=0{\scriptstyle c};
(2,5)*=0{\scriptstyle a};
(2,25)*=0{\scriptstyle d};
(18,25)*=0{\scriptstyle f};
(-10,0)*=0{S};
"1";"2" **\dir{-};
"2";"3" **\dir{-};
"2";"4" **\dir{-};
"4";"5" **\dir{-};
{\ar@{-}(10,21)*{}; (10,23.5)*{}};
(10,25)*=0{\scriptstyle e};
{\ar@{-}(10,26.5)*{}; (10,30)*{}};
%\makebox(0.2cm,6pt){}
"4";"7" **\dir{-};
(50,0)*=0{}="1";
(50,10)*=0{\bullet}="2";
(50,20)*=0{}="3";
(52,5)*=0{\scriptstyle x};
(52,15)*=0{\scriptstyle y};
(40,0)*=0{T};
"1";"2" **\dir{-};
"2";"3" **\dir{-};
\endxy
$$
Then the set of shuffles of $S$ and $T$ consists of the following three trees:
$$
\xy<0.08cm, 0cm>:
(0,0)*=0{}="1";
(0,10)*\cir<2pt>{}="2";
(-10,20)*=0{\bullet}="3";
(-10,30)*=0{}="4";
(10,20)*\cir<2pt>{}="5";
(0,30)*=0{\bullet}="6";
(10,30)*=0{\bullet}="7";
(20,30)*=0{\bullet}="8";
(0,40)*=0{}="9";
(10,40)*=0{}="10";
(20,40)*=0{}="11";
(5,5)*=0{\scriptstyle (a,x)};
(-5,35)*=0{\scriptstyle (d,y)};
(10,42)*=0{\scriptstyle (e,y)};
(25,35)*=0{\scriptstyle (f,y)};
(-15,25)*=0{\scriptstyle (b,y)};
(-11,15)*=0{\scriptstyle (b,x)};
(11,15)*=0{\scriptstyle (c,x)};
(21,25)*=0{\scriptstyle (f,x)};
(-1,25)*=0{\scriptstyle (d,x)};
"1";"2" **\dir{-};
"2";"3" **\dir{-};
"3";"4" **\dir{-};
"2";"5" **\dir{-};
"5";"6" **\dir{-};
{\ar@{-}|{(e,x)}(10,21)*{}; (10,30)*{}};
"5";"8" **\dir{-};
"6";"9" **\dir{-};
"7";"10" **\dir{-};
"8";"11" **\dir{-};
(50,0)*=0{}="1";
(50,10)*\cir<2pt>{}="2";
(40,20)*=0{\bullet}="3";
(40,30)*=0{}="4";
(60,20)*=0{\bullet}="5";
(60,30)*\cir<2pt>{}="6";
(50,40)*=0{}="7";
(60,40)*=0{}="8";
(70,40)*=0{}="9";
(55,5)*=0{\scriptstyle (a,x)};
(60,42)*=0{\scriptstyle (e,y)};
(35,25)*=0{\scriptstyle (b,y)};
(65,25)*=0{\scriptstyle (c,y)};
(39,15)*=0{\scriptstyle (b,x)};
(61,15)*=0{\scriptstyle (c,x)};
(71,35)*=0{\scriptstyle (f,y)};
(49,35)*=0{\scriptstyle (d,y)};
"1";"2" **\dir{-};
"2";"3" **\dir{-};
"3";"4" **\dir{-};
"2";"5" **\dir{-};
"5";"6" **\dir{-};
"6";"7" **\dir{-};
"6";"8" **\dir{-};
"6";"9" **\dir{-};
(100,0)*=0{}="1";
(100,10)*=0{\bullet}="2";
(100,20)*\cir<2pt>{}="3";
(90,30)*=0{}="4";
(110,30)*\cir<2pt>{}="5";
(100,40)*=0{}="6";
(110,40)*=0{}="7";
(120,40)*=0{}="8";
(105,15)*=0{\scriptstyle (a,y)};
(105,5)*=0{\scriptstyle (a,x)};
(89,25)*=0{\scriptstyle (b,y)};
(111,25)*=0{\scriptstyle (c,y)};
(99,35)*=0{\scriptstyle (d,y)};
(110,42)*=0{\scriptstyle (e,y)};
(121,35)*=0{\scriptstyle (f,y)};
"1";"2" **\dir{-};
"2";"3" **\dir{-};
"3";"4" **\dir{-};
"3";"5" **\dir{-};
"5";"6" **\dir{-};
"5";"7" **\dir{-};
"5";"8" **\dir{-};
(-10,0)*=0{R_1};
(40,0)*=0{R_2};
(90,0)*=0{R_3};
\endxy
$$
\end{exmp}
%$$
%\xy<0.08cm, 0cm>:
%(0,0)*=0{}="1";
%(0,10)*=0{\bullet}="2";
%(0,20)*\cir<2pt>{}="3";
%(-10,30)*=0{}="4";
%(0,30)*=0{}="5";
%(10,30)*=0{}="6";
%(5,15)*=0{\scriptstyle (c, y)};
%(5,5)*=0{\scriptstyle (c,x)};
%(0,32)*=0{\scriptstyle (e,y)};
%(-11,25)*=0{\scriptstyle (d,y)};
%(11,25)*=0{\scriptstyle (f,y)};
%"1";"2" **\dir{-};
%"2";"3" **\dir{-};
%"3";"4" **\dir{-};
%"3";"5" **\dir{-};
%"3";"6" **\dir{-};
%(40,0)*=0{}="1";
%(40,10)*\cir<2pt>{}="2";
%(30,20)*==0{\bullet}="3";
%(50,20)*=0{}="4";
%(30,30)*=0{}="5";
%(45,5)*=0{\scriptstyle (a,x)};
%(25,25)*=0{\scriptstyle (b,y)};
%(29,15)*=0{\scriptstyle (b,x)};
%(51,15)*=0{\scriptstyle (c,x)};
%"1";"2" **\dir{-};
%"2";"3" **\dir{-};
%"2";"4" **\dir{-};
%"3";"5" **\dir{-};
%(80,0)*=0{}="1";
%(80,10)*=0{\bullet}="2";
%(80,20)*\cir<2pt>{}="3";
%(70,30)*=0{}="4";
%(90,30)*\cir<2pt>{}="5";
%(80,40)*=0{}="6";
%(90,40)*=0{}="7";
%(100,40)*=0{}="8";
%(85,15)*=0{\scriptstyle (a,y)};
%(85,5)*=0{\scriptstyle (a,x)};
%(69,25)*=0{\scriptstyle (b,y)};
%(91,25)*=0{\scriptstyle (c,y)};
%(79,35)*=0{\scriptstyle (d,y)};
%(90,42)*=0{\scriptstyle (e,y)};
%(101,35)*=0{\scriptstyle (f,y)};
%"1";"2" **\dir{-};
%"2";"3" **\dir{-};
%"3";"4" **\dir{-};
%"3";"5" **\dir{-};
%"5";"6" **\dir{-};
%"5";"7" **\dir{-};
%"5";"8" **\dir{-};
%\endxy
%$$

The set of shuffles of two trees $S$ and $T$ has a natural partial order. The minimal tree $R_1$ in this poset is the one obtained by stacking a copy of the black tree $T$ on top of each of the inputs of the white tree $S$. More precisely, on the bottom of $R_1$ there is a copy $S\otimes r_T$ of the tree $S$ all whose edges are renamed $(a, r_T)$ where $r_T$ is the output edge at the root of $T$. For each input edge $b$ of $S$, a copy of $T$ is grafted on the edge $(b,r_T)$ of $S\otimes r_T$, with edges $x$ in $T$ renamed $(b,x)$. The maximal tree $R_N$ in the poset is the similar tree with copies of the white tree $S$ grafted on each of the input edges of the black tree. Schematically, the trees $R_1$ and $R_N$ look like
$$
\xy<0.1cm, 0cm>:
(0,0)*=0{}="1";
(0,10)*=0{}="2";
(-0,20)*=0{}="a";
(-5,20)*=0{}="3";
(5,20)*=0{}="4";
(-20,30)*=0{}="5";
(20,30)*=0{}="6";
(0,30)*=0{}="b";
(0,16)*=0{S};
"1";"2" **\dir{-};
"2";"3" **\dir{-};
"2";"4" **\dir{-};
"3";"4" **\dir{-};
"3";"5" **\dir{-};
"4";"6" **\dir{-};
"a";"b" **\dir{-};
(-25,35)*=0{}="9";
(-15,35)*=0{}="10";
(-5,35)*=0{}="11";
(5,35)*=0{}="12";
(15,35)*=0{}="15";
(25,35)*=0{}="16";
"5";"9" **\dir{-};
"5";"10" **\dir{-};
"9";"10" **\dir{-};
"b";"11" **\dir{-};
"b";"12" **\dir{-};
"11";"12" **\dir{-};
"6";"15" **\dir{-};
"6";"16" **\dir{-};
"15";"16" **\dir{-};
(-20,33)*=0{T};
(0,33)*=0{T};
(20,33)*=0{T};
(-10,0)*=0{R_1};
\endxy
\qquad
\qquad
\xy<0.1cm, 0cm>:
(0,0)*=0{}="1";
(0,10)*=0{}="2";
(-2.5,20)*=0{}="a";
(2.5,20)*=0{}="b";
(-5,20)*=0{}="3";
(5,20)*=0{}="4";
(-20,30)*=0{}="5";
(20,30)*=0{}="6";
(-5,30)*=0{}="c";
(5,30)*=0{}="d";
(0,16)*=0{T};
"1";"2" **\dir{-};
"2";"3" **\dir{-};
"2";"4" **\dir{-};
"3";"4" **\dir{-};
"3";"5" **\dir{-};
"4";"6" **\dir{-};
"a";"c" **\dir{-};
"b";"d" **\dir{-};
(-25,35)*=0{}="9";
(-15,35)*=0{}="10";
(-10,35)*=0{}="11";
(-2,35)*=0{}="12";
(2,35)*=0{}="13";
(10,35)*=0{}="14";
(15,35)*=0{}="15";
(25,35)*=0{}="16";
"5";"9" **\dir{-};
"5";"10" **\dir{-};
"9";"10" **\dir{-};
"c";"11" **\dir{-};
"c";"12" **\dir{-};
"11";"12" **\dir{-};
"d";"13" **\dir{-};
"d";"14" **\dir{-};
"13";"14" **\dir{-};
"6";"15" **\dir{-};
"6";"16" **\dir{-};
"15";"16" **\dir{-};
(-20,33)*=0{S};
(-5.5,33)*=0{S};
(5.5,33)*=0{S};
(20,33)*=0{S};
(-10,0)*=0{R_N};
\endxy
$$

There are intermediate shuffles $R_k$ $(1<k<N)$ between $R_1$ and $R_N$ obtained by letting the black vertices in $R_1$ slowly percolate in all possible ways towards the root of the tree. Shuffles are also called \emph{percolation schemes}. The \emph{percolation rule} or \emph{percolation relation} can be made explicit as follows. Each $R_k$ is obtained from an earlier $R_l$ by replacing a configuration
\begin{equation}
\xy<0.08cm, 0cm>:
(0,-15)*=0{}="1";
(0,-5)*\cir<2pt>{}="2";
(-20,5)*=0{\bullet}="3";
(-40,15)*=0{}="5";
(20,5)*=0{\bullet}="4";
(40,15)*=0{}="8";
(-15,15)*=0{}="6";
(15,15)*=0{}="7";
"1";"2" **\dir{-};
"2";"3" **\dir{-};
"2";"4" **\dir{-};
"3";"5" **\dir{-};
"3";"6" **\dir{-};
"4";"7" **\dir{-};
"4";"8" **\dir{-};
(0,3)*=0{\ldots\ldots\ldots};
(-23,10)*=0{\ldots};
(23,10)*=0{\ldots};
(5,-10)*=0{\scriptstyle (a,x)};
(16,-2)*=0{\scriptstyle (a, y_m)};
(-16,-2)*=0{\scriptstyle (a, y_1)};
(-36,8)*=0{\scriptstyle (b_1, y_1)};
(36,8)*=0{\scriptstyle (b_n, y_m)};
(-11,8)*=0{\scriptstyle (b_n, y_1)};
(11,8)*=0{\scriptstyle (b_1, y_m)};
\endxy
\label{percrel01}
\end{equation}
in $R_l$ by a configuration
\begin{equation}
\xy<0.08cm, 0cm>:
(0,-15)*=0{}="1";
(0,-5)*=0{\bullet}="2";
(-20,5)*\cir<2pt>{}="3";
(-40,15)*=0{}="5";
(20,5)*\cir<2pt>{}="4";
(40,15)*=0{}="8";
(-15,15)*=0{}="6";
(15,15)*=0{}="7";
"1";"2" **\dir{-};
"2";"3" **\dir{-};
"2";"4" **\dir{-};
"3";"5" **\dir{-};
"3";"6" **\dir{-};
"4";"7" **\dir{-};
"4";"8" **\dir{-};
(0,3)*=0{\ldots\ldots\ldots};
(-23,10)*=0{\ldots};
(23,10)*=0{\ldots};
(5,-10)*=0{\scriptstyle (a,x)};
(16,-2)*=0{\scriptstyle (b_n, x)};
(-16,-2)*=0{\scriptstyle (b_1, x)};
(-36,8)*=0{\scriptstyle (b_1, y_1)};
(36,8)*=0{\scriptstyle (b_n, y_m)};
(-11,8)*=0{\scriptstyle (b_1, y_m)};
(11,8)*=0{\scriptstyle (b_n, y_1)};
\endxy
\label{percrel02}
\end{equation}
in $R_k$. If a shuffle $R_k$ is obtained from another $R_l$ by means of the above rule, then we say that $R_k$ is obtained by a \emph{single percolation step} and denote this by $R_l\le R_k$. This generates a partial order on the set of all shuffles.

It is important to make explicit the percolation relation above for trees with no input edges, i.e., $n=0$ or $m=0$.
If $m=0$ and $n\ne 0$, then we have the relation
$$
\xy<0.08cm, 0cm>:
(0,-5)*{
\xy<0.08cm, 0cm>:
(0,0)*=0{}="1";
(0,10)*\cir<2pt>{}="2";
"1";"2" **\dir{-};
(5,5)*=0{\scriptstyle (a,x)};
\endxy
};
(75,0)*{
\xy<0.08cm, 0cm>:
(0,0)*=0{}="1";
(0,10)*=0{\bullet}="2";
(-20,20)*\cir<2pt>{}="3";
(20,20)*\cir<2pt>{}="4";
"1";"2" **\dir{-};
"2";"3" **\dir{-};
"2";"4" **\dir{-};
(0,18)*=0{\ldots\ldots\ldots};
(5,5)*=0{\scriptstyle (a,x)};
(16,13)*=0{\scriptstyle (b_n, x)};
(-16,13)*=0{\scriptstyle (b_1, x)};
\endxy
};
\ar@{<-}(15,0)*{};(50,0)*{};
\endxy
$$
If $n=0$ and $m\ne 0$, then we have the relation
$$
\xy<0.08cm, 0cm>:
(0,0)*{
\xy<0.08cm, 0cm>:
(0,0)*=0{}="1";
(0,10)*\cir<2pt>{}="2";
(-20,20)*=0{\bullet}="3";
(20,20)*=0{\bullet}="4";
"1";"2" **\dir{-};
"2";"3" **\dir{-};
"2";"4" **\dir{-};
(0,18)*=0{\ldots\ldots\ldots};
(5,5)*=0{\scriptstyle (a,x)};
(16,13)*=0{\scriptstyle (a, y_m)};
(-16,13)*=0{\scriptstyle (a, y_1)};
\endxy
};
(75,-5)*{
\xy<0.08cm, 0cm>:
(0,0)*=0{}="1";
(0,10)*=0{\bullet}="2";
"1";"2" **\dir{-};
(5,5)*=0{\scriptstyle (a,x)};
\endxy
};
\ar@{<-}(25,0)*{};(60,0)*{};
\endxy
$$
And if $n=m=0$, we have the relation
$$
\xy<0.08cm, 0cm>:
(0,0)*{
\xy<0.08cm, 0cm>:
(0,0)*=0{}="1";
(0,10)*\cir<2pt>{}="2";
"1";"2" **\dir{-};
(5,5)*=0{\scriptstyle (a,x)};
\endxy
};
(70,0)*{
\xy<0.08cm, 0cm>:
(0,0)*=0{}="1";
(0,10)*=0{\bullet}="2";
"1";"2" **\dir{-};
(5,5)*=0{\scriptstyle (a,x)};
\endxy
};
\ar@{<-}(15,0)*{};(55,0)*{};
\endxy
$$
\begin{exmp}[Taken from {\cite[Example 9.4]{MW09}}]
Let $S$ and $T$ be the following two trees; here, we have singled out
one particular edge $e$ in $S$, we have numbered the edges of $T$
as $1,\ldots,5$, and denoted the colour $(e,i)$ in $R_{j}$ by $e_{i}$.
$$
\xy<0.08cm, 0cm>:
(0,0)*=0{}="1";
(0,10)*\cir<2pt>{}="2";
(0,20)*\cir<2pt>{}="3";
(-10,30)*=0{}="4";
(10,30)*=0{}="6";
(2.5,15)*=0{\scriptstyle e};
"1";"2" **\dir{-};
"2";"3" **\dir{-};
"3";"4" **\dir{-};
"3";"6" **\dir{-};
(50,0)*=0{}="1";
(50,10)*=0{\bullet}="2";
(40,20)*=0{\bullet}="3";
(60,20)*=0{\bullet}="4";
(40,30)*=0{}="5";
(60,30)*=0{}="6";
(52.5,5)*=0{\scriptstyle 1};
(41,15)*=0{\scriptstyle 2};
(58,15)*=0{\scriptstyle 4};
(37.5,25)*=0{\scriptstyle 3};
(62.5,25)*=0{\scriptstyle 5};
"1";"2" **\dir{-};
"2";"3" **\dir{-};
"2";"4" **\dir{-};
"3";"5" **\dir{-};
"4";"6" **\dir{-};
(-10,0)*=0{S};
(40,0)*=0{T};
\endxy
$$

There are fourteen shuffles $R_1,\ldots,R_{14}$  of $S$ and $T$ in this case. Here is the complete list of them:
$$
\xy<0.08cm, 0cm>:
(0,0)*{
\xy<0.05cm, 0cm>:
(0,0)*=0{}="1";
(0,10)*\cir<2pt>{}="2";
(0,20)*\cir<2pt>{}="3";
(-20,30)*=0{\bullet}="4";
(20,30)*=0{\bullet}="5";
(-30,40)*=0{\bullet}="6";
(30,40)*=0{\bullet}="9";
(-30,50)*=0{}="10";
(30,50)*=0{}="13";
(-10,40)*=0{\bullet}="7";
(10,40)*=0{\bullet}="8";
(-10,50)*=0{}="11";
(10,50)*=0{}="12";
(4,15)*=0{\scriptstyle e_1};
"1";"2" **\dir{-};
"2";"3" **\dir{-};
"3";"4" **\dir{-};
"3";"5" **\dir{-};
"4";"6" **\dir{-};
"4";"7" **\dir{-};
"6";"10" **\dir{-};
"7";"11" **\dir{-};
"5";"8" **\dir{-};
"5";"9" **\dir{-};
"8";"12" **\dir{-};
"9";"13" **\dir{-};
(-10,0)*=0{R_1};
\endxy
};
(50,0)*{
\xy<0.05cm, 0cm>:
(0,0)*=0{}="1";
(0,10)*\cir<2pt>{}="2";
(0,20)*=0{\bullet}="3";
(-20,30)*\cir<2pt>{}="4";
(20,30)*\cir<2pt>{}="5";
(-30,40)*=0{\bullet}="6";
(30,40)*=0{\bullet}="9";
(-30,50)*=0{}="10";
(30,50)*=0{}="13";
(-10,40)*=0{\bullet}="7";
(10,40)*=0{\bullet}="8";
(-10,50)*=0{}="11";
(10,50)*=0{}="12";
(4,15)*=0{\scriptstyle e_1};
(16,23)*=0{\scriptstyle e_4};
(-16,23)*=0{\scriptstyle e_2};
"1";"2" **\dir{-};
"2";"3" **\dir{-};
"3";"4" **\dir{-};
"3";"5" **\dir{-};
"4";"6" **\dir{-};
"4";"7" **\dir{-};
"6";"10" **\dir{-};
"7";"11" **\dir{-};
"5";"8" **\dir{-};
"5";"9" **\dir{-};
"8";"12" **\dir{-};
"9";"13" **\dir{-};
(-10,0)*=0{R_2};
\endxy
};
(100,0)*{
\xy<0.05cm, 0cm>:
(0,0)*=0{}="1";
(0,10)*\cir<2pt>{}="2";
(0,20)*=0{\bullet}="3";
(-20,30)*\cir<2pt>{}="4";
(20,30)*=0{\bullet}="5";
(-30,40)*=0{\bullet}="6";
(-30,50)*=0{}="10";
(30,50)*=0{}="13";
(-10,40)*=0{\bullet}="7";
(20,40)*\cir<2pt>{}="8";
(10,50)*=0{}="8a";
(30,50)*=0{}="8b";
(-10,50)*=0{}="11";
(10,50)*=0{}="12";
(4,15)*=0{\scriptstyle e_1};
(16,23)*=0{\scriptstyle e_4};
(-16,23)*=0{\scriptstyle e_2};
(24,35)*=0{\scriptstyle e_5};
"1";"2" **\dir{-};
"2";"3" **\dir{-};
"3";"4" **\dir{-};
"3";"5" **\dir{-};
"4";"6" **\dir{-};
"4";"7" **\dir{-};
"6";"10" **\dir{-};
"7";"11" **\dir{-};
"5";"8" **\dir{-};
"8";"8a" **\dir{-};
"8";"8b" **\dir{-};
(-10,0)*=0{R_3};
\endxy
};
\endxy
$$

$$
\xy<0.08cm, 0cm>:
(0,0)*{
\xy<0.05cm, 0cm>:
(0,0)*=0{}="1";
(0,10)*\cir<2pt>{}="2";
(0,20)*=0{\bullet}="3";
(-20,30)*=0{\bullet}="4";
(20,30)*\cir<2pt>{}="5";
(-20,40)*\cir<2pt>{}="6";
(-30,50)*=0{}="6a";
(-10,50)*=0{}="6b";
(30,40)*=0{\bullet}="9";
(-30,50)*=0{}="10";
(30,50)*=0{}="13";
(10,40)*=0{\bullet}="8";
(-10,50)*=0{}="11";
(10,50)*=0{}="12";
(4,15)*=0{\scriptstyle e_1};
(16,23)*=0{\scriptstyle e_4};
(-16,23)*=0{\scriptstyle e_2};
(-24,35)*=0{\scriptstyle e_3};
"1";"2" **\dir{-};
"2";"3" **\dir{-};
"3";"4" **\dir{-};
"3";"5" **\dir{-};
"4";"6" **\dir{-};
"6";"6a" **\dir{-};
"6";"6b" **\dir{-};
"5";"8" **\dir{-};
"5";"9" **\dir{-};
"8";"12" **\dir{-};
"9";"13" **\dir{-};
(-10,0)*=0{R_4};
\endxy
};
(50,0)*{
\xy<0.05cm, 0cm>:
(0,0)*=0{}="1";
(0,10)*\cir<2pt>{}="2";
(0,20)*=0{\bullet}="3";
(-20,30)*=0{\bullet}="4";
(20,30)*=0{\bullet}="5";
(-20,40)*\cir<2pt>{}="6";
(-30,50)*=0{}="6a";
(-10,50)*=0{}="6b";
(20,40)*\cir<2pt>{}="8";
(10,50)*=0{}="8a";
(30,50)*=0{}="8b";
(4,15)*=0{\scriptstyle e_1};
(16,23)*=0{\scriptstyle e_4};
(-16,23)*=0{\scriptstyle e_2};
(-24,35)*=0{\scriptstyle e_3};
(24,35)*=0{\scriptstyle e_5};
"1";"2" **\dir{-};
"2";"3" **\dir{-};
"3";"4" **\dir{-};
"3";"5" **\dir{-};
"4";"6" **\dir{-};
"6";"6a" **\dir{-};
"6";"6b" **\dir{-};
"5";"8" **\dir{-};
"8";"8a" **\dir{-};
"8";"8b" **\dir{-};
(-10,0)*=0{R_5};
\endxy
};
(100,0)*{
\xy<0.05cm, 0cm>:
(0,0)*=0{}="1";
(0,10)*=0{\bullet}="2";
(-20,20)*\cir<2pt>{}="3";
(20,20)*\cir<2pt>{}="4";
(-20,30)*\cir<2pt>{}="5";
(20,30)*\cir<2pt>{}="6";
(-30,40)*=0{\bullet}="7";
(30,40)*=0{\bullet}="10";
(-30,50)*=0{}="11";
(30,50)*=0{}="14";
(-10,40)*=0{\bullet}="8";
(10,40)*=0{\bullet}="9";
(-10,50)*=0{}="12";
(10,50)*=0{}="13";
(-24,25)*=0{\scriptstyle e_2};
(24,25)*=0{\scriptstyle e_4};
"1";"2" **\dir{-};
"2";"3" **\dir{-};
"2";"4" **\dir{-};
"3";"5" **\dir{-};
"5";"7" **\dir{-};
"5";"8" **\dir{-};
"7";"11" **\dir{-};
"8";"12" **\dir{-};
"4";"6" **\dir{-};
"6";"9" **\dir{-};
"6";"10" **\dir{-};
"9";"13" **\dir{-};
"10";"14" **\dir{-};
(-10,0)*=0{R_6};
\endxy
};
\endxy
$$

$$
\xy<0.08cm, 0cm>:
(0,0)*{
\xy<0.05cm, 0cm>:
(0,0)*=0{}="1";
(0,10)*=0{\bullet}="2";
(-20,20)*\cir<2pt>{}="3";
(20,20)*\cir<2pt>{}="4";
(-20,30)*\cir<2pt>{}="5";
(20,30)*=0{\bullet}="6";
(-30,40)*=0{\bullet}="7";
(-30,50)*=0{}="11";
(30,50)*=0{}="9b";
(-10,40)*=0{\bullet}="8";
(20,40)*\cir<2pt>{}="9";
(-10,50)*=0{}="12";
(10,50)*=0{}="9a";
(-24,25)*=0{\scriptstyle e_2};
(24,35)*=0{\scriptstyle e_5};
"1";"2" **\dir{-};
"2";"3" **\dir{-};
"2";"4" **\dir{-};
"3";"5" **\dir{-};
"5";"7" **\dir{-};
"5";"8" **\dir{-};
"7";"11" **\dir{-};
"8";"12" **\dir{-};
"4";"6" **\dir{-};
"6";"9" **\dir{-};
"9";"9a" **\dir{-};
"9";"9b" **\dir{-};
(-10,0)*=0{R_7};
\endxy
};
(50,0)*{
\xy<0.05cm, 0cm>:
(0,0)*=0{}="1";
(0,10)*=0{\bullet}="2";
(-20,20)*\cir<2pt>{}="3";
(20,20)*=0{\bullet}="4";
(-20,30)*\cir<2pt>{}="5";
(20,30)*\cir<2pt>{}="6";
(-30,40)*=0{\bullet}="7";
(-30,50)*=0{}="11";
(30,50)*=0{}="9b";
(-10,40)*=0{\bullet}="8";
(20,40)*\cir<2pt>{}="9";
(-10,50)*=0{}="12";
(10,50)*=0{}="9a";
(-24,25)*=0{\scriptstyle e_2};
(24,35)*=0{\scriptstyle e_5};
"1";"2" **\dir{-};
"2";"3" **\dir{-};
"2";"4" **\dir{-};
"3";"5" **\dir{-};
"5";"7" **\dir{-};
"5";"8" **\dir{-};
"7";"11" **\dir{-};
"8";"12" **\dir{-};
"4";"6" **\dir{-};
"6";"9" **\dir{-};
"9";"9a" **\dir{-};
"9";"9b" **\dir{-};
(-10,0)*=0{R_8};
\endxy
};
(100,0)*{
\xy<0.05cm, 0cm>:
(0,0)*=0{}="1";
(0,10)*=0{\bullet}="2";
(-20,20)*\cir<2pt>{}="3";
(20,20)*\cir<2pt>{}="4";
(-20,30)*=0{\bullet}="5";
(20,30)*\cir<2pt>{}="6";
(-20,40)*\cir<2pt>{}="7";
(30,40)*=0{\bullet}="10";
(-30,50)*=0{}="7a";
(30,50)*=0{}="14";
(10,40)*=0{\bullet}="9";
(-10,50)*=0{}="7b";
(10,50)*=0{}="13";
%(-24,25)*=0{\scriptstyle e_2};
%(24,25)*=0{\scriptstyle e_4};
"1";"2" **\dir{-};
"2";"3" **\dir{-};
"2";"4" **\dir{-};
"3";"5" **\dir{-};
"5";"7" **\dir{-};
"7";"7a" **\dir{-};
"7";"7b" **\dir{-};
"4";"6" **\dir{-};
"6";"9" **\dir{-};
"6";"10" **\dir{-};
"9";"13" **\dir{-};
"10";"14" **\dir{-};
(-10,0)*=0{R_9};
\endxy
};
\endxy
$$

$$
\xy<0.08cm, 0cm>:
(0,0)*{
\xy<0.05cm, 0cm>:
(0,0)*=0{}="1";
(0,10)*=0{\bullet}="2";
(-20,20)*\cir<2pt>{}="3";
(20,20)*\cir<2pt>{}="4";
(-20,30)*=0{\bullet}="5";
(20,30)*=0{\bullet}="6";
(-20,40)*\cir<2pt>{}="7";
(-30,50)*=0{}="7a";
(30,50)*=0{}="9b";
(20,40)*\cir<2pt>{}="9";
(-10,50)*=0{}="7b";
(10,50)*=0{}="9a";
%(-24,25)*=0{\scriptstyle e_2};
%(24,35)*=0{\scriptstyle e_5};
"1";"2" **\dir{-};
"2";"3" **\dir{-};
"2";"4" **\dir{-};
"3";"5" **\dir{-};
"5";"7" **\dir{-};
"7";"7a" **\dir{-};
"7";"7b" **\dir{-};
"4";"6" **\dir{-};
"6";"9" **\dir{-};
"9";"9a" **\dir{-};
"9";"9b" **\dir{-};
(-10,0)*=0{R_{10}};
\endxy
};
(50,0)*{
\xy<0.05cm, 0cm>:
(0,0)*=0{}="1";
(0,10)*=0{\bullet}="2";
(-20,20)*\cir<2pt>{}="3";
(20,20)*=0{\bullet}="4";
(-20,30)*=0{\bullet}="5";
(20,30)*\cir<2pt>{}="6";
(-20,40)*\cir<2pt>{}="7";
(-30,50)*=0{}="7a";
(30,50)*=0{}="9b";
(20,40)*\cir<2pt>{}="9";
(-10,50)*=0{}="7b";
(10,50)*=0{}="9a";
(-24,25)*=0{\scriptstyle e_2};
(-24,35)*=0{\scriptstyle e_3};
(24,35)*=0{\scriptstyle e_5};
"1";"2" **\dir{-};
"2";"3" **\dir{-};
"2";"4" **\dir{-};
"3";"5" **\dir{-};
"5";"7" **\dir{-};
"7";"7a" **\dir{-};
"7";"7b" **\dir{-};
"4";"6" **\dir{-};
"6";"9" **\dir{-};
"9";"9a" **\dir{-};
"9";"9b" **\dir{-};
(-10,0)*=0{R_{11}};
\endxy
};
(100,0)*{
\xy<0.05cm, 0cm>:
(0,0)*=0{}="1";
(0,10)*=0{\bullet}="2";
(-20,20)*=0{\bullet}="3";
(20,20)*\cir<2pt>{}="4";
(-20,30)*\cir<2pt>{}="5";
(20,30)*\cir<2pt>{}="6";
(-20,40)*\cir<2pt>{}="7";
(30,40)*=0{\bullet}="10";
(-30,50)*=0{}="7a";
(30,50)*=0{}="14";
(10,40)*=0{\bullet}="9";
(-10,50)*=0{}="7b";
(10,50)*=0{}="13";
(-24,35)*=0{\scriptstyle e_3};
(24,25)*=0{\scriptstyle e_4};
"1";"2" **\dir{-};
"2";"3" **\dir{-};
"2";"4" **\dir{-};
"3";"5" **\dir{-};
"5";"7" **\dir{-};
"7";"7a" **\dir{-};
"7";"7b" **\dir{-};
"4";"6" **\dir{-};
"6";"9" **\dir{-};
"6";"10" **\dir{-};
"9";"13" **\dir{-};
"10";"14" **\dir{-};
(-10,0)*=0{R_{12}};
\endxy
};
\endxy
$$

$$
\xy<0.08cm, 0cm>:
(0,0)*{
\xy<0.05cm, 0cm>:
(0,0)*=0{}="1";
(0,10)*=0{\bullet}="2";
(-20,20)*=0{\bullet}="3";
(20,20)*\cir<2pt>{}="4";
(-20,30)*\cir<2pt>{}="5";
(20,30)*=0{\bullet}="6";
(-20,40)*\cir<2pt>{}="7";
(-30,50)*=0{}="7a";
(30,50)*=0{}="9b";
(20,40)*\cir<2pt>{}="9";
(-10,50)*=0{}="7b";
(10,50)*=0{}="9a";
(-24,35)*=0{\scriptstyle e_3};
(24,35)*=0{\scriptstyle e_5};
(24,25)*=0{\scriptstyle e_4};
"1";"2" **\dir{-};
"2";"3" **\dir{-};
"2";"4" **\dir{-};
"3";"5" **\dir{-};
"5";"7" **\dir{-};
"7";"7a" **\dir{-};
"7";"7b" **\dir{-};
"4";"6" **\dir{-};
"6";"9" **\dir{-};
"9";"9a" **\dir{-};
"9";"9b" **\dir{-};
(-10,0)*=0{R_{13}};
\endxy
};
(50,0)*{
\xy<0.05cm, 0cm>:
(0,0)*=0{}="1";
(0,10)*=0{\bullet}="2";
(-20,20)*=0{\bullet}="3";
(20,20)*=0{\bullet}="4";
(-20,30)*\cir<2pt>{}="5";
(20,30)*\cir<2pt>{}="6";
(-20,40)*\cir<2pt>{}="7";
(-30,50)*=0{}="7a";
(30,50)*=0{}="9b";
(20,40)*\cir<2pt>{}="9";
(-10,50)*=0{}="7b";
(10,50)*=0{}="9a";
"1";"2" **\dir{-};
"2";"3" **\dir{-};
"2";"4" **\dir{-};
"3";"5" **\dir{-};
"5";"7" **\dir{-};
"7";"7a" **\dir{-};
"7";"7b" **\dir{-};
"4";"6" **\dir{-};
"6";"9" **\dir{-};
"9";"9a" **\dir{-};
"9";"9b" **\dir{-};
(-10,0)*=0{R_{14}};
\endxy
};
\endxy
$$
The poset structure on the shuffles above is:
$$
\xymatrixrowsep{2.5pt}
\xymatrixcolsep{5pt}
\xymatrix{ &  & R_{1}\ar@{-}[dd]\\
\\ &  & R_{2}\ar@{-}[ddrr]\ar@{-}[dd]\ar@{-}[ddll]\\
\\R_{3}\ar@{-}[dd]\ar@{-}[ddrr] &  & R_{6}\ar@{-}'[dl][ddll]\ar@{-}'[dr][ddrr] &  & R_{4}\ar@{-}[dd]\ar@{-}[ddll]\\
 & \, &  & \,\\
R_{7}\ar@{-}[ddrr]\ar@{-}[dd] &  & R_{5}\ar@{-}[dd] &  & R_{9}\ar@{-}[ddll]\ar@{-}[dd]\\
 & \,\\
R_{8}\ar@{-}[ddr] &  & R_{10}\ar@{-}[ddr]\ar@{-}[ddl] &  & R_{12}\ar@{-}[ddl]\\
\\ & R_{11}\ar@{-}[ddr] &  & R_{13}\ar@{-}[ddl]\\
\\ &  & R_{14}}
$$
\end{exmp}

\begin{lem}
Every shuffle $R_i$ of $S$ and $T$ is equipped with a canonical monomorphism
$$
\xymatrix@C-0.3cm{m\colon \Omega[R_i]\;\ar@{>->}[r] & \,\Omega[S]\otimes \Omega[T].}
%m\colon \Omega[R_i]\rightarrowtail \Omega[S]\otimes \Omega[T].
$$
The dendroidal subset given by the image of this monomorphism will be denoted by $m(R_i)$.
\end{lem}
\begin{proof}
The vertices of the dendroidal set $\Omega[R_i]$ are the edges of the tree $R_i$. The map $m$ is completely determined by asking it to map an edge named $(a,x)$ in $R_i$ to the vertex with the same name in $\Omega[S]\otimes \Omega[T]$. This map is a monomorphism. Indeed, any map
$$
\Omega[R]\longrightarrow X
$$
from a representable dendroidal set to an arbitrary one is a monomorphism as soon as the map $\Omega[R]_{|}\longrightarrow X_{|}$ on vertices is.
\end{proof}
\begin{cor}
For any two objects $T$ and $S$ in $\Omega$, we have that
\begin{equation}
\Omega[S]\otimes\Omega[T]=\bigcup_{i=1}^N m(R_i),
\label{shufflesformula}
\end{equation}
where the union is taken over all possible shuffles of $S$ and $T$.
\end{cor}

The Boardman--Vogt relation says that if $R_k$ is obtained from $R_l$ by a single percolation step as above, then the image
under $m$ of the face of $\Omega[R_k]$ obtained by contracting all the edges $(b_1, x),\ldots, (b_n,x)$ in~(\ref{percrel02})
above coincides (as a subobject of $\Omega[S]\otimes\Omega[T]$) with the image of the face of $\Omega[R_l]$ obtained by contracting
the edges $(a,y_1),\ldots, (a,y_m)$ in~(\ref{percrel01}).

The following example illustrates that in the set of shuffles appearing in (\ref{shufflesformula}) some of them can be faces of others
when one of the trees has vertices of valence zero. Thus, not all the shuffles are always needed in the union~(\ref{shufflesformula}).

\begin{exmp}
Let $S$ and $T$ be the following two trees and observe that the tree $S$ has a vertex of valence zero.
$$
\xy<0.08cm, 0cm>:
(0,0)*=0{}="1";
(0,10)*\cir<2pt>{}="2";
(0,20)*\cir<2pt>{}="3";
(-10,20)*=0{}="4";
(10,20)*=0{}="5";
(2.5,5)*=0{\scriptstyle a};
(-5.5,13)*=0{\scriptstyle b};
(6.5,13)*=0{\scriptstyle d};
(2,16)*=0{\scriptstyle c};
"1";"2" **\dir{-};
"2";"3" **\dir{-};
"2";"4" **\dir{-};
"2";"5" **\dir{-};
(50,0)*=0{}="1";
(50,10)*=0{\bullet}="2";
(40,20)*=0{}="3";
(60,20)*=0{\bullet}="4";
(60,30)*=0{}="5";
(52.5,5)*=0{\scriptstyle 1};
(44.5,13)*=0{\scriptstyle 2};
(56.5,13)*=0{\scriptstyle 3};
(62.5,25)*=0{\scriptstyle 4};
"1";"2" **\dir{-};
"2";"3" **\dir{-};
"2";"4" **\dir{-};
"4";"5" **\dir{-};
(-10,0)*=0{S};
(40,0)*=0{T};
\endxy
$$
There are six shuffles $R_1,\ldots, R_6$ of $S$ and $T$. The colours $(e,k)$ of the edges in $R_i$ are denoted by $e_k$.
$$
\xy<0.08cm, 0cm>:
(0,0)*{
\xy<0.06cm, 0cm>:
(0,0)*=0{}="1";
(0,10)*\cir<2pt>{}="2";
(-30,20)*=0{\bullet}="3";
(0,20)*\cir<2pt>{}="4";
(30,20)*=0{\bullet}="5";
(-40,30)*=0{}="6";
(-20,30)*=0{\bullet}="7";
(20,30)*=0{}="8";
(40,30)*=0{\bullet}="9";
(-20,40)*=0{}="10";
(40,40)*=0{}="11";
"1";"2" **\dir{-};
"2";"3" **\dir{-};
"2";"4" **\dir{-};
"2";"5" **\dir{-};
"3";"6" **\dir{-};
"3";"7" **\dir{-};
"5";"8" **\dir{-};
"5";"9" **\dir{-};
"7";"10" **\dir{-};
"9";"11" **\dir{-};
(4.5,5)*=0{\scriptstyle a_1};
(15,12)*=0{\scriptstyle d_1};
(-15,12)*=0{\scriptstyle b_1};
(4,15)*=0{\scriptstyle c_1};
(-38,23)*=0{\scriptstyle b_2};
(-22,23)*=0{\scriptstyle b_3};
(22,23)*=0{\scriptstyle d_2};
(38,23)*=0{\scriptstyle d_3};
(-16,35)*=0{\scriptstyle b_4};
(44,35)*=0{\scriptstyle d_4};
(-10,0)*=0{R_1};
\endxy
};
(80,0)*{
\xy<0.06cm, 0cm>:
(0,0)*=0{}="1";
(0,10)*\cir<2pt>{}="2";
(-30,20)*=0{\bullet}="3";
(0,20)*=0{\bullet}="4";
(-10,30)*\cir<2pt>{}="4a";
(10,30)*\cir<2pt>{}="4b";
(30,20)*=0{\bullet}="5";
(-40,30)*=0{}="6";
(-20,30)*=0{\bullet}="7";
(20,30)*=0{}="8";
(40,30)*=0{\bullet}="9";
(-20,40)*=0{}="10";
(40,40)*=0{}="11";
"1";"2" **\dir{-};
"2";"3" **\dir{-};
"2";"4" **\dir{-};
"4";"4a" **\dir{-};
"4";"4b" **\dir{-};
"2";"5" **\dir{-};
"3";"6" **\dir{-};
"3";"7" **\dir{-};
"5";"8" **\dir{-};
"5";"9" **\dir{-};
"7";"10" **\dir{-};
"9";"11" **\dir{-};
(4.5,5)*=0{\scriptstyle a_1};
(15,12)*=0{\scriptstyle d_1};
(-15,12)*=0{\scriptstyle b_1};
(4,15)*=0{\scriptstyle c_1};
(8,23)*=0{\scriptstyle c_3};
(-8,23)*=0{\scriptstyle c_2};
(-38,23)*=0{\scriptstyle b_2};
(-22,23)*=0{\scriptstyle b_3};
(22,23)*=0{\scriptstyle d_2};
(38,23)*=0{\scriptstyle d_3};
(-16,35)*=0{\scriptstyle b_4};
(44,35)*=0{\scriptstyle d_4};
(-10,0)*=0{R_2};
\endxy
};
\endxy
$$

$$
\xy<0.08cm, 0cm>:
(0,0)*{
\xy<0.06cm, 0cm>:
(0,0)*=0{}="1";
(0,10)*=0{\bullet}="2";
(-30,20)*\cir<2pt>{}="3";
(30,20)*\cir<2pt>{}="4";
(-40,30)*=0{}="5";
(-30,30)*\cir<2pt>{}="6";
(-20,30)*=0{}="7";
(20,30)*=0{\bullet}="8";
(30,30)*\cir<2pt>{}="9";
(40,30)*=0{\bullet}="10";
(20,40)*=0{}="11";
(40,40)*=0{}="12";
"1";"2" **\dir{-};
"2";"3" **\dir{-};
"2";"4" **\dir{-};
"3";"5" **\dir{-};
"3";"6" **\dir{-};
"3";"7" **\dir{-};
"4";"8" **\dir{-};
"4";"9" **\dir{-};
"4";"10" **\dir{-};
"8";"11" **\dir{-};
"10";"12" **\dir{-};
(4.5,5)*=0{\scriptstyle a_1};
(15,12)*=0{\scriptstyle a_3};
(-15,12)*=0{\scriptstyle a_2};
(-38,23)*=0{\scriptstyle b_2};
(-22,23)*=0{\scriptstyle d_2};
(22,23)*=0{\scriptstyle b_3};
(38,23)*=0{\scriptstyle d_3};
(44,35)*=0{\scriptstyle d_4};
(-27,27)*=0{\scriptstyle c_2};
(33,27)*=0{\scriptstyle c_3};
(16,35)*=0{\scriptstyle b_4};
(-10,0)*=0{R_3};
\endxy
};
(80,0)*{
\xy<0.06cm, 0cm>:
(0,0)*=0{}="1";
(0,10)*\cir<2pt>{}="2";
(-30,20)*=0{\bullet}="3";
(0,20)*=0{\bullet}="4";
(-10,30)*\cir<2pt>{}="4a";
(10,30)*=0{\bullet}="4b";
(10,40)*\cir<2pt>{}="4b1";
(30,20)*=0{\bullet}="5";
(-40,30)*=0{}="6";
(-20,30)*=0{\bullet}="7";
(20,30)*=0{}="8";
(40,30)*=0{\bullet}="9";
(-20,40)*=0{}="10";
(40,40)*=0{}="11";
"1";"2" **\dir{-};
"2";"3" **\dir{-};
"2";"4" **\dir{-};
"4";"4a" **\dir{-};
"4";"4b" **\dir{-};
"4b";"4b1" **\dir{-};
"2";"5" **\dir{-};
"3";"6" **\dir{-};
"3";"7" **\dir{-};
"5";"8" **\dir{-};
"5";"9" **\dir{-};
"7";"10" **\dir{-};
"9";"11" **\dir{-};
(4.5,5)*=0{\scriptstyle a_1};
(15,12)*=0{\scriptstyle d_1};
(-15,12)*=0{\scriptstyle b_1};
(4,15)*=0{\scriptstyle c_1};
(8,23)*=0{\scriptstyle c_3};
(-8,23)*=0{\scriptstyle c_2};
(-38,23)*=0{\scriptstyle b_2};
(-22,23)*=0{\scriptstyle b_3};
(22,23)*=0{\scriptstyle d_2};
(38,23)*=0{\scriptstyle d_3};
(-16,35)*=0{\scriptstyle b_4};
(14,35)*=0{\scriptstyle c_4};
(44,35)*=0{\scriptstyle d_4};
(-10,0)*=0{R_4};
\endxy
};
\endxy
$$

$$
\xy<0.08cm, 0cm>:
(0,0)*{
\xy<0.06cm, 0cm>:
(0,0)*=0{}="1";
(0,10)*=0{\bullet}="2";
(-30,20)*\cir<2pt>{}="3";
(30,20)*\cir<2pt>{}="4";
(-40,30)*=0{}="5";
(-30,30)*\cir<2pt>{}="6";
(-20,30)*=0{}="7";
(20,30)*=0{\bullet}="8";
(30,30)*=0{\bullet}="9";
(30,40)*\cir<2pt>{}="9a";
(40,30)*=0{\bullet}="10";
(20,40)*=0{}="11";
(40,40)*=0{}="12";
"1";"2" **\dir{-};
"2";"3" **\dir{-};
"2";"4" **\dir{-};
"3";"5" **\dir{-};
"3";"6" **\dir{-};
"3";"7" **\dir{-};
"4";"8" **\dir{-};
"4";"9" **\dir{-};
"9";"9a" **\dir{-};
"4";"10" **\dir{-};
"8";"11" **\dir{-};
"10";"12" **\dir{-};
(4.5,5)*=0{\scriptstyle a_1};
(15,12)*=0{\scriptstyle a_3};
(-15,12)*=0{\scriptstyle a_2};
(-38,23)*=0{\scriptstyle b_2};
(-22,23)*=0{\scriptstyle d_2};
(22,23)*=0{\scriptstyle b_3};
(38,23)*=0{\scriptstyle d_3};
(44,35)*=0{\scriptstyle d_4};
(-27,27)*=0{\scriptstyle c_2};
(33,27)*=0{\scriptstyle c_3};
(16,35)*=0{\scriptstyle b_4};
(34,35)*=0{\scriptstyle c_4};
(-10,0)*=0{R_5};
\endxy
};
(80,0)*{
\xy<0.06cm, 0cm>:
(0,0)*=0{}="1";
(0,10)*=0{\bullet}="2";
(-30,20)*\cir<2pt>{}="3";
(30,20)*=0{\bullet}="4";
(-40,30)*=0{}="5";
(-30,30)*\cir<2pt>{}="6";
(-20,30)*=0{}="7";
(30,30)*\cir<2pt>{}="8";
(20,40)*=0{}="9";
(30,40)*\cir<2pt>{}="10";
(40,40)*=0{}="11";
"1";"2" **\dir{-};
"2";"3" **\dir{-};
"2";"4" **\dir{-};
"3";"5" **\dir{-};
"3";"6" **\dir{-};
"3";"7" **\dir{-};
"4";"8" **\dir{-};
"8";"9" **\dir{-};
"8";"10" **\dir{-};
"8";"11" **\dir{-};
(4.5,5)*=0{\scriptstyle a_1};
(15,12)*=0{\scriptstyle a_3};
(-15,12)*=0{\scriptstyle a_2};
(-38,23)*=0{\scriptstyle b_2};
(-22,23)*=0{\scriptstyle d_2};
(-27,27)*=0{\scriptstyle c_2};
(34,25)*=0{\scriptstyle a_4};
(22,33)*=0{\scriptstyle b_4};
(38,33)*=0{\scriptstyle d_4};
(33,37)*=0{\scriptstyle c_4};
(-10,0)*=0{R_6};
\endxy
};
\endxy
$$

Observe that, in this case, $R_1$ is a face of $R_2$, which is a face of $R_4$. Similarly, $R_3$ is a face of $R_5$. Hence,
$$
\Omega[S]\otimes\Omega[T]=m(R_4)\cup m(R_5)\cup m(R_6).
$$
The poset structure on the shuffles above is the following:
$$
\xymatrixrowsep{15pt}
\xymatrixcolsep{15pt}
\xymatrix{
 & R_1\ar@{-}[d] & \\
 & R_2\ar@{-}[dl] \ar@{-}[dr] & \\
R_3 \ar@{-}[dr] & & R_4 \ar@{-}[dl] \\
 & R_5 \ar@{-}[d] & \\
 & R_6 &
}
$$

\end{exmp}

%\bigskip
%\bigskip
%--------------------------------------------
%\begin{itemize}
%\item Introduce the tensor product of two coloured operads and emphasize the use of the symmetries, except if
%one of the operads has unary operations only.
%\item $dSets$ is a symmetric monoidal category with this tensor product. $pdSets$ is a simplicial category
%and $sSets$ is a cartesian closed category.
%\item Discuss behavior of the functor with respect to tensor product. Preservation of tensor product.
%\item Examples.
%\end{itemize}
