\documentclass[11pt,a4paper,openright,oneside]{article}
\usepackage{amsfonts, amsmath, amssymb,latexsym,amsthm, mathrsfs, enumerate, tikz-cd}
\usepackage{hyperref}
\usepackage[all]{xy}
\usepackage{algorithm}
\usepackage[noend]{algpseudocode}


\SelectTips{cm}{} %to change the tips and tail of the arrows 
% \usepackage[spanish]{babel}
\usepackage{epsfig}

\parskip=5pt
\parindent=15pt
\usepackage[margin=1.2in]{geometry}
\usepackage{graphicx}
\usepackage{listings}
\usepackage[latin1]{inputenc}
\usepackage{fancyhdr}
\usepackage{graphicx}
\graphicspath{{images/}}
\usepackage{blindtext}
\usepackage{subfiles}

\setcounter{page}{0}


\numberwithin{equation}{section}
\newtheorem{teo}{Teorema}[section]
\newtheorem*{teo*}{Teorema}
\newtheorem{prop}[teo]{Proposici\'on}
\newtheorem{corol}[teo]{Corolario}
\newtheorem{lema}[teo]{Lema}
\newtheorem{nota}[teo]{Notaci\'on}

\theoremstyle{definition}
\newtheorem{defi}[teo]{Definici\'on}
\newtheorem{prob}[teo]{Problema}
\newtheorem*{sol}{Soluci\'on}
\newtheorem{ex}[teo]{Ejemplo}
\newtheorem{exs}[teo]{Ejemplos}
\newtheorem{obs}[teo]{Observaci\'on}
\newtheorem{obss}[teo]{Observaciones}

\def\qed{\hfill $\square$}

\renewcommand{\refname}{Bibliograf\'ia}


% Definiciones de funciones matemáticas

\newcommand{\Set}{\mathop{\rm Set}}
\newcommand{\Grp}{\mathop{\rm Grp}}
\newcommand{\Top}{\mathop{\rm Top}}
\newcommand{\Oper}{\mathop{\rm Oper}}

\newenvironment{algoritmo}[1][htb]{%
    \floatname{algorithm}{Algoritmo}% Update algorithm name
   \begin{algorithm}[#1]%
  }{\end{algorithm}}
\renewcommand{\algorithmicrequire}{\textbf{Input:}}
\renewcommand{\algorithmicensure}{\textbf{Output:}}
\newcommand{\Break}{\textbf{break}}
\algblockdefx[ForElse]{ForElse}{EndForElse}{\textbf{else}}{}
\algtext*{EndForElse}

\lhead{}
\lfoot{}
\rhead{}
\cfoot{}
\rfoot{\thepage}
\begin{document}

\bibstyle{plain}
\thispagestyle{empty}
% --- Acaban las configuraciones -----------

% ------------------------------------------
% ------------------------------------------
% ------------------------------------------

% --- Empieza la portada -------------------
\begin{titlepage}
    \begin{center}
        \begin{figure}[htb]
            \begin{center}
                \includegraphics[width=6cm]{matematiquesinformatica-pos-rgb.png}
            \end{center}
        \end{figure}
        \vspace*{1cm}
        \textbf{\LARGE GRADO EN MATEM\'{A}TICAS } \\
        \vspace*{.5cm}
        \textbf{\LARGE Trabajo final de grado} \\
        \vspace*{1.5cm}
        \rule{16cm}{0.1mm}\\
        \begin{Huge}
            \textbf{El producto tensorial de conjuntos dendroidales} \\
        \end{Huge}
        \rule{16cm}{0.1mm}\\
        \vspace{1cm}
        \begin{flushright}
            \textbf{\LARGE Autor: Roger Brasc\'o Garc\'es}
            \vspace*{2cm}
            \renewcommand{\arraystretch}{1.5}
            \begin{tabular}{ll}
                \textbf{\Large Director:}     & \textbf{\Large Dr. Javier J. Guti\'errez }     \\
                \textbf{\Large Realizado en:} & \textbf{\Large  Departamento de Matem\'aticas} \\
                                              & \textbf{\Large e Inform\'atica}                \\
                \textbf{\Large Barcelona,}    & \textbf{\Large 24 de enero de 2022 }
            \end{tabular}
        \end{flushright}
    \end{center}
\end{titlepage}
\newpage
% --- Acaba la portada ---------------------

% ------------------------------------------
% ------------------------------------------
% ------------------------------------------

% --- Empieza el encabezado ----------------
\pagenumbering{roman}

% --- Empieza resumen ----------------------
\section*{Resumen}
The main goal of this project is to understand the tensor product operation in the category of dendroidals sets.
To do so, we introduce the notion of shuffles between two trees, and show how they are used in order to describe the tensor product.
We begin with a quick review about categories, functors and operads, focusing on the basic definitions and constructions.
Then, we introduce the formalism of trees, its relation to coloured operads and the definition of the dendroidal category $\Omega$, which serves as the indexing category for defining dendroidal sets as a presheaf category.
We also show how the dendroidal category $\Omega$ extends the simplicial category $\Delta$ via the inclusion of linear trees, and the relation between simplicial sets and dendroidal sets.
Finally, we develop a Python algorithm that generates the complete set of shuffles between two trees and prints their planar representations.

 {\let\thefootnote\relax\footnote{2010 Mathematics Subject Classification. 18A05, 18A25, 18A40, 18M05, 18M60, 18N50}}
\newpage
% --- Acaba resumen ------------------------

% ------------------------------------------

% --- Empieza agradecimientos --------------
\section*{Agradecimientos}
A todas las personas que me apoyaron e hicieron posible que este trabajo se realice con \'exito.
En especial a mi tutor por compartirme sus conocimientos.
Y para recalcar, toda mi familia y amigos por acompa\~narme en este proceso.

\newpage
% --- Acaba agradecimientos ---------------

% ------------------------------------------

% --- Empieza indice ----------------------
\tableofcontents
\newpage
% --- Acaba indice -------------------------

% --- Acaba el encabezado ------------------

% ------------------------------------------
% ------------------------------------------
% ------------------------------------------

% --- Empiezan las secciones ---------------
\pagenumbering{arabic}
\setcounter{page}{1}

% Nociones previas -------------------------
\subfile{sections/1_nociones_previas.tex}
% ------------------------------------------

% Conjuntos Simpliciales -------------------
\newpage
\subfile{sections/2_conjuntos_simpliciales.tex}
% ------------------------------------------

% Conjuntos Dendroidales -------------------
\newpage
\subfile{sections/3_conjuntos_dendroidales.tex}
% ------------------------------------------

% Shuffle de \'arboles (Naipear, Barajear o mezclar?)
\newpage
\subfile{sections/4_shuffle_arboles.tex}

% Conclusiones -----------------------------
\newpage
\section{Conclusiones}

La idea principal de este trabajo era poder entender el producto tensorial de conjuntos dendroidales, para ello hemos ido siguiendo un hilo que parte de las nociones b\'asicas de categor\'ias y op\'eradas y acaba con una t\'ecnica alternativa para encontrar dicho producto tensorial.

Una vez dadas las nociones b\'asicas, el trabajo sigue con la definici\'on de la categor\'ia simplicial $\Delta$ y la categor\'ia de los conjuntos simpliciales. Dicha categor\'ia es una categor\'ia de prehaces sobre $\Delta$.

Para definir la categor\'ia dendroidal el trabajo se basa en un formalismo de \'arboles como op\'eradas coloreadas. Este formalismo nos permite tener una definici\'on estricta de qu\'e es un \'arbol y los morfismos entre \'arboles.
Con estos morfismos de op\'eradas coloreadas y con los \'arboles planares como objetos obtenemos la categor\'ia $\Omega_p$.
La categor\'ia de \'arboles no planares $\Omega$ viene de la categor\'ia $\Omega_p$ considerando que ahora los \'arboles admiten la acci\'on del grupo sim\'etrico en sus entradas.

Para poder introducir la definici\'on de la categor\'ia de conjuntos dendroidales, tenemos que introducir la noci\'on de prehaces en $\Omega$. Para ello necesitamos un morfismo que env\'ia \'arboles a su conjunto de representaciones planares, de esta manera tenemos el prehaz $P$ en $\Omega$. Entonces, tenemos que $\Omega\backslash P = \Omega_p$ y en consecuencia, existe una proyecci\'on de $\Omega_p$ a $\Omega$.
De esta manera podemos ver que las categor\'ias dendroidales extienden la categor\'ia simplicial mediante el encaje de $\Delta$ en $\Omega_p$ viendo los objetos de $\Delta$ como \'arbolse lineales y la anterior proyecci\'on. 
Finalmente, un conjunto dendroidal $X$ es una conjunto de d\'endrices con forma $T$ para todo \'arbol $T$ en $\Omega$ junto las funciones asociadas a los morfismos de $\Omega$.

El producto tensorial de Boardman--Vogt introduce un producto distinto al producto cartesiano dentro de las op\'eradas coloreadas y gracias al nervio dendroidal podemos enviar un \'arbol como op\'erada coloreada a su representable en los conjuntos dendroidales.
De esta manera, si definimos dos conjuntos dendroidales como los colimites de representables, podemos encontrar el producto tensorial de dichos conjuntos dendroidales encontrando el producto tensorial de Boardman--Vogt de dos op\'eradas coloreadas y luego aplicando el nervio dendroidal.

Encontrar el producto tensorial no es tarea f\'acil, as\'i que necesitamos introducir la noci\'on del conjunto de shuffles, donde un shuffle es simplemente un \'arbol como op\'erada coloreada generado por dos iniciales, manteniendo la configuraci\'on de cada uno de ellos a nivel de v\'ertices. Es decir, la c\'opia de un v\'ertice blanco de la primera \'operada en un shuffle tendr\'a las mismas entradas que ten\'ia en su op\'erada original pero con la diferencia que estas entradas estar\'an conectadas al output de las c\'opias de un v\'ertice negro de la segunda op\'erada.
Luego, podemos generar todo el conjunto de shuffles distintos siguiendo la norma de intercambio. Una vez que encontramos el conjuntos de shuffles de dos \'arboles cualesquiera podemos definir el producto tensorial de sus representables en conjuntos dendroidales como la uni\'on de los representables de todos los shuffles encontrados.

Tenemos una funci\'on recursiva que nos calcula cuantos shuffles hay entre dos \'arboles sin tener que generarlos. Nos preguntamos si es posible encontrar una f\'ormula general de dicha funci\'on recursiva. Si trabajamos con los \'arboles lineales tenemos un caso especial donde s\'i existe una f\'ormula cerrada, resulta que el problema se reduce a una relaci\'on inductiva muy conocida, el coficiente binomial. En general, es un problema abierto.

Finalmente, generar todos los shuffles es una tarea relativamente f\'acil pero costosa de tiempo ya que el n\'umero de shuffles aumenta mucho seg\'un el tama\~no de los \'arboles. Por eso hemos desarrollado un paquete en Python para que genere el conjunto de shuffles entre dos \'arboles cualesquiera.
Para realizar el paquete hemos abstraido las nociones de op\'eradas coloreadas, de \'arboles como op\'eradas coloreadas y de los shuffles de \'arboles.
Gracias al paquete hemos podido ilustrar todas las figuras que aparecen en el trabajo que representan los \'arboles y los ejemplos de los conjuntos de shuffles.


% Idea del trabajo: el producto tensorial de conjuntos dendroidales

% Proceso y nociones: 
% 1. Conjuntos Dendroidales
% 1.1. formalismo de \'arboles
% 1.2. Cateogria $\Omega_p$
% 1.3. Categoria $\Omega$
% Para poder hilar la primera parte del trabajo 
% 2. Conjuntos simpliciales
% 3. Prehaz
% 4. conjunto dendroidal deficion categorica
% 4.1 nervio dendroidal
% 5. producto tensorial
% 5.1. Boardman-Vogt
% 5.2. producto tensorial en conjuntos dendroidales
% 6. shuffle de arboles
% 6.1. conjunto de shuffles
% 6.2. cardinal
% 6.3. estructura de orden parcial
% 6.4. equivalencia de union de shuffles

% Resultados pr\'acticos: shuffle de arboles en Python

% Presentación
% Introduccion
% 1. Nociones previas
% 2. Árboles como operadas
% 3. Conjuntos dendroidales
% 4. Producto tensorial
% 5. Conjunto de shuffles
% 6. Conclusión 


% ------------------------------------------

% --- Acaban las secciones -----------------

% --- Empieza la bibliografía ---
\newpage
\begin{thebibliography}{4}
    \bibitem{BCT} Tom Leinster: \textit{Basic Category Theory}, Cambridge Studies in Advanced Mathematics, Vol. 143, Cambridge University Press, 2014. 
    \bibitem{SS} Greg Friedman: \textit{An elementary illustrated introduction to simplicial sets}, Rocky Mountain J. Math. 42 (2012), no. 2, 353-423, \href{https://arxiv.org/abs/0809.4221}{arXiv:0809.4221 [math.AT]}.
    \bibitem{DS} Ieke Moerdijk and Bertrand To\"en: \textit{Simplicial Methods for Operads and Algebraic Geometry}, Springer Basel AG 2010.
    \bibitem{SoT} Eric Hoffbeck and Ieke Moerdijk: \textit{Shuffles of trees}, \href{https://arxiv.org/abs/1705.03638}{arXiv:1705.03638 [math.CO]}.
    \bibitem{CWM} Saunders Mac Lane: \textit{Categories for the Working Mathematician}, Springer Science+Business Media New York 1978.
    \bibitem{BCEC} G. M. Kelly: \textit{Basic Concepts of Enriched Category Theory}, London Math. Soc. Lecture Notes, vol. 64, Cambridge University Press, Cambridge, 1982.
\end{thebibliography}
% --- Empieza la bibliografía ---

\end{document}