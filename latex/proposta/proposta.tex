% --- Empiezan las configuraciones ---------
\documentclass[11pt,a4paper,openright,oneside]{article}
\usepackage{amsfonts, amsmath, amssymb,latexsym,amsthm, mathrsfs, enumerate}
\usepackage[all]{xy}
\SelectTips{cm}{} %to change the tips and tail of the arrows 
\usepackage[spanish]{babel}
\usepackage{epsfig}

\parskip=5pt
\parindent=15pt
\usepackage[margin=1.2in]{geometry}
\usepackage{graphicx}
\usepackage{listings}
\usepackage[latin1]{inputenc}

\setcounter{page}{0}


\numberwithin{equation}{section}
\newtheorem{teo}{Teorema}[section]
\newtheorem*{teo*}{Teorema}
\newtheorem{prop}[teo]{Proposici\'on}
\newtheorem{corol}[teo]{Corolario}
\newtheorem{lema}[teo]{Lema}
\newtheorem{defi}[teo]{Definici\'on}
\newtheorem{nota}[teo]{Notaci\'on}

\theoremstyle{definition}
\newtheorem{prob}[teo]{Problema}
\newtheorem*{sol}{Soluci\'ona}
\newtheorem{ex}[teo]{Ejemplo}
\newtheorem{exs}[teo]{Ejemplos}
\newtheorem{obs}[teo]{Observaci\'on}
\newtheorem{obss}[teo]{Observaciones}

\def\qed{\hfill $\square$}

\renewcommand{\refname}{Bibliografia}
\usepackage{fancyhdr}

\lhead{}
\lfoot{}
\rhead{}
\cfoot{}
\rfoot{\thepage}
\begin{document}
\bibstyle{plain}
\thispagestyle{empty}
% --- Acaban las configuraciones -----------

% ------------------------------------------
% ------------------------------------------
% ------------------------------------------

% --- Empieza la portada -------------------
\begin{titlepage}
    \begin{center}
        \begin{figure}[htb]
            \begin{center}
                % \includegraphics[width=6cm]{matematiquesinformatica-pos-rgb.png}
            \end{center}
        \end{figure}
        \vspace*{1cm}
        \textbf{\LARGE GRAU DE MATEM\`{A}TIQUES } \\
        \vspace*{.5cm}
        \textbf{\LARGE Treball final de grau} \\
        \vspace*{1.5cm}
        \rule{16cm}{0.1mm}\\
        \begin{Huge}
            \textbf{Aspectos combinatorios del producto tensorial de conjuntos dendroidales} \\
        \end{Huge}
        \rule{16cm}{0.1mm}\\
        \vspace{1cm}
        \begin{flushright}
            \textbf{\LARGE Autor: Roger Brasc\'o Garc\'es}
            \vspace*{2cm}
            \renewcommand{\arraystretch}{1.5}
            \begin{tabular}{ll}
                \textbf{\Large Director:}    & \textbf{\Large Dr. Javier J. Guti\'errez }  \\
                \textbf{\Large Realitzat a:} & \textbf{\Large  Departament de Topolog\'ia} \\
                \\
                \textbf{\Large Barcelona,}   & \textbf{\Large 23 de enero de 2022 }
            \end{tabular}
        \end{flushright}
    \end{center}
\end{titlepage}
\newpage
% --- Acaba la portada ---------------------

% ------------------------------------------
% ------------------------------------------
% ------------------------------------------

% --- Empieza el encabezado ----------------
\pagenumbering{roman}

% --- Empieza resumen ----------------------
\section*{Resumen}
 {\let\thefootnote\relax\footnote{2010 Mathematics Subject Classification. 11G05, 11G10, 14G10}}
\newpage
% --- Acaba resumen ------------------------

% ------------------------------------------

% --- Empieza agradecimientos --------------
\section*{Agradecimientos}
\newpage
% --- Acaba agradecimientos ---------------

% ------------------------------------------

% --- Empieza indice ----------------------
\tableofcontents
\newpage
% --- Acaba indice -------------------------

% --- Acaba el encabezado ------------------

% ------------------------------------------
% ------------------------------------------
% ------------------------------------------

% --- Empiezan las secciones ---------------
\pagenumbering{arabic}
\setcounter{page}{1}

% --- Empieza sección 1 --------------------
\section{Nociones previas}
\subsection{Categor\'ias}
% Bibliografía Adámek, Jiří; Herrlich, Horst; Strecker, George E. (1990), Abstract and Concrete Categories (PDF), Wiley, ISBN 0-471-60922-6 (now free on-line edition, GNU FDL). http://katmat.math.uni-bremen.de/acc/acc.pdf
\begin{defi}
    Categor\'ia
\end{defi}
Una categor\'ia is una cuadr\'upula $\mathbf{A} = (\mathcal{O}, \text{hom}, \mathit{id}, \circ)$ que consiste en:
\begin{enumerate}[(1)]
    \item Una clase $\mathcal{O}$ que sus elementos ser\'an llamados $\mathbf{A}$\textbf{-objetos}. Usaremos la notaci\'on $\mathit{Ob}(\mathbf{A})$ oara simplificar.
    \item Para cada pareja de objetos $(A, B)$ de $\mathbf{A}$, tenemos un conjunto de $\text{hom}(A,B)$, cuyos elementos ser\'an llamados $\mathbf{A}$\textbf{-morfismos} de $A$ a $B$; \'es decir, los morfismos $A \overset{f}{\longrightarrow} B$ para todo $f \in \text{hom}(A,B)$.
    \item Para cada objeto $A$ de $\mathbf{A}$ definimos el morfismo $A \overset{\mathit{id}_{A}}{\longrightarrow} A$ como la identidad $A$.
    \item Sean $A \overset{f}{\longrightarrow} B$ y $B \overset{g}{\longrightarrow} C$ dos morifismos de $\mathbf{A}$, definimos la composici\'on $\circ$ como:
          $$
              \xymatrix{
                  A \ar[rd]_{g\circ f} \ar[r]^f
                  &B \ar[d]^g\\
                  &C
              }
          $$
          Composici\'on que cumple con las siguientes condiciones:
          \begin{enumerate}[(a)]
              \item Es asociativa: sean $A \overset{f}{\longrightarrow} B$, $B \overset{g}{\longrightarrow} C$ y $C \overset{h}{\longrightarrow} D$ morifismos de $\mathbf{A}$, entonces se cumple $h\circ (g\circ f) = (h\circ g)\circ f$.
              \item Respecta la identidad: para todo morfismo $A \overset{f}{\longrightarrow} B$ de $\mathbf{A}$, se cumple $\mathit{id}_B\circ f = f$ y $f\circ \mathit{id}_A = f$.
          \end{enumerate}
\end{enumerate}
\begin{ex}
    Categor\'ia $\mathbf{Set}$ cuyos objetos son todos los conjuntos y los morfismos son las funciones totales.
\end{ex}

\begin{defi}
    Categor\'ia opuesta
\end{defi}
Para toda categor\'ia $\mathbf{A} = (\mathcal{O}, \text{hom}_{\mathbf{A}}, \mathit{id}, \circ)$ definimos la categor\'ia opuesta como $\mathbf{A}^{\text{op}} = (\mathcal{O}, \text{hom}_{\mathbf{A}^{\text{op}}}, \mathit{id}, \circ^{\text{op}})$,
donde $\text{hom}_{\mathbf{A}^{\text{op}}}(A,B) = \text{hom}_{\mathbf{A}}(B,A)$ y $f\circ^{\text{op}}g = g\circ f$. Podemos observar que $\mathbf{A}$ y $\mathbf{A}^{\text{op}}$ tienen los mismos objetos y los mismos morfismos pero cambiados de direcci\'on.

\subsubsection{Functores}
\begin{defi}
    Functor
\end{defi}
Sean $\mathbf{A}$ y $\mathbf{B}$ dos categor\'ias, definimos un functor $F$ de $\mathbf{A}$ a $\mathbf{B}$ como una funci\'on que asigna cada objeto $A \in\mathit{Ob}(\mathbf{A})$ un objeto $F(A) \in\mathit{Ob}(\mathbf{B})$,
y para cada morfismo de $\mathbf{A}$ $A \overset{f}{\longrightarrow} A'$ un morfismo de $\mathbf{B}$ $F(A) \overset{F(f)}{\longrightarrow} F(A')$.
\begin{align*}
    F: \mathbf{A} & \longrightarrow \mathbf{B} \\
    A & \longmapsto\!
    \begin{aligned}[t]
        F(A)
    \end{aligned} \\
    f & \longmapsto\!
    \begin{aligned}[t]
    F(f)
    \end{aligned} 
\end{align*}
De manera que:
\begin{enumerate}[(1)]
    \item $F$ conserva la composici\'on: $F(f\circ g) = F(f)\circ F(g)$, siempre y cuando $f\circ g$ est\'e bien definido.
    \item $F$ conserva los morfismos identidad: $F(\mathit{id}_A)=\mathit{id}_{F(A)}$, para cada $A \in\mathit{Ob}(\mathbf{A})$.
\end{enumerate}

\subsection{Operadas}
\subsubsection{Operadas coloradas}
\newpage
% --- Acaba sección 1 ----------------------

% ------------------------------------------

% --- Empieza sección 2 --------------------
\section{Conjuntos Simpliciales}
\subsection{Complejos simpliciales}
\subsubsection{Morfismos simpliciales}
\subsection{Conjuntos Delta}
\subsubsection{Morfismos Delta}
\subsection{Conjunto simplicial}
\newpage
% --- Acaba sección 2 ----------------------

% ------------------------------------------

% --- Empieza sección 3 --------------------
\section{Conjuntos Dendroidales}
\subsection{\'Arbol como operadas}
\subsubsection{Caras}
\subsubsection{Funciones degenerativas}
\subsubsection{Identidades de morfismos}
\subsubsection{\'Arboles no planares}
\subsection{Conjunto Dendroidal}
\subsection{Producto tensorial de conjuntos dendroidales}
\subsubsection{Producto tensorial Boardman Vogt}
\subsubsection{Producto Producto tensorial de conjuntos dendroidales}
\newpage
% --- Acaba sección 3 ----------------------

% ------------------------------------------

% --- Empieza sección 4 --------------------
\section{Injertos de \'arboles}
\subsection{Producto tensorial de \'arboles lineales}
\subsection{Producto tensorial de \'arboles}
\subsubsection{Injertos de \'arboles resultantes}
\subsection{C\'alculo de \'arboles resultantes}
\subsubsection{Conjunto de \'arboles resultantes}
\subsubsection{Generarador de \'arboles en Python}
\newpage
% --- Acaba sección 4 ----------------------

% ------------------------------------------

% --- Empieza sección 5 --------------------
\section{Conclusiones}
\newpage
% --- Acaba sección 5 ----------------------

% --- Acaban las secciones -----------------

% --- Empieza la bibliografía ---
\begin{thebibliography}{25}
    \bibitem{pari} Batut, C.; Belabas, K.; Bernardi, D.; Cohen, H.; Olivier, M.: User's guide to \textit{PARI-GP},  \newline \texttt{pari.math.u-bordeaux.fr/pub/pari/manuals/2.3.3/users.pdf}, 2000.
    \bibitem{cw} Chen, J. R.; Wang, T. Z.: On the Goldbach problem, \textit{Acta Math. Sinica}, 32(5):702-718, 1989.
    \bibitem{desh} Deshouillers, J. M.: Sur la constante de $\check{\text{S}}\text{nirel}^{\prime} \text{man}$, \textit{S\'eminaire Delange-Pisot-Poitou, 17e ann\'ee: (1975/76), Th\'eorie des nombres: Fac. 2, Exp. No.} G16, p\'ag. 6, Secr\'etariat Math., Paris, 1977.
    \bibitem{derz} Deshouillers, J. M.; Effinger, G.; te Riele, H.; Zinoviev, D.: A complete Vinogradov 3-primes theorem under the Riemann hypothesis, \textit{Electron. Res. Announc. Amer. Math. Soc.}, 3:99-104, 1997.
    \bibitem{dick} Dickson, L. E.: \textit{History of the theory of numbers. Vol. I: Divisibility and primality}, Chelsea Publishing Co., New York, 1966.
    \bibitem{hl} Hardy, G. H.; Littlewood, J. E.: Some problems of \textquoteleft Partitio numerorum\textquoteright; III: On the expression of a number as a sum of primes, \textit{Acta Math.}, 44(1):1-70, 1923.
    \bibitem{hara} Hardy, G. H.; Ramanujan, S.: Asymptotic formulae in combinatory analysis, \textit{Proc. Lond. Math. Soc.}, 17:75-115, 1918.
    \bibitem{haw} Hardy, G. H.; Wright, E. M.: \textit{An introduction to the theory of numbers}, 5a edici\'on, Oxford University Press, 1979.
    \bibitem{minarc} Helfgott, H. A.: Minor arcs for Goldbach's problem, \newline \texttt{arXiv:1205.5252v4 [math.NT]}, diciembre de 2013.
    \bibitem{majarc} Helfgott, H. A.: Major arcs for Goldbach's problem, \newline \texttt{arXiv:1305.2897v4 [math.NT]}, abril de 2014.
    \bibitem{istrue} Helfgott, H. A.: The ternary Goldbach conjecture is true, \newline \texttt{arXiv:1312.7748v2 [math.NT]}, enero de 2014.
    \bibitem{HP} Helfgott, H. A.; Platt, D.: Numerical verification of the ternary Goldbach conjecture up to $8.875 \cdot 10^{30}$, \texttt{arXiv:1305.3062v2 [math.NT]}, abril de 2014.
    \bibitem{KPS} Klimov, N. I.; $\text{Pil}^{\prime} \text{tja}\breve{\imath}$, G. Z.; $\check{\text{S}}\text{eptickaja}$, T. A.: An estimate of the absolute constant in the Goldbach-$\check{\text{S}}\text{nirel}^{\prime} \text{man}$ problem, \textit{Studies in number theory, No. 4}, p\'ags. 35-51, Izdat. Saratov. Univ., Saratov, 1972.
    \bibitem{lw} Liu, M. C.; Wang, T.: On the Vinogradov bound in the three primes Goldbach conjecture, \textit{Acta Arith.}, 105(2):133-175, 2002.
    \bibitem{OSHP} Oliveira e Silva, T.; Herzog, S.; Pardi, S.: Empirical verification of the even Goldbach conjecture and computation of prime gaps up to $4\cdot10^{18}$, \textit{Math. Comp.}, 83:2033-2060, 2014.
    \bibitem{ram} Ramar\'e, O.: On $\check{\text{S}}\text{nirel}^{\prime} \text{man's}$ constant, \textit{Ann. Scuola Norm. Sup. Pisa Cl. Sci.}, 22(4):645-706, 1995.
    \bibitem{riva} Riesel, H.; Vaughan, R. C.: On sums of primes, \textit{Ark. Mat.}, 21(1):46-74, 1983.
    \bibitem{RS} Rosser, J. B.; Schoenfeld, L.: Approximate formulas for some functions of prime numbers, \textit{Illinois J. Math.}, 6:64-94, 1962.
    \bibitem{sch} Schnirelmann, L.: \"Uber additive Eigenschaften von Zahlen, \textit{Math. Ann.}, 107(1):649-690, 1933.
    \bibitem{tao} Tao, T.: Every odd number greater than $1$ is the sum of at most five primes, \textit{Math. Comp.}, 83:997-1038, 2014.
    \bibitem{ari} Travesa, A.: \textit{Aritm\`etica}, Co{\l}ecci\'o UB, No. 25, Barcelona, 1998.
    \bibitem{vau} Vaughan, R. C.: On the estimation of Schnirelman's constant, \textit{J. Reine Angew. Math.}, 290:93-108, 1977.
    \bibitem{vgn} Vaughan, R. C.: \textit{The Hardy-Littlewood method}, Cambridge Tracts in Mathematics, No. 125, 2a edici\'on, Cambridge University Press, 1997.
    \bibitem{vino} Vinogradov, I. M.: Sur le th\'eor\`eme de Waring, \textit{C. R. Acad. Sci. URSS}, 393-400, 1928.
    \bibitem{vin} Vinogradov, I. M.: Representation of an odd number as a sum of three primes, \textit{Dokl. Akad. Nauk. SSSR}, 15:291-294, 1937.
\end{thebibliography}
% --- Empieza la bibliografía ---

\end{document}

