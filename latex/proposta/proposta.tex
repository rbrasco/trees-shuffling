% --- Empiezan las configuraciones ---------
\documentclass[11pt,a4paper,openright,oneside]{article}
%\usepackage{amsfonts, amsmath, amssymb,latexsym,amsthm, mathrsfs, enumerate, tikz-cd}
\usepackage{amsfonts, amsmath, amssymb,latexsym,amsthm, mathrsfs, enumerate}

\usepackage[all]{xy}

\SelectTips{cm}{} %to change the tips and tail of the arrows 
\usepackage[spanish]{babel}
\usepackage{epsfig}

\parskip=5pt
\parindent=15pt
\usepackage[margin=1.2in]{geometry}
\usepackage{graphicx}
\usepackage{listings}
\usepackage[latin1]{inputenc}
\usepackage{fancyhdr}

\setcounter{page}{0}


\numberwithin{equation}{section}
\newtheorem{teo}{Teorema}[section]
\newtheorem*{teo*}{Teorema}
\newtheorem{prop}[teo]{Proposici\'on}
\newtheorem{corol}[teo]{Corolario}
\newtheorem{lema}[teo]{Lema}
\newtheorem{nota}[teo]{Notaci\'on}

\theoremstyle{definition}
\newtheorem{defi}[teo]{Definici\'on}
\newtheorem{prob}[teo]{Problema}
\newtheorem*{sol}{Soluci\'on}
\newtheorem{ex}[teo]{Ejemplo}
\newtheorem{exs}[teo]{Ejemplos}
\newtheorem{obs}[teo]{Observaci\'on}
\newtheorem{obss}[teo]{Observaciones}

\def\qed{\hfill $\square$}

\renewcommand{\refname}{Bibliografia}


% Definiciones de funciones matemáticas

\newcommand{\Set}{\mathop{\rm Set}}
\newcommand{\Grp}{\mathop{\rm Grp}}
\newcommand{\Top}{\mathop{\rm Top}}
\newcommand{\Oper}{\mathop{\rm Oper}}

\lhead{}
\lfoot{}
\rhead{}
\cfoot{}
\rfoot{\thepage}
\begin{document}
\bibstyle{plain}
\thispagestyle{empty}
% --- Acaban las configuraciones -----------

% ------------------------------------------
% ------------------------------------------
% ------------------------------------------

% --- Empieza la portada -------------------
\begin{titlepage}
    \begin{center}
        \begin{figure}[htb]
            \begin{center}
                % \includegraphics[width=6cm]{matematiquesinformatica-pos-rgb.png}
            \end{center}
        \end{figure}
        \vspace*{1cm}
        \textbf{\LARGE GRAU DE MATEM\`{A}TIQUES } \\
        \vspace*{.5cm}
        \textbf{\LARGE Treball final de grau} \\
        \vspace*{1.5cm}
        \rule{16cm}{0.1mm}\\
        \begin{Huge}
            \textbf{Aspectos combinatorios del producto tensorial de conjuntos dendroidales} \\
        \end{Huge}
        \rule{16cm}{0.1mm}\\
        \vspace{1cm}
        \begin{flushright}
            \textbf{\LARGE Autor: Roger Brasc\'o Garc\'es}
            \vspace*{2cm}
            \renewcommand{\arraystretch}{1.5}
            \begin{tabular}{ll}
                \textbf{\Large Director:}    & \textbf{\Large Dr. Javier J. Guti\'errez }  \\
                \textbf{\Large Realitzat a:} & \textbf{\Large  Departament de Topolog\'ia} \\
                \\
                \textbf{\Large Barcelona,}   & \textbf{\Large 23 de enero de 2022 }
            \end{tabular}
        \end{flushright}
    \end{center}
\end{titlepage}
\newpage
% --- Acaba la portada ---------------------

% ------------------------------------------
% ------------------------------------------
% ------------------------------------------

% --- Empieza el encabezado ----------------
\pagenumbering{roman}

% --- Empieza resumen ----------------------
\section*{Resumen}
 {\let\thefootnote\relax\footnote{2010 Mathematics Subject Classification. 11G05, 11G10, 14G10}}
\newpage
% --- Acaba resumen ------------------------

% ------------------------------------------

% --- Empieza agradecimientos --------------
\section*{Agradecimientos}
\newpage
% --- Acaba agradecimientos ---------------

% ------------------------------------------

% --- Empieza indice ----------------------
\tableofcontents
\newpage
% --- Acaba indice -------------------------

% --- Acaba el encabezado ------------------

% ------------------------------------------
% ------------------------------------------
% ------------------------------------------

% --- Empiezan las secciones ---------------
\pagenumbering{arabic}
\setcounter{page}{1}

% --- Empieza sección 1 --------------------
\section{Nociones previas}
\subsection{Categor\'ias}
% Bibliografía Adámek, Jiří; Herrlich, Horst; Strecker, George E. (1990), Abstract and Concrete Categories (PDF), Wiley, ISBN 0-471-60922-6 (now free on-line edition, GNU FDL). http://katmat.math.uni-bremen.de/acc/acc.pdf
% borceux handbook of categorical algebra vol I
\begin{defi}
    Una \emph{categor\'{\i}a} $\mathcal{C}$ consiste en:
    \begin{itemize}
        \item Una clase ${\rm Ob}(\mathcal{C})$, cuyos elementos llamaremos \emph{objectos} de la categor\'{\i}a.
        \item Para cada par de objectos $A, B\in{\rm Ob}(\mathcal{C})$ un conjunto $\mathcal{C}(A,B)$ de \emph{morfismos} o \emph{flechas} de $A$ a $B$.
        \item Para cada tres objectos $A, B, C\in{\rm Ob}(\mathcal{C})$ una \emph{funci\'on de composici\'on}
              $$
                  \mathcal{C}(B,C)\times \mathcal{C}(A,B)\stackrel{\circ}{\longrightarrow} \mathcal{C}(A,C)
              $$
              que env\'{\i}a el par $(g,f)$ a $g\circ f$.
        \item Para cada objeto $A$, un elemento ${\rm id}_A\in\mathcal{C}(A,A)$ que llamaremos la \emph{identidad} en $A$.
    \end{itemize}
    Adem\'as, esta estructura cumple los siguientes axiomas:
    \begin{itemize}
        \item \emph{Asociatividad}. La funci\'on de composici\'on es asociativa, esto es, dados $f\in\mathcal{C}(A,B)$, $g\in\mathcal{C}(B,C)$ y $h\in\mathcal{C}(C,D)$, se cumple que $(h\circ g)\circ f=h\circ(g\circ f)$.
        \item \emph{Unidad}. La identidad es un elemento neutro para la composici\'on, es decir, para toda $f\in\mathcal{C}(A,B)$ tenemos que $f\circ {\rm id}_A=f={\rm id}_B\circ f$.
    \end{itemize}
\end{defi}

A menudo, denotaremos un objecto $A$ de $\mathcal{C}$ como $A\in \mathcal{C}$, en vez de $A\in{\rm Ob}(\mathcal{C})$ y un morfismo $f\in\mathcal{C}(A,B)$ como $f\colon A\to B$. Una categor\'{\i}a $\mathcal{C}$ es \emph{peque\~na} si ${\rm Ob}(\mathcal{C})$ es un conjunto.
\begin{ex}
    Los siguientes son algunos ejemplos de categor\'{\i}as.
    \begin{enumerate}
        \item[{\rm (i)}] La categor\'{\i}a $\Set$ cuyos objetos son todos los conjuntos y cuyos morfismos son la aplicaciones entre conjuntos
        \item[{\rm (ii)}] La categor\'{\i}a $\Grp$ cuyos objetos son los grupos y cuyos morfismos son los morfismos de grupo.
        \item[{\rm (iii)}] La categor\'{\i}a $\Top$ cuyos objetos son los espacios topol\'ogicos y cuyos morfismos son las aplicaciones continuas.
    \end{enumerate}
\end{ex}

\begin{defi}
    Dada una categor\'{\i}a $\mathcal{C}$, podemos definir su \emph{categor\'{\i}a opuesta} $\mathcal{C}^{\rm op}$ de la siguiente manera. Los objectos de $\mathcal{C}^{\rm{op}}$ son los mismos que los de $\mathcal{C}$, los morfismos cambian de direcci\'on $\mathcal{C}^{\rm op}(A,B)=\mathcal{C}(B,A)$ y la funci\'on de composici\'on es $f\circ^{\rm op}g=g\circ f$.
\end{defi}

\subsubsection{Funtores}
\begin{defi}
    Sean $\mathcal{C}$ y $\mathcal{D}$ dos categor\'{\i}as. Un \emph{funtor} $F$ de $\mathcal{C}$ en $\mathcal{D}$, que denotaremos por $F\colon\mathcal{C}\to \mathcal{D}$ consiste en:
    \begin{itemize}
        \item Una aplicaci\'on ${\rm Ob}(\mathcal{C})\to {\rm Ob}(\mathcal{D})$. La imagen de un objeto $A$ de $\mathcal{C}$ la denotaremos por $F(A)$
        \item Para cada par de objetos $A,B\in\mathcal{C}$ una aplicaci\'on
              $$
                  \mathcal{C}(A,B)\longrightarrow\mathcal{D}(F(A), F(B)).
              $$
              La imagen de un morfismo $f\colon A\to B$ por esta aplicaci\'on la denotaremos por $F(f)\colon F(A)\to F(B)$.
    \end{itemize}
    Adem\'as, estas aplicaciones son compatibles con la composici\'on y la unidad, esto es, se cumplen los siguientes axiomas:
    \begin{itemize}
        \item Dados $f\in\mathcal{C}(A,B)$ y $g\in\mathcal{C}(B,C)$ se cumple que $F(g\circ f)=F(g)\circ F(f)$.
        \item Para todo objeto $A\in\mathcal{C}$ se cumple que $F({\rm id}_A)={\rm id}_{F(A)}$.
    \end{itemize}
\end{defi}
\begin{obs}
    La noci\'on de funtor que acabamos se llama tambi\'en \emph{funtor covariante} de $\mathcal{C}$ en $\mathcal{D}$. Un funtor de $\mathcal{C}^{\rm op}$ en $\mathcal{D}$ se llama \emph{functor contravariante} de $\mathcal{C}$ en $\mathcal{D}$. Observar que si $F$ es un funtor contravariante de $\mathcal{C}$ en $\mathcal{D}$ y $f\colon A\to B$ es un morfismo en $\mathcal{C}$, entonces $F(f)\colon F(B)\to F(A)$.
\end{obs}
\begin{ex}
    Dado un conjunto $X$ cualquiera, podemos construir el grupo libre en los elementos de este conjunto $F(X)$. Esto define un funtor $F\colon\Set\to\Grp$.
\end{ex}

\begin{defi}
    Sea $F: \mathcal{C}\to \mathcal{D}$ un funtor entre dos categor\'{\i}as $\mathcal{C}$ y $\mathcal{D}$. Dados un par de objetos $A,B\in\mathcal{C}$ consideremos la aplicaci\'on
    $$
        F_{A,B}\colon \mathcal{C}(A,B)\longrightarrow\mathcal{D}(F(A), F(B)).
    $$
    \begin{itemize}
        \item Diremos que $F$ es un funtor \emph{fiel} si para cada par de objetos $A, B\in\mathcal{C}$ la aplicaci\'on $F_{A,B}$ es inyectiva.
        \item Diremos que $F$ es un funtor \emph{pleno} si para cada par de objetos $A, B\in\mathcal{C}$ la aplicaci\'on $F_{A,B}$ es exhaustiva.
        \item Diremos que $F$ es un funtor \emph{plenamente fiel} si para cada par de objetos $A, B\in\mathcal{C}$ la aplicaci\'on $F_{A,B}$ es biyectiva.
    \end{itemize}
\end{defi}
\subsection{Op\'eradas en conjuntos}
% Bibliografía Ieke Moerdijk; Bertrand Toën. (2010), SImplicial Methods for Operads and Algebraic Geometry (PDF), Birkhäuser, ISBN 978-3-0348-0051-8
Para cada $n\ge 0$, denotaremos por $\Sigma_n$ el grupo sim\'etrico de $n$ letras (en el caso $n=0,1$, $\Sigma_n$ ser\'a el grupo trivial).
\begin{defi}
    Una \emph{op\'erada} $P$ consiste en una sucesi\'on de conjuntos $\{P(n)\}_{n\ge 0}$ junto con la siguiente estructura:
    \begin{itemize}
        \item Un elemento \emph{unidad} $1\in P(1)$.
        \item Un \emph{producto composici\'on}
              $$
                  P(n)\times P(k_1) \times\cdots\times P(k_n)\longrightarrow P(k)
              $$
              para cada $n$ y $k_1,\dots,k_n$ tal que $k=\sum_{i=1}^{n}{k_i}$.
        \item Para cada $\sigma\in\Sigma_n$ una \emph{acci\'on por la derecha} $\sigma^*\colon P(n)\to P(n)$.
    \end{itemize}
    Adem\'as el producto composici\'on es asociativo, equivariante y compatible con la unidad.
\end{defi}
%TODO: afirmaciones/axiomas? 

\begin{defi}
    Dadas dos o\'eradas $P$ y $Q$, un morfismo de op\'eradas $f\colon P\to Q$ consiste en aplicaciones $f_n\colon P(n)\to Q(n)$ para cada $n\ge 0$ compatibles con el producto composici\'on, la unidad y la acci\'on del grupo sim\'etrico.
\end{defi}

\subsubsection{Op\'eradas coloreadas}
La noci\'on de op\'erada coloreada generaliza a la vez el concepto de categor{\'i}a y de op\'erada.
\begin{defi}
    Sea $C$ un conjunto, cuyos elementos llameremos colores. Una op\'erada $C$-coloreada $P$ consiste en, para cada $(n+1)$-tupla de colores $(c_1,\ldots,c_n,c)$ con $n\ge 0$, un conjunto $P(c_1,\ldots, c_n;c)$ (que representar\'a el conjunto de operaciones cuyas entradas est\'an coloreadas por los colores $c_1,\ldots, c_n$ y cuya salida esta coloreada por $c$), junto con la siguiente estructura:
    \begin{itemize}
        \item Un elemento \emph{unidad} $1_c\in P(c;c)$ para cada $c\in C$.
        \item Un \emph{producto composici\'on}
              \begin{align*}
                  P(c_1,\dots,c_n;c) & \otimes P(d_{1,1},\dots,d_{1,k_1};c_1) \otimes\dots\otimes P(d_{n,1},\dots,d_{n,k_n};c_n) \\
                                     & \longrightarrow P(d_{1,1},\dots,d_{1,k_1},\dots,d_{n,1},\dots,d_{n,k_n};c)
              \end{align*}
              para cada $(n+1)$-tupla de colores $(c_1,\dots,c_n;c)$ y $n$ tuplas cualesquiera
              $$
                  (d_{1,1},\dots,d_{1,k_1};c_1),\dots,(d_{n,1},\dots,d_{n,k_n};c_n)
              $$
        \item Para cada elemento $\sigma\in\Sigma_n$ una \emph{acci\'on}
              $$
                  \sigma^{*}: P(c_1,\dots,c_n;c) \longrightarrow P(c_{\sigma(1)},\dots,c_{\sigma(n)};c).
              $$
    \end{itemize}
    Adem\'as el producto composici\'on es asociativo, equivariante y compatible con las unidades.
\end{defi}

\begin{defi}
    Sea $P$ una op\'erada $C$-coloreada y $Q$ una op\'erada $D$-coloreada. Un \emph{morfismo de op\'eradas} $f\colon P\to Q$ consiste en una aplicaciones entre los conjuntos de colores $f\colon C\to D$ y aplicaciones
    $$
        f_{c_1,\dots,c_n;c}: P(c_1,\dots,c_n;c) \longrightarrow Q(f(c_1),\dots,f(c_n);c)
    $$
    compatibles con el producto composici\'on, las unidades y la acci\'on del grupo sim\'etrico.
\end{defi}

Denotaremos por $\Oper$ la categor\'ia cuyos objetos son operadas coloreadas y cuyos morfismos son los morfismos de operadas coloreadas.

\begin{ex}
    Si $C=\{*\}$, entonces una op\'erada $C$-coloreada es lo mismo que una op\'erada. Si $P$ es una op\'erad $C$-coloreada tal que solamente tiene operaciones de aridad uno, es decir $P(c_1,\ldots, c_n;c)=\emptyset$ si $n\ne 1$, entonces $P$ es una categor\'{\i}a, cuyo conjunto de objetos es $C$.
\end{ex}
\newpage
% --- Acaba sección 1 ----------------------

% ------------------------------------------

% --- Empieza sección 2 --------------------
\section{Conjuntos Simpliciales}
% Bibliografía Greg Friedman. (2011), An elementary illustrated introduction to simplicial sets (PDF)
\subsection{Complejos simpliciales}
\begin{defi}
    N-simplex
\end{defi}
Un $n\text{-simplex}$ es un politopo de $n\ge 0$ dimensiones formando una envoltura convexa de $n+1$ vertices. Es decir, es un conjunto de puntos afines independientes en un espacio eucl\'ideo de dimensi\'on $n$.

Una cara $m$ de un $n\text{-simplex}$ es una envolutra convexa de $m\le n$ vertices.

\begin{defi}
    Complejo simplicial
\end{defi}
Sea $n\in\mathbb{N}^{*}$, un complejo simplicial $X$ es un conjunto finito de $m\text{-simplex}$ con $m\le n$ que cumplen las condiciones:

\begin{enumerate}[(1)]
    \item Si $m\text{-simplex}\in X \Rightarrow \forall m'\le m\text{, }m'\text{-simplex}\in X$.
    \item Si dos simplices de $X$ se cortan, entonces su intersecci\'on es una cara com\'un.
\end{enumerate}

Sea $X^k$ un complejo simplicial formado por todos los $k\text{-simplex}$ de $X$. Observamos que todo elemento de $X^k$ es un subconjunto de $X^0$ con cardinal $k+1$, donde $X^0=\{v_0,\dots ,v_n\}$.
Generalmente, todo subconjunto de $X^k$ de $j+1$ elementos es un elemento de $X^j$.

Sea $X_k$ un conjunto formado por $k\text{-simplices}$.

\begin{defi}
    N-simplex ordenado
\end{defi}
Un $n\text{-simplex}$ formado por los v\'ertices $v_0,\dots,v_n \in X^0$ es ordenado cuando cuando los v\'ertices estan ordenados, en ese caso nombramos cada v\'ertice por los n\'umeros $0,\dots,n$. Usaremos la notaci\'on $|\Delta^n| = [0,\dots,n]$ para simplificar.


\subsubsection{Morfismos simpliciales}
\begin{defi}
    Morfismo simplicial
\end{defi}
Sea $K$ y $L$ complejos simpliciales. Sea un morfismo simplicial $F: K \longrightarrow L$ que envia los vertices de K a los vertices de L. Es decir, $\forall v \in K^0 \text{, } v \longmapsto F(v) \in L^0$.

\begin{defi}
    Cara
\end{defi}
Para todo $|\Delta^n|$ tenemos $n+1$ caras definidas por los morfismos $\delta_0,\dots,\delta_n$
\begin{align*}
    \delta_j: X_n & \longrightarrow X_{n-1} \\
    [0,\dots,n]   & \longmapsto\!
    \begin{aligned}[t]
        [0,\dots,\hat{j},\dots,n]
    \end{aligned}
\end{align*}
Donde $X_n$ y $X_{n-1}$ son conjuntos de simplices ordenados de $n$ y $n-1$ v\'ertices, respectivamente. Observamos que $\forall i<j$, $\delta_i\delta_j = \delta_{j-1}\delta_{i}$.

\begin{defi}
    Morifismo degenerativo
\end{defi}
Para todo $|\Delta^n|$ tenemos $n+1$ morfismos degenerativos $\sigma_0,\dots,\sigma_n$
\begin{align*}
    \sigma_j: X_n & \longrightarrow X_{n+1} \\
    [0,\dots,n]   & \longmapsto\!
    \begin{aligned}[t]
        [0,\dots,j,j,\dots,n]
    \end{aligned}
\end{align*}
Donde $X_n$ y $X_{n+1}$ son conjuntos de simplices ordenados de $n$ y $n+1$ v\'ertices, respectivamente. Observamos que $\forall i\le j$, $\sigma_i\sigma_j = \sigma_{j+1}\sigma_{i}$.

\subsection{Conjunto Delta}
\begin{defi}
    Conjunto Delta
\end{defi}
Definimos un conjunto Delta como una secuencia de conjuntos $X_0,X_1,\dots$ y para cada $n\ge 0$ las funciones $\delta_i: X_{n+1} \longrightarrow X_n$, $\forall 0\le i \le n+1$, que cumplen $\delta_i\delta_j = \delta_{j-1}\delta_{i}$, $\forall i\le j$.
Formando el siguiente diagrama (Falta por hacer)

$$ %Not working
    \xymatrix{
    X_0  \ar@/{}^{1pc}/[r]  & X_1  \ar@/{}^{1pc}/[r] & X_2  \dots
    }
$$

\subsubsection{Definici\'on categ\'orica del conjunto Delta}
\begin{defi}
    Categor\'ia $\hat{\Delta}$
\end{defi}
Sea la categor\'ia $\hat{\Delta}$ cuyos objetos son los conjuntos estrictamente ordenados finitos $[n] = \{0,\dots,n\}$ y los morfismos son las funciones, que mantienen el orden estrictamente, $f: [m] \longrightarrow [n]$, $m\le n$. Podemos pensar que sea la inclusi\'on de un $m\text{-simplex}$ como cara de un $n\text{-simplex}$.
Para todo $0\le i \le n$ consideramos los morfismos:
\begin{align*}
    d_i: [n]      & \longrightarrow [n+1] \\
    \{0,\dots,n\} & \longmapsto\!
    \begin{aligned}[t]
        \{0,\dots, \hat{i}, \dots,n+1\}
    \end{aligned}
\end{align*}

\begin{defi}
    Categor\'ia $\hat{\Delta}^{op}$
\end{defi}
Sea la categor\'ia $\hat{\Delta}^{op}$, la categor\'ia opuesta de $\hat{\Delta}$, cuyos objetos son los conjuntos estrictamente ordenados finitos $[n] = \{0,\dots,n\}$ y los morfismos son las funciones, que mantienen el orden estrictamente, $f: [n] \longrightarrow [m]$, $m\le n$. Podemos pensar que sea la extracci\'on de la cara $m\text{-simplex}$ de un $n\text{-simplex}$.
Para todo $0\le i \le n$ consideramos los morfismos:
\begin{align*}
    \delta_i: [n] & \longrightarrow [n-1] \\
    \{0,\dots,n\} & \longmapsto\!
    \begin{aligned}[t]
        \{0,\dots, \hat{i}, \dots,n\}
    \end{aligned}
\end{align*}

\begin{defi}
    Conjunto Delta
\end{defi}
Un conjunto Delta es un functor covariante $X: \hat{\Delta}^{op} \longrightarrow \mathbf{Set}$, equivalentemente es un functor contravariante $X: \hat{\Delta} \longrightarrow \mathbf{Set}$.

Faltan observaciones.

\subsection{Conjunto simplicial}
\begin{defi}
    Conjunto simplicial
\end{defi}
Definimos un conjunto simplicial como una secuencia de conjuntos $X_0,X_1,\dots$ y para cada $n\ge 0$ las funciones $\delta_i: X_n \longrightarrow X_{n-1}$ y $\sigma_i: X_n \longrightarrow X_{n+1}$, $\forall 0\le i \le n$, que cumplen:

\begin{enumerate}[(1)]
    \item $\delta_i\delta_j = \delta_{j-1}\delta_{i}$, $i<j$
    \item $\delta_i\sigma_j = \sigma_{j-1}\delta_{i}$, $i<j$
    \item $\delta_j\sigma_j = \delta{j+1}\sigma_{j} = id$
    \item $\delta_i\sigma_j = \sigma_{j}\delta_{i-1}$, $i>j+1$
    \item $\sigma_i\sigma_j = \sigma_{j+1}\sigma_{i}, i\le j$
\end{enumerate}
Formando el siguiente diagrama (Falta por hacer)

$$ %Not working
    \xymatrix{
    X_0  \ar@/{}^{1pc}/[r]  & X_1  \ar@/{}^{1pc}/[r] & X_2  \dots
    }
$$

\subsubsection{Definici\'on categ\'orica del conjunto simplicial}
\begin{defi}
    Categor\'ia $\Delta$
\end{defi}
Sea la categor\'ia $\Delta$ cuyos objetos son los conjuntos ordenados finitos $[n] = \{0,\dots,n\}$ y los morfismos son las funciones, que mantienen solamente el orden, $f: [m] \longrightarrow [n]$.
Para todo $0\le i \le n$ consideramos los morfismos:
\begin{align*}
    d_i: [n]      & \longrightarrow [n+1] \\
    \{0,\dots,n\} & \longmapsto\!
    \begin{aligned}[t]
        \{0,\dots, \hat{i}, \dots,n+1\}
    \end{aligned}
\end{align*}
\begin{align*}
    s_i: [n+1]      & \longrightarrow [n] \\
    \{0,\dots,n+1\} & \longmapsto\!
    \begin{aligned}[t]
        \{0,\dots,i,i, \dots,n\}
    \end{aligned}
\end{align*}

\begin{defi}
    Categor\'ia $\hat{\Delta}^{op}$
\end{defi}
Sea la categor\'ia $\Delta^{op}$, la categor\'ia opuesta de $\Delta$, cuyos objetos son los conjuntos ordenados finitos $[n] = \{0,\dots,n\}$ y los morfismos son las funciones, que mantienen solamente el orden, $f: [m] \longrightarrow [n]$.
Para todo $0\le i \le n$ consideramos los morfismos:
\begin{align*}
    \delta_i: [n] & \longrightarrow [n-1] \\
    \{0,\dots,n\} & \longmapsto\!
    \begin{aligned}[t]
        \{0,\dots, \hat{i}, \dots,n\}
    \end{aligned}
\end{align*}
\begin{align*}
    \sigma_i: [n] & \longrightarrow [n+1] \\
    \{0,\dots,n\} & \longmapsto\!
    \begin{aligned}[t]
        \{0,\dots, i,i, \dots,n\}
    \end{aligned}
\end{align*}

\begin{defi}
    Conjunto simplicial
\end{defi}
Un conjunto simplicial es un functor covariante $X: \Delta^{op} \longrightarrow \mathbf{Set}$, equivalentemente es un functor contravariante $X: \Delta \longrightarrow \mathbf{Set}$.
Usaremos la notaci\'on $\Delta[n]=\Delta(\_,[n])$.
\begin{align*}
    \Delta[n]: \Delta^{op} & \longrightarrow \mathbf{Set} \\
    [m]                    & \longmapsto\!
    \begin{aligned}[t]
        \Delta([m],[n])
    \end{aligned}
\end{align*}
Faltan observaciones.

\subsection{Realizaci\'on geom\'etrica}
\begin{defi}
    Realizaci\'on geom\'etrica
\end{defi}
Sea $X$ un conjunto simplicial. Dotamos cada $X_n$ con la topolog\'ia discreta y sea $|\Delta^n$ el $n\text{-simplex}$ dotado de su topolog\'ia estandard. Definimos la realizaci\'on geom\'etrica como
$$
    |X| = \coprod_{n=0}^{\infty}X_n \times |\Delta^n| / \sim
$$
Donde $\sim$ es la relaci\'on de equivalencia generada por las relaciones:

\begin{enumerate}[(1)]
    \item $(x, d_i(p)) \sim (\delta_i(x), p)$, $x\in X_{n+1}$ y $p\in|\Delta^n|$
    \item $(x, s_i(p)) \sim (\sigma_i(x), p)$, $x\in X_{n-1}$ y $p\in|\Delta^n|$
\end{enumerate}

\begin{ex}
    $\Delta[2] = \Delta(\_, [2])$
\end{ex}
Falta por escribir

% --- Acaba sección 2 ----------------------

% ------------------------------------------

% --- Empieza sección 3 --------------------
\section{Conjuntos Dendroidales}
\subsection{\'Arbol como operadas}
\subsubsection{Caras}
\subsubsection{Funciones degenerativas}
\subsubsection{Identidades de morfismos}
\subsubsection{\'Arboles no planares}
\subsection{Conjunto Dendroidal}
\subsection{Producto tensorial de conjuntos dendroidales}
\subsubsection{Producto tensorial Boardman Vogt}
\subsubsection{Producto Producto tensorial de conjuntos dendroidales}
\newpage
% --- Acaba sección 3 ----------------------

% ------------------------------------------

% --- Empieza sección 4 --------------------
\section{Injertos de \'arboles}
\subsection{Producto tensorial de \'arboles lineales}
\subsection{Producto tensorial de \'arboles}
\subsubsection{Injertos de \'arboles resultantes}
\subsection{C\'alculo de \'arboles resultantes}
\subsubsection{Conjunto de \'arboles resultantes}
\subsubsection{Generarador de \'arboles en Python}
\newpage
% --- Acaba sección 4 ----------------------

% ------------------------------------------

% --- Empieza sección 5 --------------------
\section{Conclusiones}
\newpage
% --- Acaba sección 5 ----------------------

% --- Acaban las secciones -----------------

% --- Empieza la bibliografía ---
\begin{thebibliography}{25}
    \bibitem{pari} Batut, C.; Belabas, K.; Bernardi, D.; Cohen, H.; Olivier, M.: User's guide to \textit{PARI-GP},  \newline \texttt{pari.math.u-bordeaux.fr/pub/pari/manuals/2.3.3/users.pdf}, 2000.
    \bibitem{cw} Chen, J. R.; Wang, T. Z.: On the Goldbach problem, \textit{Acta Math. Sinica}, 32(5):702-718, 1989.
    \bibitem{desh} Deshouillers, J. M.: Sur la constante de $\check{\text{S}}\text{nirel}^{\prime} \text{man}$, \textit{S\'eminaire Delange-Pisot-Poitou, 17e ann\'ee: (1975/76), Th\'eorie des nombres: Fac. 2, Exp. No.} G16, p\'ag. 6, Secr\'etariat Math., Paris, 1977.
    \bibitem{derz} Deshouillers, J. M.; Effinger, G.; te Riele, H.; Zinoviev, D.: A complete Vinogradov 3-primes theorem under the Riemann hypothesis, \textit{Electron. Res. Announc. Amer. Math. Soc.}, 3:99-104, 1997.
    \bibitem{dick} Dickson, L. E.: \textit{History of the theory of numbers. Vol. I: Divisibility and primality}, Chelsea Publishing Co., New York, 1966.
    \bibitem{hl} Hardy, G. H.; Littlewood, J. E.: Some problems of \textquoteleft Partitio numerorum\textquoteright; III: On the expression of a number as a sum of primes, \textit{Acta Math.}, 44(1):1-70, 1923.
    \bibitem{hara} Hardy, G. H.; Ramanujan, S.: Asymptotic formulae in combinatory analysis, \textit{Proc. Lond. Math. Soc.}, 17:75-115, 1918.
    \bibitem{haw} Hardy, G. H.; Wright, E. M.: \textit{An introduction to the theory of numbers}, 5a edici\'on, Oxford University Press, 1979.
    \bibitem{minarc} Helfgott, H. A.: Minor arcs for Goldbach's problem, \newline \texttt{arXiv:1205.5252v4 [math.NT]}, diciembre de 2013.
    \bibitem{majarc} Helfgott, H. A.: Major arcs for Goldbach's problem, \newline \texttt{arXiv:1305.2897v4 [math.NT]}, abril de 2014.
    \bibitem{istrue} Helfgott, H. A.: The ternary Goldbach conjecture is true, \newline \texttt{arXiv:1312.7748v2 [math.NT]}, enero de 2014.
    \bibitem{HP} Helfgott, H. A.; Platt, D.: Numerical verification of the ternary Goldbach conjecture up to $8.875 \cdot 10^{30}$, \texttt{arXiv:1305.3062v2 [math.NT]}, abril de 2014.
    \bibitem{KPS} Klimov, N. I.; $\text{Pil}^{\prime} \text{tja}\breve{\imath}$, G. Z.; $\check{\text{S}}\text{eptickaja}$, T. A.: An estimate of the absolute constant in the Goldbach-$\check{\text{S}}\text{nirel}^{\prime} \text{man}$ problem, \textit{Studies in number theory, No. 4}, p\'ags. 35-51, Izdat. Saratov. Univ., Saratov, 1972.
    \bibitem{lw} Liu, M. C.; Wang, T.: On the Vinogradov bound in the three primes Goldbach conjecture, \textit{Acta Arith.}, 105(2):133-175, 2002.
    \bibitem{OSHP} Oliveira e Silva, T.; Herzog, S.; Pardi, S.: Empirical verification of the even Goldbach conjecture and computation of prime gaps up to $4\cdot10^{18}$, \textit{Math. Comp.}, 83:2033-2060, 2014.
    \bibitem{ram} Ramar\'e, O.: On $\check{\text{S}}\text{nirel}^{\prime} \text{man's}$ constant, \textit{Ann. Scuola Norm. Sup. Pisa Cl. Sci.}, 22(4):645-706, 1995.
    \bibitem{riva} Riesel, H.; Vaughan, R. C.: On sums of primes, \textit{Ark. Mat.}, 21(1):46-74, 1983.
    \bibitem{RS} Rosser, J. B.; Schoenfeld, L.: Approximate formulas for some functions of prime numbers, \textit{Illinois J. Math.}, 6:64-94, 1962.
    \bibitem{sch} Schnirelmann, L.: \"Uber additive Eigenschaften von Zahlen, \textit{Math. Ann.}, 107(1):649-690, 1933.
    \bibitem{tao} Tao, T.: Every odd number greater than $1$ is the sum of at most five primes, \textit{Math. Comp.}, 83:997-1038, 2014.
    \bibitem{ari} Travesa, A.: \textit{Aritm\`etica}, Co{\l}ecci\'o UB, No. 25, Barcelona, 1998.
    \bibitem{vau} Vaughan, R. C.: On the estimation of Schnirelman's constant, \textit{J. Reine Angew. Math.}, 290:93-108, 1977.
    \bibitem{vgn} Vaughan, R. C.: \textit{The Hardy-Littlewood method}, Cambridge Tracts in Mathematics, No. 125, 2a edici\'on, Cambridge University Press, 1997.
    \bibitem{vino} Vinogradov, I. M.: Sur le th\'eor\`eme de Waring, \textit{C. R. Acad. Sci. URSS}, 393-400, 1928.
    \bibitem{vin} Vinogradov, I. M.: Representation of an odd number as a sum of three primes, \textit{Dokl. Akad. Nauk. SSSR}, 15:291-294, 1937.
\end{thebibliography}
% --- Empieza la bibliografía ---

\end{document}

