% --- Empiezan las configuraciones ---------
\documentclass[11pt,a4paper,openright,oneside]{article}
%\usepackage{amsfonts, amsmath, amssymb,latexsym,amsthm, mathrsfs, enumerate, tikz-cd}
\usepackage{amsfonts, amsmath, amssymb,latexsym,amsthm, mathrsfs, enumerate, tikz-cd}

\usepackage[all]{xy}

\SelectTips{cm}{} %to change the tips and tail of the arrows 
\usepackage[spanish]{babel}
\usepackage{epsfig}

\parskip=5pt
\parindent=15pt
\usepackage[margin=1.2in]{geometry}
\usepackage{graphicx}
\usepackage{listings}
\usepackage[latin1]{inputenc}
\usepackage{fancyhdr}

\setcounter{page}{0}


\numberwithin{equation}{section}
\newtheorem{teo}{Teorema}[section]
\newtheorem*{teo*}{Teorema}
\newtheorem{prop}[teo]{Proposici\'on}
\newtheorem{corol}[teo]{Corolario}
\newtheorem{lema}[teo]{Lema}
\newtheorem{nota}[teo]{Notaci\'on}

\theoremstyle{definition}
\newtheorem{defi}[teo]{Definici\'on}
\newtheorem{prob}[teo]{Problema}
\newtheorem*{sol}{Soluci\'on}
\newtheorem{ex}[teo]{Ejemplo}
\newtheorem{exs}[teo]{Ejemplos}
\newtheorem{obs}[teo]{Observaci\'on}
\newtheorem{obss}[teo]{Observaciones}

\def\qed{\hfill $\square$}

\renewcommand{\refname}{Bibliografia}


% Definiciones de funciones matemáticas

\newcommand{\Set}{\mathop{\rm Set}}
\newcommand{\Grp}{\mathop{\rm Grp}}
\newcommand{\Top}{\mathop{\rm Top}}
\newcommand{\Oper}{\mathop{\rm Oper}}

\lhead{}
\lfoot{}
\rhead{}
\cfoot{}
\rfoot{\thepage}
\begin{document}
\bibstyle{plain}
\thispagestyle{empty}
% --- Acaban las configuraciones -----------

% ------------------------------------------
% ------------------------------------------
% ------------------------------------------

% --- Empieza la portada -------------------
\begin{titlepage}
    \begin{center}
        \begin{figure}[htb]
            \begin{center}
                % \includegraphics[width=6cm]{matematiquesinformatica-pos-rgb.png}
            \end{center}
        \end{figure}
        \vspace*{1cm}
        \textbf{\LARGE GRADO EN MATEM\'{A}TICAS } \\
        \vspace*{.5cm}
        \textbf{\LARGE Trabajo final de grado} \\
        \vspace*{1.5cm}
        \rule{16cm}{0.1mm}\\
        \begin{Huge}
            \textbf{Aspectos combinatorios del producto tensorial de conjuntos dendroidales} \\
        \end{Huge}
        \rule{16cm}{0.1mm}\\
        \vspace{1cm}
        \begin{flushright}
            \textbf{\LARGE Autor: Roger Brasc\'o Garc\'es}
            \vspace*{2cm}
            \renewcommand{\arraystretch}{1.5}
            \begin{tabular}{ll}
                \textbf{\Large Director:}     & \textbf{\Large Dr. Javier J. Guti\'errez }     \\
                \textbf{\Large Realizado en:} & \textbf{\Large  Departamento de Matem\'aticas} \\
                                              & \textbf{\Large e Inform\'atica}                \\
                \textbf{\Large Barcelona,}    & \textbf{\Large 23 de enero de 2022 }
            \end{tabular}
        \end{flushright}
    \end{center}
\end{titlepage}
\newpage
% --- Acaba la portada ---------------------

% ------------------------------------------
% ------------------------------------------
% ------------------------------------------

% --- Empieza el encabezado ----------------
\pagenumbering{roman}

% --- Empieza resumen ----------------------
\section*{Resumen}
 {\let\thefootnote\relax\footnote{2010 Mathematics Subject Classification. 11G05, 11G10, 14G10}}
\newpage
% --- Acaba resumen ------------------------

% ------------------------------------------

% --- Empieza agradecimientos --------------
\section*{Agradecimientos}
\newpage
% --- Acaba agradecimientos ---------------

% ------------------------------------------

% --- Empieza indice ----------------------
\tableofcontents
\newpage
% --- Acaba indice -------------------------

% --- Acaba el encabezado ------------------

% ------------------------------------------
% ------------------------------------------
% ------------------------------------------

% --- Empiezan las secciones ---------------
\pagenumbering{arabic}
\setcounter{page}{1}

% --- Empieza sección 1 --------------------
\section{Nociones previas}
\subsection{Categor\'ias}
% Bibliografía Adámek, Jiří; Herrlich, Horst; Strecker, George E. (1990), Abstract and Concrete Categories (PDF), Wiley, ISBN 0-471-60922-6 (now free on-line edition, GNU FDL). http://katmat.math.uni-bremen.de/acc/acc.pdf
% borceux handbook of categorical algebra vol I
\begin{defi}
    Una \emph{categor\'{\i}a} $\mathcal{C}$ consiste en:
    \begin{itemize}
        \item Una clase ${\rm Ob}(\mathcal{C})$, cuyos elementos llamaremos \emph{objetos} de la categor\'{\i}a.
        \item Para cada par de objectos $A, B\in{\rm Ob}(\mathcal{C})$ un conjunto $\mathcal{C}(A,B)$ de \emph{morfismos} o \emph{flechas} de $A$ a $B$.
        \item Para cada tres objectos $A, B, C\in{\rm Ob}(\mathcal{C})$ una \emph{funci\'on de composici\'on}
              $$
                  \mathcal{C}(B,C)\times \mathcal{C}(A,B)\stackrel{\circ}{\longrightarrow} \mathcal{C}(A,C)
              $$
              que env\'{\i}a el par $(g,f)$ a $g\circ f$.
        \item Para cada objeto $A$, un elemento ${\rm id}_A\in\mathcal{C}(A,A)$ que llamaremos la \emph{identidad} en $A$.
    \end{itemize}
    Adem\'as, esta estructura cumple los siguientes axiomas:
    \begin{itemize}
        \item \emph{Asociatividad}. La funci\'on de composici\'on es asociativa, esto es, dados $f\in\mathcal{C}(A,B)$, $g\in\mathcal{C}(B,C)$ y $h\in\mathcal{C}(C,D)$, se cumple que $(h\circ g)\circ f=h\circ(g\circ f)$.
        \item \emph{Unidad}. La identidad es un elemento neutro para la composici\'on, es decir, para toda $f\in\mathcal{C}(A,B)$ tenemos que $f\circ {\rm id}_A=f={\rm id}_B\circ f$.
    \end{itemize}
\end{defi}

A menudo, denotaremos un objecto $A$ de $\mathcal{C}$ como $A\in \mathcal{C}$, en vez de $A\in{\rm Ob}(\mathcal{C})$ y un morfismo $f\in\mathcal{C}(A,B)$ como $f\colon A\to B$. Una categor\'{\i}a $\mathcal{C}$ es \emph{peque\~na} si ${\rm Ob}(\mathcal{C})$ es un conjunto.
\begin{ex}
    Los siguientes son algunos ejemplos de categor\'{\i}as.
    \begin{enumerate}
        \item[{\rm (i)}] La categor\'{\i}a $\Set$ cuyos objetos son todos los conjuntos y cuyos morfismos son la aplicaciones entre conjuntos
        \item[{\rm (ii)}] La categor\'{\i}a $\Grp$ cuyos objetos son los grupos y cuyos morfismos son los morfismos de grupo.
        \item[{\rm (iii)}] La categor\'{\i}a $\Top$ cuyos objetos son los espacios topol\'ogicos y cuyos morfismos son las aplicaciones continuas.
    \end{enumerate}
\end{ex}

\begin{defi}
    Dada una categor\'{\i}a $\mathcal{C}$, podemos definir su \emph{categor\'{\i}a opuesta} $\mathcal{C}^{\rm op}$ de la siguiente manera. Los objectos de $\mathcal{C}^{\rm{op}}$ son los mismos que los de $\mathcal{C}$, los morfismos cambian de direcci\'on $\mathcal{C}^{\rm op}(A,B)=\mathcal{C}(B,A)$ y la funci\'on de composici\'on es $f\circ^{\rm op}g=g\circ f$.
\end{defi}

\subsubsection{Funtores}
\begin{defi}
    Sean $\mathcal{C}$ y $\mathcal{D}$ dos categor\'{\i}as. Un \emph{funtor} $F$ de $\mathcal{C}$ en $\mathcal{D}$, que denotaremos por $F\colon\mathcal{C}\to \mathcal{D}$ consiste en:
    \begin{itemize}
        \item Una aplicaci\'on ${\rm Ob}(\mathcal{C})\to {\rm Ob}(\mathcal{D})$. La imagen de un objeto $A$ de $\mathcal{C}$ la denotaremos por $F(A)$
        \item Para cada par de objetos $A,B\in\mathcal{C}$ una aplicaci\'on
              $$
                  \mathcal{C}(A,B)\longrightarrow\mathcal{D}(F(A), F(B)).
              $$
              La imagen de un morfismo $f\colon A\to B$ por esta aplicaci\'on la denotaremos por $F(f)\colon F(A)\to F(B)$.
    \end{itemize}
    Adem\'as, estas aplicaciones son compatibles con la composici\'on y la unidad, esto es, se cumplen los siguientes axiomas:
    \begin{itemize}
        \item Dados $f\in\mathcal{C}(A,B)$ y $g\in\mathcal{C}(B,C)$ se cumple que $F(g\circ f)=F(g)\circ F(f)$.
        \item Para todo objeto $A\in\mathcal{C}$ se cumple que $F({\rm id}_A)={\rm id}_{F(A)}$.
    \end{itemize}
\end{defi}
\begin{obs}
    La noci\'on de funtor que acabamos se llama tambi\'en \emph{funtor covariante} de $\mathcal{C}$ en $\mathcal{D}$. Un funtor de $\mathcal{C}^{\rm op}$ en $\mathcal{D}$ se llama \emph{functor contravariante} de $\mathcal{C}$ en $\mathcal{D}$. Observar que si $F$ es un funtor contravariante de $\mathcal{C}$ en $\mathcal{D}$ y $f\colon A\to B$ es un morfismo en $\mathcal{C}$, entonces $F(f)\colon F(B)\to F(A)$.
\end{obs}
\begin{ex}
    Dado un conjunto $X$ cualquiera, podemos construir el grupo libre en los elementos de este conjunto $F(X)$. Esto define un funtor $F\colon\Set\to\Grp$.
\end{ex}

\begin{defi}
    Sea $F: \mathcal{C}\to \mathcal{D}$ un funtor entre dos categor\'{\i}as $\mathcal{C}$ y $\mathcal{D}$. Dados un par de objetos $A,B\in\mathcal{C}$ consideremos la aplicaci\'on
    $$
        F_{A,B}\colon \mathcal{C}(A,B)\longrightarrow\mathcal{D}(F(A), F(B)).
    $$
    \begin{itemize}
        \item Diremos que $F$ es un funtor \emph{fiel} si para cada par de objetos $A, B\in\mathcal{C}$ la aplicaci\'on $F_{A,B}$ es inyectiva.
        \item Diremos que $F$ es un funtor \emph{pleno} si para cada par de objetos $A, B\in\mathcal{C}$ la aplicaci\'on $F_{A,B}$ es exhaustiva.
        \item Diremos que $F$ es un funtor \emph{plenamente fiel} si para cada par de objetos $A, B\in\mathcal{C}$ la aplicaci\'on $F_{A,B}$ es biyectiva.
    \end{itemize}
\end{defi}
\subsection{Op\'eradas en conjuntos}
% Bibliografía Ieke Moerdijk; Bertrand Toën. (2010), SImplicial Methods for Operads and Algebraic Geometry (PDF), Birkhäuser, ISBN 978-3-0348-0051-8
Para cada $n\ge 0$, denotaremos por $\Sigma_n$ el grupo sim\'etrico de $n$ letras (en el caso $n=0,1$, $\Sigma_n$ ser\'a el grupo trivial).
\begin{defi}
    Una \emph{op\'erada} $P$ consiste en una sucesi\'on de conjuntos $\{P(n)\}_{n\ge 0}$ junto con la siguiente estructura:
    \begin{itemize}
        \item Un elemento \emph{unidad} $1\in P(1)$.
        \item Un \emph{producto composici\'on}
              $$
                  P(n)\times P(k_1) \times\cdots\times P(k_n)\longrightarrow P(k)
              $$
              para cada $n$ y $k_1,\dots,k_n$ tal que $k=\sum_{i=1}^{n}{k_i}$.
        \item Para cada $\sigma\in\Sigma_n$ una \emph{acci\'on por la derecha} $\sigma^*\colon P(n)\to P(n)$.
    \end{itemize}
    Adem\'as el producto composici\'on es asociativo, equivariante y compatible con la unidad.
\end{defi}
%TODO: afirmaciones/axiomas? 

\begin{defi}
    Dadas dos op\'eradas $P$ y $Q$, un morfismo de op\'eradas $f\colon P\to Q$ consiste en aplicaciones $f_n\colon P(n)\to Q(n)$ para cada $n\ge 0$ compatibles con el producto composici\'on, la unidad y la acci\'on del grupo sim\'etrico.
\end{defi}

\subsubsection{Op\'eradas coloreadas}
La noci\'on de op\'erada coloreada generaliza a la vez el concepto de categor{\'i}a y de op\'erada.
\begin{defi}
    Sea $C$ un conjunto, cuyos elementos llameremos colores. Una op\'erada $C$-coloreada $P$ consiste en, para cada $(n+1)$-tupla de colores $(c_1,\ldots,c_n,c)$ con $n\ge 0$, un conjunto $P(c_1,\ldots, c_n;c)$ (que representar\'a el conjunto de operaciones cuyas entradas est\'an coloreadas por los colores $c_1,\ldots, c_n$ y cuya salida esta coloreada por $c$), junto con la siguiente estructura:
    \begin{itemize}
        \item Un elemento \emph{unidad} $1_c\in P(c;c)$ para cada $c\in C$.
        \item Un \emph{producto composici\'on}
              \begin{align*}
                  P(c_1,\dots,c_n;c) & \otimes P(d_{1,1},\dots,d_{1,k_1};c_1) \otimes\dots\otimes P(d_{n,1},\dots,d_{n,k_n};c_n) \\
                                     & \longrightarrow P(d_{1,1},\dots,d_{1,k_1},\dots,d_{n,1},\dots,d_{n,k_n};c)
              \end{align*}
              para cada $(n+1)$-tupla de colores $(c_1,\dots,c_n;c)$ y $n$ tuplas cualesquiera
              $$
                  (d_{1,1},\dots,d_{1,k_1};c_1),\dots,(d_{n,1},\dots,d_{n,k_n};c_n)
              $$
        \item Para cada elemento $\sigma\in\Sigma_n$ una \emph{acci\'on}
              $$
                  \sigma^{*}: P(c_1,\dots,c_n;c) \longrightarrow P(c_{\sigma(1)},\dots,c_{\sigma(n)};c).
              $$
    \end{itemize}
    Adem\'as el producto composici\'on es asociativo, equivariante y compatible con las unidades.
\end{defi}

\begin{defi}
    Sea $P$ una op\'erada $C$-coloreada y $Q$ una op\'erada $D$-coloreada. Un \emph{morfismo de op\'eradas} $f\colon P\to Q$ consiste en una aplicaciones entre los conjuntos de colores $f\colon C\to D$ y aplicaciones
    $$
        f_{c_1,\dots,c_n;c}: P(c_1,\dots,c_n;c) \longrightarrow Q(f(c_1),\dots,f(c_n);c)
    $$
    compatibles con el producto composici\'on, las unidades y la acci\'on del grupo sim\'etrico.
\end{defi}

Denotaremos por $\Oper$ la categor\'ia cuyos objetos son operadas coloreadas y cuyos morfismos son los morfismos de operadas coloreadas.

\begin{ex}
    Si $C=\{*\}$, entonces una op\'erada $C$-coloreada es lo mismo que una op\'erada. Si $P$ es una op\'erad $C$-coloreada tal que solamente tiene operaciones de aridad uno, es decir $P(c_1,\ldots, c_n;c)=\emptyset$ si $n\ne 1$, entonces $P$ es una categor\'{\i}a, cuyo conjunto de objetos es $C$.
\end{ex}
\newpage
% --- Acaba sección 1 ----------------------

% ------------------------------------------

% --- Empieza sección 2 --------------------
\section{Conjuntos Simpliciales}
% Bibliografía Greg Friedman. (2011), An elementary illustrated introduction to simplicial sets (PDF)
\subsection{Complejos simpliciales}
\begin{defi}
    N-simplex
\end{defi}
Un $n\text{-simplex}$ es un politopo de $n\ge 0$ dimensiones formando una envoltura convexa de $n+1$ vertices. Es decir, es un conjunto de puntos afines independientes en un espacio eucl\'ideo de dimensi\'on $n$.

Una cara $m$ de un $n\text{-simplex}$ es una envolutra convexa de $m\le n$ vertices.

\begin{defi}
    Complejo simplicial
\end{defi}
Sea $n\in\mathbb{N}^{*}$, un complejo simplicial $X$ es un conjunto finito de $m\text{-simplex}$ con $m\le n$ que cumplen las condiciones:

\begin{enumerate}[(1)]
    \item Si $m\text{-simplex}\in X \Rightarrow \forall m'\le m\text{, }m'\text{-simplex}\in X$.
    \item Si dos simplices de $X$ se cortan, entonces su intersecci\'on es una cara com\'un.
\end{enumerate}

Sea $X^k$ un complejo simplicial formado por todos los $k\text{-simplex}$ de $X$. Observamos que todo elemento de $X^k$ es un subconjunto de $X^0$ con cardinal $k+1$, donde $X^0=\{v_0,\dots ,v_n\}$.
Generalmente, todo subconjunto de $X^k$ de $j+1$ elementos es un elemento de $X^j$.

Sea $X_k$ un conjunto formado por $k\text{-simplices}$.

\begin{defi}
    N-simplex ordenado
\end{defi}
Un $n\text{-simplex}$ formado por los v\'ertices $v_0,\dots,v_n \in X^0$ es ordenado cuando cuando los v\'ertices estan ordenados, en ese caso nombramos cada v\'ertice por los n\'umeros $0,\dots,n$. Usaremos la notaci\'on $|\Delta^n| = [0,\dots,n]$ para simplificar.


\subsubsection{Morfismos simpliciales}
\begin{defi}
    Morfismo simplicial
\end{defi}
Sea $K$ y $L$ complejos simpliciales. Sea un morfismo simplicial $F: K \longrightarrow L$ que envia los vertices de K a los vertices de L. Es decir, $\forall v \in K^0 \text{, } v \longmapsto F(v) \in L^0$.

\begin{defi}
    Cara
\end{defi}
Para todo $|\Delta^n|$ tenemos $n+1$ caras definidas por los morfismos $\delta_0,\dots,\delta_n$
\begin{align*}
    \delta_j: X_n & \longrightarrow X_{n-1} \\
    [0,\dots,n]   & \longmapsto\!
    \begin{aligned}[t]
        [0,\dots,\hat{j},\dots,n]
    \end{aligned}
\end{align*}
Donde $X_n$ y $X_{n-1}$ son conjuntos de simplices ordenados de $n$ y $n-1$ v\'ertices, respectivamente. Observamos que $\forall i<j$, $\delta_i\delta_j = \delta_{j-1}\delta_{i}$.

\begin{defi}
    Morifismo degenerativo
\end{defi}
Para todo $|\Delta^n|$ tenemos $n+1$ morfismos degenerativos $\sigma_0,\dots,\sigma_n$
\begin{align*}
    \sigma_j: X_n & \longrightarrow X_{n+1} \\
    [0,\dots,n]   & \longmapsto\!
    \begin{aligned}[t]
        [0,\dots,j,j,\dots,n]
    \end{aligned}
\end{align*}
Donde $X_n$ y $X_{n+1}$ son conjuntos de simplices ordenados de $n$ y $n+1$ v\'ertices, respectivamente. Observamos que $\forall i\le j$, $\sigma_i\sigma_j = \sigma_{j+1}\sigma_{i}$.

\subsection{Conjunto Delta}
\begin{defi}
    Conjunto Delta
\end{defi}
Definimos un conjunto Delta como una secuencia de conjuntos $X_0,X_1,\dots$ y para cada $n\ge 0$ las funciones $\delta_i: X_{n+1} \longrightarrow X_n$, $\forall 0\le i \le n+1$, que cumplen $\delta_i\delta_j = \delta_{j-1}\delta_{i}$, $\forall i\le j$.
Formando el siguiente diagrama (Falta por hacer)

$$ %Not working
    \xymatrix{
    X_0  \ar@/{}^{1pc}/[r]  & X_1  \ar@/{}^{1pc}/[r] & X_2  \dots
    }
$$

\subsubsection{Definici\'on categ\'orica del conjunto Delta}
\begin{defi}
    Categor\'ia $\hat{\Delta}$
\end{defi}
Sea la categor\'ia $\hat{\Delta}$ cuyos objetos son los conjuntos estrictamente ordenados finitos $[n] = \{0,\dots,n\}$ y los morfismos son las funciones, que mantienen el orden estrictamente, $f: [m] \longrightarrow [n]$, $m\le n$. Podemos pensar que sea la inclusi\'on de un $m\text{-simplex}$ como cara de un $n\text{-simplex}$.
Para todo $0\le i \le n$ consideramos los morfismos:
\begin{align*}
    d_i: [n]      & \longrightarrow [n+1] \\
    \{0,\dots,n\} & \longmapsto\!
    \begin{aligned}[t]
        \{0,\dots, \hat{i}, \dots,n+1\}
    \end{aligned}
\end{align*}

\begin{defi}
    Categor\'ia $\hat{\Delta}^{op}$
\end{defi}
Sea la categor\'ia $\hat{\Delta}^{op}$, la categor\'ia opuesta de $\hat{\Delta}$, cuyos objetos son los conjuntos estrictamente ordenados finitos $[n] = \{0,\dots,n\}$ y los morfismos son las funciones, que mantienen el orden estrictamente, $f: [n] \longrightarrow [m]$, $m\le n$. Podemos pensar que sea la extracci\'on de la cara $m\text{-simplex}$ de un $n\text{-simplex}$.
Para todo $0\le i \le n$ consideramos los morfismos:
\begin{align*}
    \delta_i: [n] & \longrightarrow [n-1] \\
    \{0,\dots,n\} & \longmapsto\!
    \begin{aligned}[t]
        \{0,\dots, \hat{i}, \dots,n\}
    \end{aligned}
\end{align*}

\begin{defi}
    Conjunto Delta
\end{defi}
Un conjunto Delta es un functor covariante $X: \hat{\Delta}^{op} \longrightarrow \mathbf{Set}$, equivalentemente es un functor contravariante $X: \hat{\Delta} \longrightarrow \mathbf{Set}$.

Faltan observaciones.

\subsection{Conjunto simplicial}
\begin{defi}
    Conjunto simplicial
\end{defi}
Definimos un conjunto simplicial como una secuencia de conjuntos $X_0,X_1,\dots$ y para cada $n\ge 0$ las funciones $\delta_i: X_n \longrightarrow X_{n-1}$ y $\sigma_i: X_n \longrightarrow X_{n+1}$, $\forall 0\le i \le n$, que cumplen:

\begin{enumerate}[(1)]
    \item $\delta_i\delta_j = \delta_{j-1}\delta_{i}$, $i<j$
    \item $\delta_i\sigma_j = \sigma_{j-1}\delta_{i}$, $i<j$
    \item $\delta_j\sigma_j = \delta{j+1}\sigma_{j} = id$
    \item $\delta_i\sigma_j = \sigma_{j}\delta_{i-1}$, $i>j+1$
    \item $\sigma_i\sigma_j = \sigma_{j+1}\sigma_{i}, i\le j$
\end{enumerate}
Formando el siguiente diagrama (Falta por hacer)

$$ %Not working
    \xymatrix{
    X_0  \ar@/{}^{1pc}/[r]  & X_1  \ar@/{}^{1pc}/[r] & X_2  \dots
    }
$$

\subsubsection{Definici\'on categ\'orica del conjunto simplicial}
\begin{defi}
    Categor\'ia $\Delta$
\end{defi}
Sea la categor\'ia $\Delta$ cuyos objetos son los conjuntos ordenados finitos $[n] = \{0,\dots,n\}$ y los morfismos son las funciones, que mantienen solamente el orden, $f: [m] \longrightarrow [n]$.
Para todo $0\le i \le n$ consideramos los morfismos:
\begin{align*}
    d_i: [n]      & \longrightarrow [n+1] \\
    \{0,\dots,n\} & \longmapsto\!
    \begin{aligned}[t]
        \{0,\dots, \hat{i}, \dots,n+1\}
    \end{aligned}
\end{align*}
\begin{align*}
    s_i: [n+1]      & \longrightarrow [n] \\
    \{0,\dots,n+1\} & \longmapsto\!
    \begin{aligned}[t]
        \{0,\dots,i,i, \dots,n\}
    \end{aligned}
\end{align*}

\begin{defi}
    Categor\'ia $\hat{\Delta}^{op}$
\end{defi}
Sea la categor\'ia $\Delta^{op}$, la categor\'ia opuesta de $\Delta$, cuyos objetos son los conjuntos ordenados finitos $[n] = \{0,\dots,n\}$ y los morfismos son las funciones, que mantienen solamente el orden, $f: [m] \longrightarrow [n]$.
Para todo $0\le i \le n$ consideramos los morfismos:
\begin{align*}
    \delta_i: [n] & \longrightarrow [n-1] \\
    \{0,\dots,n\} & \longmapsto\!
    \begin{aligned}[t]
        \{0,\dots, \hat{i}, \dots,n\}
    \end{aligned}
\end{align*}
\begin{align*}
    \sigma_i: [n] & \longrightarrow [n+1] \\
    \{0,\dots,n\} & \longmapsto\!
    \begin{aligned}[t]
        \{0,\dots, i,i, \dots,n\}
    \end{aligned}
\end{align*}

\begin{defi}
    Conjunto simplicial
\end{defi}
Un conjunto simplicial es un functor covariante $X: \Delta^{op} \longrightarrow \mathbf{Set}$, equivalentemente es un functor contravariante $X: \Delta \longrightarrow \mathbf{Set}$.
Usaremos la notaci\'on $\Delta[n]=\Delta(\_,[n])$.
\begin{align*}
    \Delta[n]: \Delta^{op} & \longrightarrow \mathbf{Set} \\
    [m]                    & \longmapsto\!
    \begin{aligned}[t]
        \Delta([m],[n])
    \end{aligned}
\end{align*}
Faltan observaciones.

\subsection{Realizaci\'on geom\'etrica}
\begin{defi}
    Realizaci\'on geom\'etrica
\end{defi}
Sea $X$ un conjunto simplicial. Dotamos cada $X_n$ con la topolog\'ia discreta y sea $|\Delta^n$ el $n\text{-simplex}$ dotado de su topolog\'ia estandard. Definimos la realizaci\'on geom\'etrica como
$$
    |X| = \coprod_{n=0}^{\infty}X_n \times |\Delta^n| / \sim
$$
Donde $\sim$ es la relaci\'on de equivalencia generada por las relaciones:

\begin{enumerate}[(1)]
    \item $(x, d_i(p)) \sim (\delta_i(x), p)$, $x\in X_{n+1}$ y $p\in|\Delta^n|$
    \item $(x, s_i(p)) \sim (\sigma_i(x), p)$, $x\in X_{n-1}$ y $p\in|\Delta^n|$
\end{enumerate}

\begin{ex}
    $\Delta[2] = \Delta(\_, [2])$
\end{ex}
Falta por escribir

% --- Acaba sección 2 ----------------------

% ------------------------------------------

% --- Empieza sección 3 --------------------
\newpage
\section{Conjuntos Dendroidales}

\subsection{\'Arbol como operadas}
\subsubsection{Formalismo de \'arboles}
Un \emph{\'arbol} es un grafo no vac\'io, finito, conectado y sin lazos. Llamaremos un \emph{v\'ertice exterior} si tiene solamente una arista adjunta.
Todos los \'arboles que consideraremos tendr\'an \emph{ra\'iz}, es decir, para cada \'arbol existe un v\'ertice exterior, llamado \emph{output} o \emph{salida}, donde se ve claramente que tiene un conjunto
de v\'ertices exteriores, llamado \emph{inputs} o \emph{entradas}. Este \'ultimo conjunto puede ser vac\'io y no contiene el v\'ertice output.

Para dibujar dichos \'arboles, borraremos los v\'ertices output e inputs de la figura. De tal manera que los v\'ertices restantes ser\'an los \emph{v\'ertices} del \'arbol.
Dado un \'arbol T, definimos el conjunto de v\'ertices como $V(T)$ y el conjunto de aristas como $E(T)$.

Llamaremos \emph{hojas} o \emph{aristas externas} a las aristas adjuntas de los v\'ertices inputs y \emph{ra\'iz} a la arista adjunta del v\'ertice output.
De tal manera que las aristas restantes las llamaremos \emph{aristas internas}. Podemos observar que existe una direcci\'on clara en cada \'arbol, desde las hojas hasta la ra\'iz.

Sea \emph{v} un v\'ertice de un \'arbol finito con ra\'iz, definimos out$(v)$ como la \'unica arista de salida y in$(v)$ como el conjunto de aristas de entrada, observamos que este \'ultimo conjunto puede ser vac\'io.
Llamaremos la \emph{valencia} de $v$ a la cardinalidad del conjunto in$(v)$.

Finalmente, consideramos a la siguiente figura como un \'arbol de ejemplo:
% Figura arbol + explicación de sus partes %

\subsubsection{\'Arboles planares}
\begin{defi}
    Un \emph{\'arbol planar con ra\'iz} es un \'arbol con ra\'iz $T$ dotado con un orden lineal del conjunto in$(v)$ para cada $v$ de $T$.
\end{defi}
\begin{obs}
    El orden de los conjuntos in$(v)$ se obtiene de la idea de dibujar los \'arboles en un plano. Es decir, para dibujar un \'arbol siempre pondremos la ra\'iz debajo y las hojas arriba con un orden trivial.
    Observamos con esta t\'ecnica que tendremos varias representaciones planares del mismo \'arbol. Por ejemplo,
    % Figura de dos representaciones planares del mismo arbol %
\end{obs}
\begin{defi}
    Un \'arbol es \emph{unitario} cuando la arista de entrada y salida son la misma. En ese caso lo denotaremos como $\eta$.
\end{defi}
\begin{defi}
    Sea $T$ un \'arbol planar con ra\'iz. Denotaremos una op\'erada coloreada no-sim\'etrica generada por T como $\Omega_p(T)$. El conjunto de colores de $\Omega_p(T)$ es el conjunto de aristas $E(T)$ de $T$ y las operaciones son generadas por los v\'ertices del \'arbol.
    Es decir, para cada v\'ertice $v$ con entradas $e_1,\dots,e_n$ y salida $e$, definimos una operaci\'on $v\in \Omega_p(T)(e_1,\dots,e_n;e)$. Las otras operaciones son las operaciones unitarias y las operaciones obtenidas por composici\'on.
\end{defi}
\begin{obs}
    Para todo $e_1,\dots,e_n,e$, el conjunto de operaciones $\Omega_p(T)(e_1,\dots,e_n;e)$ contiene como mucho un solo elemento.
\end{obs}
\begin{ex}
    Vamos a realizar la descripci\'on completa del siguiente \'arbol T:
    % Figura de un árbol para describir %

    La operada $\Omega_p(T)$ tiene seis colores \textit{a, b, c, d, e,} y \textit{f}. Las operaciones generadoras son $v\in \Omega_p(T)(e,f;b)$, $w\in \Omega_p(T)(\_;d)$ y $r\in \Omega_p(T)(b,c,d;a)$.
    Mientras que las otras operaciones son las operaciones unitarias $1_a,1_b,\dots,1_f$ y las operaciones composici\'on $r\circ_1v\in \Omega_p(T)(e,f,c,d;a)$, $r\circ_2w\in \Omega_p(T)(b,c;a)$ y
    $$(r\circ_1 v)\circ_3 w = (r\circ_2 w)\circ_1 v  \in \Omega_p(T)(e,f,c;a)$$
\end{ex}
\begin{defi}
    La \emph{categor\'ia de \'arboles planares con ra\'iz} $\Omega_p$ es la subcategor\'ia plena de la categor\'ia de op\'eradas coloreadas no-sim\'etricas cuyos objetos son $\Omega_p(T)$ para todo \'arbol $T$.

    Podemos pensar que $\Omega_p$ es una categor\'ia cuyos objetos son \'arboles planares con ra\'iz.
    Sean $S$ y $T$ dos \'arboles planares con ra\'iz, el conjunto de morfismos $\Omega_p(S, T)$ es dado por los morfismos entre op\'eradas coloreadas no-sim\'etricas de $\Omega_p(S)$ a $\Omega_p(T)$.
\end{defi}
\begin{obs}
    La categor\'ia $\Omega_p$ extiende la categor\'ia simplicial $\Delta$. Para todo $n\ge 0$ se define un \emph{\'arbol lineal} $L_n$ como un \'arbol planar con $n+1$ aristas y $n$ v\'ertices $v_1,\dots,v_n$, donde la valencia de todos los v\'ertices es uno.
    % Figura del arbol L_n %

    Denotaremos este \'arbol por $[n]$. Toda apliaci\'on que mantiene el orden de manera que env\'ie $\{0,\dots,n\}$ a $\{0,\dots,m\}$, define un morfismo $[n] \to [m]$ en la categor\'ia $\Omega_p$. De esta manera obtenemos el siguiente funtor
    \[
        \begin{tikzcd}
            \Delta \arrow[hook]{r}{i} & \Omega_p
        \end{tikzcd}
    \]
    Este funtor es plenamente fiel. Es decir, para toda flecha $S \to T$ en $\Omega_p$, si $T$ es lineal entonces $S$ tambi\'en lo es. (Falta demostraci\'on)
\end{obs}

\subsection{Morfismos en $\Omega_p$}
En las siguientes secciones vamos a tratar con todos los tipos de morfismos en $\Omega_p$ y dar una descripci\'on m\'as expl\'icita.
\subsubsection{Caras}
Sea $T$ un \'arbol planar con ra\'iz.
\begin{defi}
    Una \emph{cara interna} asociada a una arista interna $b$ en $T$ es una funci\'on $\partial_b \colon T/b\to T$ en $\Omega_p$, donde $T/b$ es el \'arbol que se obtiene al contraer la arista $b$.

    Esta funci\'on es una inclusi\'on de los colores y de las operaciones generadoras de $\Omega_p(T/b)$, excepto por la operaci\'on $u$, que se env\'ia a la composici\'on $r\circ_b v$.
    Donde $r$ y $v$ son dos v\'ertices en $T$ con la arista $b$ entre ellos, y $u$ es el v\'ertice correspondiente en $T/b$. Tomamos la siguiente figura para visualizar la funci\'on.
    % Figura de la funcion T/b -> T %
\end{defi}
\begin{defi}
    Una \emph{cara externa} asociada a un v\'ertice $v$ en $T$, con solo una arista interna adjunta, es una funci\'on $\partial_v \colon T/v\to T$ en $\Omega_p$, donde $T/v$ es el \'arbol que se obtiene al cortar el v\'ertice $v$ con todas sus aristas externas.

    Esta funci\'on es una inclusi\'on de los colores y de las operaciones generadoras de $\Omega_p(T/v)$.
    Donde $r$ y $v$ son dos v\'ertices en $T$ con la arista $b$ entre ellos, y $u$ es el v\'ertice correspondiente en $T/b$. Tenemos dos tipos de cara externa que mostramos en las siguientes figuras.
    % Figura de la funcion T/v -> T con vertice con hojas y la funcion T/v -> T con vertice sin hojas %
\end{defi}
\begin{obs}
    Con esta \'ultima definici\'on no queda exclu\'ida la posibilidad de cortar la ra\'iz. Esta situaci\'on solo sera posible si la ra\'iz tiene solamente una arista interna adjunta. Entonces, no todo \'arbol $T$ tiene una cara externa asociada a su ra\'iz.
\end{obs}
\begin{obs}
    Vale la pena mencionar un caso en especial, la inclusi\'on de un \'arbol sin v\'ertices $\eta$ en un \'arbol con un v\'ertice, llamado \emph{corola}. En este caso tendremos $n+1$ caras si la corola tiene $n$ hojas.
    La op\'erada $\Omega_p(\eta)$ consiste solamente de un color y la operaci\'on identidad de dicho color. Entonces, una funci\'on de op\'eradas $\Omega_p(\eta)\to\Omega(T)$ es simplemente un color de una corola $T$.
\end{obs}

Para concluir, llamaremos \emph{caras} tanto las caras internas como las caras externas.

\subsubsection{Funciones degenerativas}
Sea $T$ un \'arbol planar con ra\'iz y $v$ un v\'ertice de valencia uno en $T$.
\begin{defi}
    Una \emph{funci\'on degenerativa} asociada al v\'ertice $v$ es una funci\'on $\sigma_v\colon T\to T \backslash  v$ en $\Omega_p$, donde $T \backslash  v$ es el \'arbol que se obtiene al cortar el v\'ertice $v$ y juntar las dos aristas adjuntas en una nueva arista $e$.

    Esta funci\'on env\'ia los colores $e_1$ y $e_2$ de $\Omega_p(T)$ al color $e$ de $\Omega_p(T\backslash v)$ y env\'ia la operacion generativa $v$ a la operaci\'on identidad $id_e$, mientras que es la identidad para los colores y operaciones generativas restantes.
    Tomamos la siguiente figura para visualizar la funci\'on.
    % Figura de la funcion T -> T\v %
\end{defi}

\begin{obs}
    Las caras y las funciones degenerativas generan toda la categor\'ia $\Omega_p$.
\end{obs}
El siguiente lema es una generalizaci\'on hacia $\Omega_p$ del lema en la categor\'ia $\Delta$, diciendo que toda flecha en dicha categor\'ia se puede escribir como composici\'on de funciones degenerativas seguidas por caras.

\begin{lema}
    Sea $S$, $T$ y $H$ unos \'arboles en $\Omega_p$, toda flecha $f\colon S \to T$ en $\Omega_p$ descompone, salvo isomorf\'ias, como
    $$
        \xymatrix{
            S \ar[rd]_{\sigma} \ar[r]^f
            &T \\
            &H \ar[u]_\partial
        }
    $$
    donde  $\sigma\colon S\to H$ es una composici\'on de funciones degenerativas y $\partial\colon H\to T$ es una composici\'on de caras.
\end{lema}
Demo: por hacer

\subsubsection{Identidades dendroidales}
En esta secci\'on vamos a dar las relaciones entre los morfismos generadores de $\Omega_p$. Las identidades que obtenemos generalizan los morfismos en la categor\'ia $\Delta$.

\subsubsection*{Relaciones elementales de caras}
Sea $\partial_a \colon T/a\to T$ y $\partial_b \colon T/b\to T$ dos caras internas distintas de $T$.
Seguidamente tenemos las caras internas $\partial_a \colon (T/b)/a \to T/b$ y $\partial_b \colon (T/a)/b \to T/a$. Observamos que $(T/a)/b = (T/b)/a$, entonces el siguiente diagrama conmuta:
$$
    \xymatrix{
        (T/a)/b \ar[d]_{\partial_a} \ar[r]^{\partial_b}
        &T/a \ar[d]^{\partial_a} \\
        T/b \ar[r]^{\partial_b}
        &T
    }
$$

Sea $\partial_v \colon T/v\to T$ y $\partial_w \colon T/w\to T$ dos caras externas distintas de $T$, y asumimos que $T$ tiene como m\'inimo tres v\'ertices.
Seguidamente tenemos las caras externas $\partial_v \colon (T/w)/v \to T/w$ y $\partial_w \colon (T/v)/w \to T/v$. Observamos que $(T/v)/w = (T/w)/v$, entonces el siguiente diagrama conmuta:
$$
    \xymatrix{
        (T/v)/w \ar[d]_{\partial_v} \ar[r]^{\partial_w}
        &T/v \ar[d]^{\partial_v} \\
        T/w \ar[r]^{\partial_w}
        &T
    }
$$

En el caso que $T$ solo tenga dos v\'ertices, existe un diagrama conmutativo similar mediante la inclusi\'on de $\eta$ a una $n$-corla.
\'Ultimo caso componer interna con externa. - por hacer!

\subsubsection*{Relaciones elementales de funciones degenerativas}
Sea $\sigma_v \colon T\to T\backslash v$ y $\sigma_w \colon T\to T\backslash w$ dos funciones degenerativas distintas de $T$.
Seguidamente tenemos las funciones degenerativas $\sigma_v \colon T\backslash w \to (T\backslash w)\backslash v$ y $\sigma_w \colon T\backslash v \to (T\backslash v)\backslash w$. Observamos que $(T\backslash v)\backslash w = (T\backslash w)\backslash v$, entonces el siguiente diagrama conmuta:
$$
    \xymatrix{
        T \ar[d]_{\sigma_v} \ar[r]^{\sigma_w}
        &T\backslash w \ar[d]^{\sigma_v} \\
        T\backslash v \ar[r]^{\sigma_w}
        &(T\backslash v)\backslash w
    }
$$

\subsubsection*{Relaciones combinadas}
Sea $\sigma_v\colon T\to T\backslash v$ una funci\'on degenerativa y $\partial\colon T' \to T$ es una cara de tal manera que la funci\'on degenerativa $\sigma_v\colon T'\to T'\backslash v$ esta bien definida. Entonces existe una cara $\partial\colon T'\backslash v \to T\backslash v$ determinada por el mismo v\'ertice o arista que $\partial\colon T' \to T$.
Adem\'as, el siguiente diagrama conmuta:
$$
    \xymatrix{
        T  \ar[r]^{\sigma_v}
        &T\backslash v \\
        T' \ar[u]^{\partial} \ar[r]^{\sigma_v}
        &T'\backslash v \ar[u]_{\partial}
    }
$$

Sea $\sigma_v\colon T\to T\backslash v$ una funci\'on degenerativa y $\partial\colon T' \to T$ es una cara interna en una arista adjunta a $v$ o una cara externa en $v$, si es posible. Entonces, tenemos que $T'=T\backslash v$ y la composici\'on
$    T\backslash v \overset{\partial}{\longrightarrow} T \overset{\sigma_v}{\longrightarrow} T\backslash v$ es la funci\'on identidad $id_{T\backslash v}$.


\subsection{\'Arboles no planares}
\begin{defi}
    Sea $T$ un \'arbol no-planar. Denotaremos una op\'erada coloreada sim\'etrica generada por T como $\Omega(T)$. El conjunto de colores de $\Omega(T)$ es el conjunto de aristas $E(T)$ de $T$.
    Las operaciones son generadas por los v\'ertices del \'arbol, y el grupo sim\'etrico de $n$ letras $\Sigma_n$ act\'ua en cada operaci\'on de $n$ entradas permutando el orden de las entradas.
    Es decir, para cada v\'ertice $v$ con entradas $e_1,\dots,e_n$ y salida $e$, definimos una operaci\'on $v\in \Omega(T)(e_1,\dots,e_n;e)$. Las otras operaciones son las operaciones unitarias, las operaciones obtenidas por composici\'on y la acci\'on del grupo sim\'etrico.
\end{defi}
\begin{ex}
    Consideramos la figura del siguiente \'arbol T:
    % Figura de un árbol para describir %

    La op\'erada $\Omega(T)$ tiene seis colores \textit{a, b, c, d, e,} y \textit{f}. Las operaciones generadoras son las mismas que las operaciones generativas en $\Omega_p(T)$. Observamos que toda operaci\'on de $\Omega_p(T)$ son operaciones de $\Omega(T)$, pero no a la inversa ya que hay m\'as operaciones en $\Omega(T)$ obtenidas por la acci\'on del grupo sim\'etrico.
    Por ejemplo, sea $\sigma$ la transposici\'on de dos elementos de $\Sigma_2$, entonces tenemos una operaci\'on $v\circ\sigma\in\Omega(f,e;b)$.
\end{ex}
\begin{obs}
    Sea $T$ cualquier \'arbol, entonces $\Omega(T) = \Sigma(\Omega_p(\overline{T}))$, donde $\overline{T}$ es una representaci\'on planar de $T$. De hecho, se elige una estructura planar de $T$ como generador de $\Omega(T)$.
\end{obs}
\begin{defi}
    La \emph{categor\'ia de \'arboles con ra\'iz} $\Omega$ es la subcategor\'ia completa$|$llena de la categor\'ia de op\'eradas coloreadas cuyos objetos son $\Omega(T)$ para todo \'arbol $T$.

    Podemos pensar que $\Omega$ es una categor\'ia cuyos objetos son \'arboles con ra\'iz.
    Sean $S$ y $T$ dos \'arboles con ra\'iz, el conjunto de morfismos $\Omega(S, T)$ es dado por los morfismos entre op\'eradas coloreadas de $\Omega(S)$ a $\Omega(T)$.
\end{defi}
\begin{obs}
    Los morfismos de la categor\'ia $\Omega$ son generados por las caras y las funciones degenerativas, an\'alogas al caso planar, y las isomorf\'ias no-planares.
\end{obs}
\begin{lema}
    Sea $S$, $S'$, $T$ y $T'$ unos \'arboles en $\Omega$, toda flecha $f\colon S \to T$ en $\Omega$ descompone como
    $$
        \xymatrix{
            S \ar[d]_{\sigma} \ar[r]^f
            &T \\
            S' \ar[r]^\varphi
            &T' \ar[u]_\partial
        }
    $$
    donde  $\sigma\colon S\to S'$ es una composici\'on de funciones degenerativas, $\varphi\colon S'\to T'$ es un isomorf\'ia, y $\partial\colon T\to T'$ es una composici\'on de caras.
\end{lema}
Demo: por hacer

\subsubsection{Prehaz? de estructuras planares}
Sea $P:\Omega^{\rm op} \to \Set$ el prehaz en $\Omega$ que env\'ia cada \'arbol a su conjunto de estructuras planares. Observamos que $P(T)$ es un torsor en $\text{Aut}(T)$ para cada \'arbol $T$, donde $\text{Aut}(T)$ denota el conjunto de automorfismos de $T$.
Recordamos que la categor\'ia $\Omega\backslash P$ es la categor\'ia cuyos objetos son pares $(T,x)$ con $x\in P(T)$. Sean $(T,x)$ y $(S,y)$ dos objetos, un morfismo entre ellos es dado por el morfismo $f\colon\to S$ en $\Omega$, tal que $P(f)(y) = x$.
Entonces, tenemos que $\Omega\backslash P = \Omega_p$ y existe una proyecci\'on $v\colon\Omega_p\to\Omega$. Tenemos el siguiente tri\'angulo conmutativo:
$$
    \xymatrix{
        \Delta \ar[rd]_{i} \ar[r]^u
        &\Omega_p \ar[d]^v\\
        &\Omega
    }
$$
Donde $i$ es un encaje plenamente fiel  de $\Delta$ hacia $\Omega$, que env\'ia el objeto $[n]$ de $\Delta$ al \'arbol lineal $L_n$ de $\Omega$, para todo $n\ge 0$.
\subsubsection{Relaci\'on con la categor\'ia simplicial}

Hemos podido ver que las dos categor\'ias, $\Omega_p$ y $\Omega$, extienden la categor\'ia $\Delta$, gracias a ver los objetos de $\Delta$ como \'arboles lineales. Adem\'as, se puede obtener $\Delta$ como la categor\'ia coma de $\Omega_p$ o $\Omega$.

Sea $\eta$ un \'arbol en $\Omega$ que no contiene ning\'un v\'ertice y tan solo una arista, y sea $\eta_p$ su representaci\'on planar en $\Omega_p$.
Si $T$ es un \'arbol cualquiera en $\Omega$, entonces $\Omega(T,\eta)$ consiste en un solo morfismo o es el conjunto vac\'io, dependiendo si $T$ es un \'arbol lineal o no. Pasa lo mismo con $\Omega_p$ y $\eta_p$. Entonces, $\Omega\backslash\eta = \Omega_p\backslash\eta_p = \Delta$.

\subsection{Conjunto Dendroidal}
En esta secci\'on vamos a introducir nociones b\'asicas y terminolog\'ia para la categor\'ia de los conjuntos dendroidales. Describiremos la categor\'ia de los conjuntos dendroidales y los conjuntos dendroidales planares como categor\'ias de los prehaces en $\Omega$ y $\Omega_p$, respectivamente.
Hemos visto la relaci\'on entre estas categor\'ias con la categor\'ia de conjuntos simpliciales y la categor\'ia de las op\'eradas, mediante la uni\'on 'adjoint' natural de funtores entre ellas. M\'as adelante definiremos un nervio dendroidal, desde op\'eradas hacia conjuntos dendroidales, generalizando as\'i la construcci\'on cl\'asica del nervio, desde categor\'ias peque\~{n}as hacia conjuntos simplicales.

\begin{defi}
    La categor\'ia $d\Set$ de \emph{conjuntos dendroidales} es la categor\'ia de prehaces en $\Omega$. Los objetos son funtors $\Omega^{\rm op}\to\Set$ y los morfismos vienen de las transformaciones naturales. La categor\'ia $pd\Set$ de \emph{conjuntos dendroidales planares} esta definida de manera an\'aloga intercambiando $\Omega$ por $\Omega_p$.

    Entonces, un conjunto dendroidal $X$ viene definido como un conjunto $X(T)$, denotado por $X_T$, para cada \'arbol $T$, conjuntamente con una funci\'on $\alpha^{*}\colon X_T \to X_S$ para cada morfismo $\alpha\colon S\to T$ en $\Omega$. Como $X$ es un funtor, entonces $(id)^{*}=id$ y si $\alpha\colon S\to T$ y $\beta\colon R\to S$ son morfismos en $\Omega$, entonces $(\alpha\circ\beta)^{*}=\beta^{*}\circ\alpha^{*}$. El conjunto $X_T$ lo llamaremos conjunto de \emph{dendrices con forma T}, o simplemente como el conjunto de $T$-dendrices.

    Sean $X$ y $Y$ dos conjuntos dendroidales, un \emph{morfismo de conjuntos dendroidales} $f\colon X \to Y$ viene definido por las funciones $f\colon X_T\to Y_T$, para cada \'arbol $T$, conmutando con las funciones de estructura. Es decir, si $\alpha\colon S\to T$ es cualquier morfismo en $\Omega$ y $x\in X_T$, entonces $f(\alpha^{*}x)=\alpha^{*}f(x)$.

    Decimos que $Y$ es un \emph{subconjunto dendroidal} de $X$ si para cada \'arbol $T$ tenemos que $Y_T\subseteq X_T$ y la inclusi\'on $Y \hookrightarrow X$ es un morfismo de conjuntos dendroidales.
\end{defi}


\subsection{Producto tensorial de conjuntos dendroidales}
\begin{defi}
    Sea $P$ una op\'erada sim\'etrica $C$-coloreada, y sea $Q$ una op\'erada sim\'etrica $D$-coloreada. El \emph{producto tensorial Boardman-Vogt} $P\otimes_{BV}Q$ es una op\'erada $(C\times D)$-coloreada definida en terminos de genreadores y relaciones de la siguiente manera.
    Para cada color $d\in D$ y cada operaci\'on $p\in P(c_1,\dots,c_n;c)$ existe un generador
    $$
        p \otimes d \in P\otimes_{BV}Q((c_1,d),\dots,(c_n,d);(c,d))
    $$
    De manera an\'aloga, para cada color $c\in C$ y cada operaci\'on $q\in Q(d_1,\dots,d_m;d)$ existe un generador
    $$
        c \otimes q \in P\otimes_{BV}Q((c,d_1),\dots,(c,d_m);(c,d))
    $$
    Estos generadores estan sujetos a las siguientes relaciones:
    \begin{enumerate}
        \item[{\rm (i)}] $(p\otimes d) \circ ((p_1\otimes d),\dots,(p_n\otimes d)) = (p\circ(p_1,\dots,p_n))\otimes d$
        \item[{\rm (ii)}] $\sigma^{*}(p\otimes d) = (\sigma^{*}p)\otimes d$, para cada $\sigma\in\Sigma_n$
        \item[{\rm (iii)}] $(c\otimes q) \circ ((c\otimes q_1),\dots,(c\otimes q_m)) = c\otimes (q\circ(q_1,\dots,q_m))$
        \item[{\rm (iv)}] $\sigma^{*}(c\otimes q) = c\otimes (\sigma^{*}q)$, para cada $\sigma\in\Sigma_m$
        \item[{\rm (v)}] Falta
    \end{enumerate}
\end{defi}


\subsubsection{Producto tensorial Boardman Vogt}
\subsubsection{Producto Producto tensorial de conjuntos dendroidales}
\newpage
% --- Acaba sección 3 ----------------------

% ------------------------------------------

% --- Empieza sección 4 --------------------
\section{Injertos de \'arboles}
\subsection{Producto tensorial de \'arboles lineales}
\subsection{Producto tensorial de \'arboles}
\subsubsection{Injertos de \'arboles resultantes}
\subsection{C\'alculo de \'arboles resultantes}
\subsubsection{Conjunto de \'arboles resultantes}
\subsubsection{Generarador de \'arboles en Python}
\newpage
% --- Acaba sección 4 ----------------------

% ------------------------------------------

% --- Empieza sección 5 --------------------
\section{Conclusiones}
\newpage
% --- Acaba sección 5 ----------------------

% --- Acaban las secciones -----------------

% --- Empieza la bibliografía ---
\begin{thebibliography}{25}
    \bibitem{pari} Batut, C.; Belabas, K.; Bernardi, D.; Cohen, H.; Olivier, M.: User's guide to \textit{PARI-GP},  \newline \texttt{pari.math.u-bordeaux.fr/pub/pari/manuals/2.3.3/users.pdf}, 2000.
    \bibitem{cw} Chen, J. R.; Wang, T. Z.: On the Goldbach problem, \textit{Acta Math. Sinica}, 32(5):702-718, 1989.
    \bibitem{desh} Deshouillers, J. M.: Sur la constante de $\check{\text{S}}\text{nirel}^{\prime} \text{man}$, \textit{S\'eminaire Delange-Pisot-Poitou, 17e ann\'ee: (1975/76), Th\'eorie des nombres: Fac. 2, Exp. No.} G16, p\'ag. 6, Secr\'etariat Math., Paris, 1977.
    \bibitem{derz} Deshouillers, J. M.; Effinger, G.; te Riele, H.; Zinoviev, D.: A complete Vinogradov 3-primes theorem under the Riemann hypothesis, \textit{Electron. Res. Announc. Amer. Math. Soc.}, 3:99-104, 1997.
    \bibitem{dick} Dickson, L. E.: \textit{History of the theory of numbers. Vol. I: Divisibility and primality}, Chelsea Publishing Co., New York, 1966.
    \bibitem{hl} Hardy, G. H.; Littlewood, J. E.: Some problems of \textquoteleft Partitio numerorum\textquoteright; III: On the expression of a number as a sum of primes, \textit{Acta Math.}, 44(1):1-70, 1923.
    \bibitem{hara} Hardy, G. H.; Ramanujan, S.: Asymptotic formulae in combinatory analysis, \textit{Proc. Lond. Math. Soc.}, 17:75-115, 1918.
    \bibitem{haw} Hardy, G. H.; Wright, E. M.: \textit{An introduction to the theory of numbers}, 5a edici\'on, Oxford University Press, 1979.
    \bibitem{minarc} Helfgott, H. A.: Minor arcs for Goldbach's problem, \newline \texttt{arXiv:1205.5252v4 [math.NT]}, diciembre de 2013.
    \bibitem{majarc} Helfgott, H. A.: Major arcs for Goldbach's problem, \newline \texttt{arXiv:1305.2897v4 [math.NT]}, abril de 2014.
    \bibitem{istrue} Helfgott, H. A.: The ternary Goldbach conjecture is true, \newline \texttt{arXiv:1312.7748v2 [math.NT]}, enero de 2014.
    \bibitem{HP} Helfgott, H. A.; Platt, D.: Numerical verification of the ternary Goldbach conjecture up to $8.875 \cdot 10^{30}$, \texttt{arXiv:1305.3062v2 [math.NT]}, abril de 2014.
    \bibitem{KPS} Klimov, N. I.; $\text{Pil}^{\prime} \text{tja}\breve{\imath}$, G. Z.; $\check{\text{S}}\text{eptickaja}$, T. A.: An estimate of the absolute constant in the Goldbach-$\check{\text{S}}\text{nirel}^{\prime} \text{man}$ problem, \textit{Studies in number theory, No. 4}, p\'ags. 35-51, Izdat. Saratov. Univ., Saratov, 1972.
    \bibitem{lw} Liu, M. C.; Wang, T.: On the Vinogradov bound in the three primes Goldbach conjecture, \textit{Acta Arith.}, 105(2):133-175, 2002.
    \bibitem{OSHP} Oliveira e Silva, T.; Herzog, S.; Pardi, S.: Empirical verification of the even Goldbach conjecture and computation of prime gaps up to $4\cdot10^{18}$, \textit{Math. Comp.}, 83:2033-2060, 2014.
    \bibitem{ram} Ramar\'e, O.: On $\check{\text{S}}\text{nirel}^{\prime} \text{man's}$ constant, \textit{Ann. Scuola Norm. Sup. Pisa Cl. Sci.}, 22(4):645-706, 1995.
    \bibitem{riva} Riesel, H.; Vaughan, R. C.: On sums of primes, \textit{Ark. Mat.}, 21(1):46-74, 1983.
    \bibitem{RS} Rosser, J. B.; Schoenfeld, L.: Approximate formulas for some functions of prime numbers, \textit{Illinois J. Math.}, 6:64-94, 1962.
    \bibitem{sch} Schnirelmann, L.: \"Uber additive Eigenschaften von Zahlen, \textit{Math. Ann.}, 107(1):649-690, 1933.
    \bibitem{tao} Tao, T.: Every odd number greater than $1$ is the sum of at most five primes, \textit{Math. Comp.}, 83:997-1038, 2014.
    \bibitem{ari} Travesa, A.: \textit{Aritm\`etica}, Co{\l}ecci\'o UB, No. 25, Barcelona, 1998.
    \bibitem{vau} Vaughan, R. C.: On the estimation of Schnirelman's constant, \textit{J. Reine Angew. Math.}, 290:93-108, 1977.
    \bibitem{vgn} Vaughan, R. C.: \textit{The Hardy-Littlewood method}, Cambridge Tracts in Mathematics, No. 125, 2a edici\'on, Cambridge University Press, 1997.
    \bibitem{vino} Vinogradov, I. M.: Sur le th\'eor\`eme de Waring, \textit{C. R. Acad. Sci. URSS}, 393-400, 1928.
    \bibitem{vin} Vinogradov, I. M.: Representation of an odd number as a sum of three primes, \textit{Dokl. Akad. Nauk. SSSR}, 15:291-294, 1937.
\end{thebibliography}
% --- Empieza la bibliografía ---

\end{document}

