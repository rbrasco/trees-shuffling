% --- Empiezan las configuraciones ---------
\documentclass[11pt,a4paper,openright,oneside]{article}
\usepackage{amsfonts, amsmath, amssymb,latexsym,amsthm, mathrsfs, enumerate, tikz-cd}
\usepackage[all]{xy}
\SelectTips{cm}{} %to change the tips and tail of the arrows 
\usepackage[spanish]{babel}
\usepackage{epsfig}

\parskip=5pt
\parindent=15pt
\usepackage[margin=1.2in]{geometry}
\usepackage{graphicx}
\usepackage{listings}
\usepackage[latin1]{inputenc}

\setcounter{page}{0}


\numberwithin{equation}{section}
\newtheorem{teo}{Teorema}[section]
\newtheorem*{teo*}{Teorema}
\newtheorem{prop}[teo]{Proposici\'on}
\newtheorem{corol}[teo]{Corolario}
\newtheorem{lema}[teo]{Lema}
\newtheorem{defi}[teo]{Definici\'on}
\newtheorem{nota}[teo]{Notaci\'on}

\theoremstyle{definition}
\newtheorem{prob}[teo]{Problema}
\newtheorem*{sol}{Soluci\'on}
\newtheorem{ex}[teo]{Ejemplo}
\newtheorem{exs}[teo]{Ejemplos}
\newtheorem{obs}[teo]{Observaci\'on}
\newtheorem{obss}[teo]{Observaciones}

\def\qed{\hfill $\square$}

\renewcommand{\refname}{Bibliografia}
\usepackage{fancyhdr}

\lhead{}
\lfoot{}
\rhead{}
\cfoot{}
\rfoot{\thepage}
\begin{document}
\bibstyle{plain}
\thispagestyle{empty}
% --- Acaban las configuraciones -----------

% ------------------------------------------
% ------------------------------------------
% ------------------------------------------

% --- Empieza la portada -------------------
\begin{titlepage}
    \begin{center}
        \begin{figure}[htb]
            \begin{center}
                % \includegraphics[width=6cm]{matematiquesinformatica-pos-rgb.png}
            \end{center}
        \end{figure}
        \vspace*{1cm}
        \textbf{\LARGE GRAU DE MATEM\`{A}TIQUES } \\
        \vspace*{.5cm}
        \textbf{\LARGE Treball final de grau} \\
        \vspace*{1.5cm}
        \rule{16cm}{0.1mm}\\
        \begin{Huge}
            \textbf{Aspectos combinatorios del producto tensorial de conjuntos dendroidales} \\
        \end{Huge}
        \rule{16cm}{0.1mm}\\
        \vspace{1cm}
        \begin{flushright}
            \textbf{\LARGE Autor: Roger Brasc\'o Garc\'es}
            \vspace*{2cm}
            \renewcommand{\arraystretch}{1.5}
            \begin{tabular}{ll}
                \textbf{\Large Director:}    & \textbf{\Large Dr. Javier J. Guti\'errez }  \\
                \textbf{\Large Realitzat a:} & \textbf{\Large  Departament de Topolog\'ia} \\
                \\
                \textbf{\Large Barcelona,}   & \textbf{\Large 23 de enero de 2022 }
            \end{tabular}
        \end{flushright}
    \end{center}
\end{titlepage}
\newpage
% --- Acaba la portada ---------------------

% ------------------------------------------
% ------------------------------------------
% ------------------------------------------

% --- Empieza el encabezado ----------------
\pagenumbering{roman}

% --- Empieza resumen ----------------------
\section*{Resumen}
 {\let\thefootnote\relax\footnote{2010 Mathematics Subject Classification. 11G05, 11G10, 14G10}}
\newpage
% --- Acaba resumen ------------------------

% ------------------------------------------

% --- Empieza agradecimientos --------------
\section*{Agradecimientos}
\newpage
% --- Acaba agradecimientos ---------------

% ------------------------------------------

% --- Empieza indice ----------------------
\tableofcontents
\newpage
% --- Acaba indice -------------------------

% --- Acaba el encabezado ------------------

% ------------------------------------------
% ------------------------------------------
% ------------------------------------------

% --- Empiezan las secciones ---------------
\pagenumbering{arabic}
\setcounter{page}{1}

% --- Empieza sección 1 --------------------
\section{Nociones previas}
\subsection{Categor\'ias}
% Bibliografía Adámek, Jiří; Herrlich, Horst; Strecker, George E. (1990), Abstract and Concrete Categories (PDF), Wiley, ISBN 0-471-60922-6 (now free on-line edition, GNU FDL). http://katmat.math.uni-bremen.de/acc/acc.pdf
\begin{defi}
    Categor\'ia
\end{defi}
Una categor\'ia is una cuadr\'upula $\mathcal{A} = (\mathcal{O}, \text{hom}, \mathit{id}, \circ)$ que consiste en:

\begin{enumerate}[(1)]
    \item Una clase $\mathcal{O}$ que sus elementos ser\'an llamados $\mathcal{A}$\textbf{-objetos}. Usaremos la notaci\'on $\mathit{Ob}(\mathcal{A})$ para simplificar.
    \item Para cada pareja de objetos $(A, B)$ de $\mathcal{A}$, tenemos un conjunto de $\text{hom}(A,B)$, cuyos elementos ser\'an llamados $\mathcal{A}$\textbf{-morfismos} de $A$ a $B$; \'es decir, los morfismos $A \overset{f}{\longrightarrow} B$ para todo $f \in \text{hom}(A,B)$.
    \item Para cada objeto $A$ de $\mathcal{A}$ definimos el morfismo $A \overset{\mathit{id}_{A}}{\longrightarrow} A$ como la identidad $A$.
    \item Sean $A \overset{f}{\longrightarrow} B$ y $B \overset{g}{\longrightarrow} C$ dos morifismos de $\mathcal{A}$, definimos la composici\'on $\circ$ como:
          $$
              \xymatrix{
                  A \ar[rd]_{g\circ f} \ar[r]^f
                  &B \ar[d]^g\\
                  &C
              }
          $$
          Composici\'on que cumple con las siguientes condiciones:

          \begin{enumerate}[(a)]
              \item Es asociativa: sean $A \overset{f}{\longrightarrow} B$, $B \overset{g}{\longrightarrow} C$ y $C \overset{h}{\longrightarrow} D$ morifismos de $\mathcal{A}$, entonces se cumple $h\circ (g\circ f) = (h\circ g)\circ f$.
              \item Respecta la identidad: para todo morfismo $A \overset{f}{\longrightarrow} B$ de $\mathcal{A}$, se cumple $\mathit{id}_B\circ f = f$ y $f\circ \mathit{id}_A = f$.
          \end{enumerate}
\end{enumerate}
\begin{ex}
    Categor\'ia $\mathbf{Set}$ cuyos objetos son todos los conjuntos y los morfismos son las funciones totales.
\end{ex}

\begin{defi}
    Categor\'ia opuesta
\end{defi}
Para toda categor\'ia $\mathcal{A} = (\mathcal{O}, \text{hom}_{\mathcal{A}}, \mathit{id}, \circ)$ definimos la categor\'ia opuesta como $\mathcal{A}^{\text{op}} = (\mathcal{O}, \text{hom}_{\mathcal{A}^{\text{op}}}, \mathit{id}, \circ^{\text{op}})$,
donde $\text{hom}_{\mathcal{A}^{\text{op}}}(A,B) = \text{hom}_{\mathcal{A}}(B,A)$ y $f\circ^{\text{op}}g = g\circ f$. Podemos observar que $\mathcal{A}$ y $\mathcal{A}^{\text{op}}$ tienen los mismos objetos y los mismos morfismos pero cambiados de direcci\'on.


\subsubsection{Functores}
\begin{defi}
    Functor
\end{defi}
Sean $\mathcal{A}$ y $\mathcal{B}$ dos categor\'ias, definimos un functor $F$ de $\mathcal{A}$ a $\mathcal{B}$ como una funci\'on que asigna cada objeto $A \in\mathit{Ob}(\mathcal{A})$ un objeto $F(A) \in\mathit{Ob}(\mathcal{B})$,
y para cada morfismo de $\mathcal{A}$ $A \overset{f}{\longrightarrow} A'$ un morfismo de $\mathcal{B}$ $F(A) \overset{F(f)}{\longrightarrow} F(A')$.
\begin{align*}
    F: \mathcal{A} & \longrightarrow \mathcal{B} \\
    A              & \longmapsto\!
    \begin{aligned}[t]
        F(A)
    \end{aligned}                    \\
    f              & \longmapsto\!
    \begin{aligned}[t]
        F(f)
    \end{aligned}
\end{align*}
De manera que:

\begin{enumerate}[(1)]
    \item $F$ conserva la composici\'on: $F(f\circ g) = F(f)\circ F(g)$, siempre y cuando $f\circ g$ est\'e bien definido.
    \item $F$ conserva los morfismos identidad: $F(\mathit{id}_A)=\mathit{id}_{F(A)}$, para cada $A \in\mathit{Ob}(\mathcal{A})$.
\end{enumerate}
\begin{defi}
    Tipos de functores
\end{defi}
Sea $F: \mathcal{A}\longrightarrow\mathcal{B}$ un functor de las categor\'ias $\mathcal{A}$ y $\mathcal{B}$.

\begin{enumerate}[(1)]
    \item $F$ es un functor covariante si preserva la direcci\'on de los morfismos; es decir, el morfismo $f: A \longrightarrow A'$ de $\mathcal{A}$ es asignado al morfismo $F(f): F(A) \longrightarrow F(A')$ de $\mathcal{B}$.
    \item $F$ es un functor contravariante si invierte la direcci\'on de los morfismos; es decir, el morfismo $f: A \longrightarrow A'$ de $\mathcal{A}$ es asignado al morfismo $F(f): F(A') \longrightarrow F(A)$ de $\mathcal{B}$.
    \item $F$ es un functor fiel si para cada par de objetos $A, A'\in\mathit{Ob}(\mathcal{A})$ la funci\'on $F_{A,A'}: \text{hom}_{\mathcal{A}}(A, A') \longrightarrow\text {hom}_{\mathcal{B}}(F(A), F(A'))$ es inyectiva.
\end{enumerate}

\subsection{Operadas}
% Bibliografía Ieke Moerdijk; Bertrand Toën. (2010), SImplicial Methods for Operads and Algebraic Geometry (PDF), Birkhäuser, ISBN 978-3-0348-0051-8
Sea $\mathcal{C}$ una categor\'ia cocompleta, sim\'etrica y monoidal, con producto tensorial $\otimes$ y unidad $I$. % TODO: Definicions de cocompleta, sim\'etrica y monoidal
Suponemos que $\mathcal{C}$ es cerrada y la $\text{hom}(X, Y)$ es la $\text{hom}$ interna. Finalmente, denotamos el grupo sim\'etrico de $n$ letras como $\sum_n$.
\begin{defi}
    Operada
\end{defi}
Una operada $P$ en $\mathcal{C}$ consiste en objetos $P(n)$ de $\mathcal{C}$ para todo $n\ge 0$ y las siguientes afirmaciones: % TODO: afirmaciones/axiomas? 

\begin{enumerate}[(1)]
    \item Elemento unidad, definido por el morfismo $I \longrightarrow P(1)$.
    \item Un producto composici\'on definido por los morfismos
          $$
              P(n)\otimes P(k_1) \otimes\dots\otimes P(k_n)\longrightarrow P(k)
          $$
          para todo $n$ y $k_1,\dots,k_n$ tal que $k=\sum_{i=1}^{n}{k_i}$. El producto composici\'on es equivariante y asociativo con la unidad.
    \item Acci\'on permutaci\'on de variables definido por la acci\'on de $\sum_n$ por la de derecha en $P(n)$ para cada $n$.
\end{enumerate}

\begin{defi}
    Morfismo de operadas
\end{defi}
Sean $P$ y $Q$ dos operadas en $\mathcal{C}$. Un morfismo de operadas $f: P \longrightarrow Q$ es definido por los morfismos $f_n: P(n)\longrightarrow Q(n)$ para cada $n$
que sean compatibles con el producto composici\'on, el elemento unidad y la acci\'on del grupo sim\'etrico.


\subsubsection{Operadas coloreadas}
Sea $C$ un conjunto cuyos elementos los nombramos colores. Una operada $C\text{-coloreada}$ $P$ consiste en:

\begin{enumerate}[(1)]
    \item Para cada secuencia de colores $c_1,\dots,c_n,c\in C$, tenemos un objeto $P(c_1,\dots,c_n;c)\in\mathcal{C}$.
          Este objeto reperesenta el conjunto de operaciones cuyas entradas son los colores $c_1,\dots,c_n$ y las salidas son el color $c$.
    \item Elemento unidad, definido por el morfismo $I \longrightarrow P(c;c)$ para todo $c\in C$.
    \item Para cada tupla de $n+1$ colores $(c_1,\dots,c_n;c)$ y $n$ tuplas cualesquiera
          $$
              (d_{1,1},\dots,d_{1,k_1};c_1),\dots,(d_{n,1},\dots,d_{n,k_n};c_n)
          $$
          definimos un producto composici\'on asociativo mediante los morfismos
          \begin{align*}
              P(c_1,\dots,c_n;c) & \otimes P(d_{1,1},\dots,d_{1,k_1};c_1) \otimes\dots\otimes P(d_{n,1},\dots,d_{n,k_n};c_n) \\
                                 & \longrightarrow P(d_{1,1},\dots,d_{1,k_1},\dots,d_{n,1},\dots,d_{n,k_n};c)
          \end{align*}
    \item Acci\'on permutaci\'on de variables definido por la acci\'on del grupo sim\'etrico. Sea $\sigma\in\sum_n$ una permutaci\'on, definimos el morfismo
          $$
              \sigma^{*}: P(c_1,\dots,c_n;c) \longrightarrow P(c_{\sigma(1)},\dots,c_{\sigma(n)};c)
          $$
\end{enumerate}

\begin{defi}
    Morfismo de operadas coloreadas
\end{defi}
Sean $P$ y $Q$ dos operadas $C\text{-coloreada}$ y $D\text{-coloreada}$, respectivamente, en $\mathcal{C}$. Un morfismo de $P$ a $Q$ de operadas coloreadas es formado
por un morfismo de colores $f: C \longrightarrow D$ y los morfismos
$$
    \varphi_{c_1,\dots,c_n;c}: P(c_1,\dots,c_n;c) \longrightarrow Q(f(c_1),\dots,f(c_n);c)
$$
que sean compatibles con el producto composici\'on, el elemento unidad y la acci\'on del grupo sim\'etrico.

Usaremos la notaci\'on $Oper(\mathcal{C})$ para referenciar a la categor\'ia cuyos objetos son operadas coloreadas en $\mathcal{C}$ y cuyos morfismos son morfismos de operadas coloreadas.

\newpage
% --- Acaba sección 1 ----------------------

% ------------------------------------------

% --- Empieza sección 2 --------------------
\section{Conjuntos Simpliciales}
% Bibliografía Greg Friedman. (2011), An elementary illustrated introduction to simplicial sets (PDF)
\subsection{Complejos simpliciales}
\begin{defi}
    N-simplex
\end{defi}
Un $n\text{-simplex}$ es un politopo de $n\ge 0$ dimensiones formando una envoltura convexa de $n+1$ vertices. Es decir, es un conjunto de puntos afines independientes en un espacio eucl\'ideo de dimensi\'on $n$.

Una cara $m$ de un $n\text{-simplex}$ es una envolutra convexa de $m\le n$ vertices.

\begin{defi}
    Complejo simplicial
\end{defi}
Sea $n\in\mathbb{N}^{*}$, un complejo simplicial $X$ es un conjunto finito de $m\text{-simplex}$ con $m\le n$ que cumplen las condiciones:

\begin{enumerate}[(1)]
    \item Si $m\text{-simplex}\in X \Rightarrow \forall m'\le m\text{, }m'\text{-simplex}\in X$.
    \item Si dos simplices de $X$ se cortan, entonces su intersecci\'on es una cara com\'un.
\end{enumerate}

Sea $X^k$ un complejo simplicial formado por todos los $k\text{-simplex}$ de $X$. Observamos que todo elemento de $X^k$ es un subconjunto de $X^0$ con cardinal $k+1$, donde $X^0=\{v_0,\dots ,v_n\}$.
Generalmente, todo subconjunto de $X^k$ de $j+1$ elementos es un elemento de $X^j$.

Sea $X_k$ un conjunto formado por $k\text{-simplices}$.

\begin{defi}
    N-simplex ordenado
\end{defi}
Un $n\text{-simplex}$ formado por los v\'ertices $v_0,\dots,v_n \in X^0$ es ordenado cuando cuando los v\'ertices estan ordenados, en ese caso nombramos cada v\'ertice por los n\'umeros $0,\dots,n$. Usaremos la notaci\'on $|\Delta^n| = [0,\dots,n]$ para simplificar.


\subsubsection{Morfismos simpliciales}
\begin{defi}
    Morfismo simplicial
\end{defi}
Sea $K$ y $L$ complejos simpliciales. Sea un morfismo simplicial $F: K \longrightarrow L$ que envia los vertices de K a los vertices de L. Es decir, $\forall v \in K^0 \text{, } v \longmapsto F(v) \in L^0$.

\begin{defi}
    Cara
\end{defi}
Para todo $|\Delta^n|$ tenemos $n+1$ caras definidas por los morfismos $\delta_0,\dots,\delta_n$
\begin{align*}
    \delta_j: X_n & \longrightarrow X_{n-1} \\
    [0,\dots,n]   & \longmapsto\!
    \begin{aligned}[t]
        [0,\dots,\hat{j},\dots,n]
    \end{aligned}
\end{align*}
Donde $X_n$ y $X_{n-1}$ son conjuntos de simplices ordenados de $n$ y $n-1$ v\'ertices, respectivamente. Observamos que $\forall i<j$, $\delta_i\delta_j = \delta_{j-1}\delta_{i}$.

\begin{defi}
    Morifismo degenerativo
\end{defi}
Para todo $|\Delta^n|$ tenemos $n+1$ morfismos degenerativos $\sigma_0,\dots,\sigma_n$
\begin{align*}
    \sigma_j: X_n & \longrightarrow X_{n+1} \\
    [0,\dots,n]   & \longmapsto\!
    \begin{aligned}[t]
        [0,\dots,j,j,\dots,n]
    \end{aligned}
\end{align*}
Donde $X_n$ y $X_{n+1}$ son conjuntos de simplices ordenados de $n$ y $n+1$ v\'ertices, respectivamente. Observamos que $\forall i\le j$, $\sigma_i\sigma_j = \sigma_{j+1}\sigma_{i}$.

\subsection{Conjunto Delta}
\begin{defi}
    Conjunto Delta
\end{defi}
Definimos un conjunto Delta como una secuencia de conjuntos $X_0,X_1,\dots$ y para cada $n\ge 0$ las funciones $\delta_i: X_{n+1} \longrightarrow X_n$, $\forall 0\le i \le n+1$, que cumplen $\delta_i\delta_j = \delta_{j-1}\delta_{i}$, $\forall i\le j$.
Formando el siguiente diagrama (Falta por hacer)

$$ %Not working
    \xymatrix{
    X_0  \ar@/{}^{1pc}/[r]  & X_1  \ar@/{}^{1pc}/[r] & X_2  \dots
    }
$$

\subsubsection{Definici\'on categ\'orica del conjunto Delta}
\begin{defi}
    Categor\'ia $\hat{\Delta}$
\end{defi}
Sea la categor\'ia $\hat{\Delta}$ cuyos objetos son los conjuntos estrictamente ordenados finitos $[n] = \{0,\dots,n\}$ y los morfismos son las funciones, que mantienen el orden estrictamente, $f: [m] \longrightarrow [n]$, $m\le n$. Podemos pensar que sea la inclusi\'on de un $m\text{-simplex}$ como cara de un $n\text{-simplex}$.
Para todo $0\le i \le n$ consideramos los morfismos:
\begin{align*}
    d_i: [n]      & \longrightarrow [n+1] \\
    \{0,\dots,n\} & \longmapsto\!
    \begin{aligned}[t]
        \{0,\dots, \hat{i}, \dots,n+1\}
    \end{aligned}
\end{align*}

\begin{defi}
    Categor\'ia $\hat{\Delta}^{op}$
\end{defi}
Sea la categor\'ia $\hat{\Delta}^{op}$, la categor\'ia opuesta de $\hat{\Delta}$, cuyos objetos son los conjuntos estrictamente ordenados finitos $[n] = \{0,\dots,n\}$ y los morfismos son las funciones, que mantienen el orden estrictamente, $f: [n] \longrightarrow [m]$, $m\le n$. Podemos pensar que sea la extracci\'on de la cara $m\text{-simplex}$ de un $n\text{-simplex}$.
Para todo $0\le i \le n$ consideramos los morfismos:
\begin{align*}
    \delta_i: [n] & \longrightarrow [n-1] \\
    \{0,\dots,n\} & \longmapsto\!
    \begin{aligned}[t]
        \{0,\dots, \hat{i}, \dots,n\}
    \end{aligned}
\end{align*}

\begin{defi}
    Conjunto Delta
\end{defi}
Un conjunto Delta es un functor covariante $X: \hat{\Delta}^{op} \longrightarrow \mathbf{Set}$, equivalentemente es un functor contravariante $X: \hat{\Delta} \longrightarrow \mathbf{Set}$.

Faltan observaciones.

\subsection{Conjunto simplicial}
\begin{defi}
    Conjunto simplicial
\end{defi}
Definimos un conjunto simplicial como una secuencia de conjuntos $X_0,X_1,\dots$ y para cada $n\ge 0$ las funciones $\delta_i: X_n \longrightarrow X_{n-1}$ y $\sigma_i: X_n \longrightarrow X_{n+1}$, $\forall 0\le i \le n$, que cumplen:

\begin{enumerate}[(1)]
    \item $\delta_i\delta_j = \delta_{j-1}\delta_{i}$, $i<j$
    \item $\delta_i\sigma_j = \sigma_{j-1}\delta_{i}$, $i<j$
    \item $\delta_j\sigma_j = \delta{j+1}\sigma_{j} = id$
    \item $\delta_i\sigma_j = \sigma_{j}\delta_{i-1}$, $i>j+1$
    \item $\sigma_i\sigma_j = \sigma_{j+1}\sigma_{i}, i\le j$
\end{enumerate}
Formando el siguiente diagrama (Falta por hacer)

$$ %Not working
    \xymatrix{
    X_0  \ar@/{}^{1pc}/[r]  & X_1  \ar@/{}^{1pc}/[r] & X_2  \dots
    }
$$

\subsubsection{Definici\'on categ\'orica del conjunto simplicial}
\begin{defi}
    Categor\'ia $\Delta$
\end{defi}
Sea la categor\'ia $\Delta$ cuyos objetos son los conjuntos ordenados finitos $[n] = \{0,\dots,n\}$ y los morfismos son las funciones, que mantienen solamente el orden, $f: [m] \longrightarrow [n]$.
Para todo $0\le i \le n$ consideramos los morfismos:
\begin{align*}
    d_i: [n]      & \longrightarrow [n+1] \\
    \{0,\dots,n\} & \longmapsto\!
    \begin{aligned}[t]
        \{0,\dots, \hat{i}, \dots,n+1\}
    \end{aligned}
\end{align*}
\begin{align*}
    s_i: [n+1]      & \longrightarrow [n] \\
    \{0,\dots,n+1\} & \longmapsto\!
    \begin{aligned}[t]
        \{0,\dots,i,i, \dots,n\}
    \end{aligned}
\end{align*}

\begin{defi}
    Categor\'ia $\hat{\Delta}^{op}$
\end{defi}
Sea la categor\'ia $\Delta^{op}$, la categor\'ia opuesta de $\Delta$, cuyos objetos son los conjuntos ordenados finitos $[n] = \{0,\dots,n\}$ y los morfismos son las funciones, que mantienen solamente el orden, $f: [m] \longrightarrow [n]$.
Para todo $0\le i \le n$ consideramos los morfismos:
\begin{align*}
    \delta_i: [n] & \longrightarrow [n-1] \\
    \{0,\dots,n\} & \longmapsto\!
    \begin{aligned}[t]
        \{0,\dots, \hat{i}, \dots,n\}
    \end{aligned}
\end{align*}
\begin{align*}
    \sigma_i: [n] & \longrightarrow [n+1] \\
    \{0,\dots,n\} & \longmapsto\!
    \begin{aligned}[t]
        \{0,\dots, i,i, \dots,n\}
    \end{aligned}
\end{align*}

\begin{defi}
    Conjunto simplicial
\end{defi}
Un conjunto simplicial es un functor covariante $X: \Delta^{op} \longrightarrow \mathbf{Set}$, equivalentemente es un functor contravariante $X: \Delta \longrightarrow \mathbf{Set}$.
Usaremos la notaci\'on $\Delta[n]=\Delta(\_,[n])$.
\begin{align*}
    \Delta[n]: \Delta^{op} & \longrightarrow \mathbf{Set} \\
    [m]                    & \longmapsto\!
    \begin{aligned}[t]
        \Delta([m],[n])
    \end{aligned}
\end{align*}
Faltan observaciones.

\subsection{Realizaci\'on geom\'etrica}
\begin{defi}
    Realizaci\'on geom\'etrica
\end{defi}
Sea $X$ un conjunto simplicial. Dotamos cada $X_n$ con la topolog\'ia discreta y sea $|\Delta^n$ el $n\text{-simplex}$ dotado de su topolog\'ia estandard. Definimos la realizaci\'on geom\'etrica como
$$
    |X| = \coprod_{n=0}^{\infty}X_n \times |\Delta^n| / \sim
$$
Donde $\sim$ es la relaci\'on de equivalencia generada por las relaciones:

\begin{enumerate}[(1)]
    \item $(x, d_i(p)) \sim (\delta_i(x), p)$, $x\in X_{n+1}$ y $p\in|\Delta^n|$
    \item $(x, s_i(p)) \sim (\sigma_i(x), p)$, $x\in X_{n-1}$ y $p\in|\Delta^n|$
\end{enumerate}

\begin{ex}
    $\Delta[2] = \Delta(\_, [2])$
\end{ex}
Falta por escribir

% --- Acaba sección 2 ----------------------

% ------------------------------------------

% --- Empieza sección 3 --------------------
\section{Conjuntos Dendroidales}
\subsection{\'Arbol como operadas}
\subsubsection{Caras}
\subsubsection{Funciones degenerativas}
\subsubsection{Identidades de morfismos}
\subsubsection{\'Arboles no planares}
\subsection{Conjunto Dendroidal}
\subsection{Producto tensorial de conjuntos dendroidales}
\subsubsection{Producto tensorial Boardman Vogt}
\subsubsection{Producto Producto tensorial de conjuntos dendroidales}
\newpage
% --- Acaba sección 3 ----------------------

% ------------------------------------------

% --- Empieza sección 4 --------------------
\section{Injertos de \'arboles}
\subsection{Producto tensorial de \'arboles lineales}
\subsection{Producto tensorial de \'arboles}
\subsubsection{Injertos de \'arboles resultantes}
\subsection{C\'alculo de \'arboles resultantes}
\subsubsection{Conjunto de \'arboles resultantes}
\subsubsection{Generarador de \'arboles en Python}
\newpage
% --- Acaba sección 4 ----------------------

% ------------------------------------------

% --- Empieza sección 5 --------------------
\section{Conclusiones}
\newpage
% --- Acaba sección 5 ----------------------

% --- Acaban las secciones -----------------

% --- Empieza la bibliografía ---
\begin{thebibliography}{25}
    \bibitem{pari} Batut, C.; Belabas, K.; Bernardi, D.; Cohen, H.; Olivier, M.: User's guide to \textit{PARI-GP},  \newline \texttt{pari.math.u-bordeaux.fr/pub/pari/manuals/2.3.3/users.pdf}, 2000.
    \bibitem{cw} Chen, J. R.; Wang, T. Z.: On the Goldbach problem, \textit{Acta Math. Sinica}, 32(5):702-718, 1989.
    \bibitem{desh} Deshouillers, J. M.: Sur la constante de $\check{\text{S}}\text{nirel}^{\prime} \text{man}$, \textit{S\'eminaire Delange-Pisot-Poitou, 17e ann\'ee: (1975/76), Th\'eorie des nombres: Fac. 2, Exp. No.} G16, p\'ag. 6, Secr\'etariat Math., Paris, 1977.
    \bibitem{derz} Deshouillers, J. M.; Effinger, G.; te Riele, H.; Zinoviev, D.: A complete Vinogradov 3-primes theorem under the Riemann hypothesis, \textit{Electron. Res. Announc. Amer. Math. Soc.}, 3:99-104, 1997.
    \bibitem{dick} Dickson, L. E.: \textit{History of the theory of numbers. Vol. I: Divisibility and primality}, Chelsea Publishing Co., New York, 1966.
    \bibitem{hl} Hardy, G. H.; Littlewood, J. E.: Some problems of \textquoteleft Partitio numerorum\textquoteright; III: On the expression of a number as a sum of primes, \textit{Acta Math.}, 44(1):1-70, 1923.
    \bibitem{hara} Hardy, G. H.; Ramanujan, S.: Asymptotic formulae in combinatory analysis, \textit{Proc. Lond. Math. Soc.}, 17:75-115, 1918.
    \bibitem{haw} Hardy, G. H.; Wright, E. M.: \textit{An introduction to the theory of numbers}, 5a edici\'on, Oxford University Press, 1979.
    \bibitem{minarc} Helfgott, H. A.: Minor arcs for Goldbach's problem, \newline \texttt{arXiv:1205.5252v4 [math.NT]}, diciembre de 2013.
    \bibitem{majarc} Helfgott, H. A.: Major arcs for Goldbach's problem, \newline \texttt{arXiv:1305.2897v4 [math.NT]}, abril de 2014.
    \bibitem{istrue} Helfgott, H. A.: The ternary Goldbach conjecture is true, \newline \texttt{arXiv:1312.7748v2 [math.NT]}, enero de 2014.
    \bibitem{HP} Helfgott, H. A.; Platt, D.: Numerical verification of the ternary Goldbach conjecture up to $8.875 \cdot 10^{30}$, \texttt{arXiv:1305.3062v2 [math.NT]}, abril de 2014.
    \bibitem{KPS} Klimov, N. I.; $\text{Pil}^{\prime} \text{tja}\breve{\imath}$, G. Z.; $\check{\text{S}}\text{eptickaja}$, T. A.: An estimate of the absolute constant in the Goldbach-$\check{\text{S}}\text{nirel}^{\prime} \text{man}$ problem, \textit{Studies in number theory, No. 4}, p\'ags. 35-51, Izdat. Saratov. Univ., Saratov, 1972.
    \bibitem{lw} Liu, M. C.; Wang, T.: On the Vinogradov bound in the three primes Goldbach conjecture, \textit{Acta Arith.}, 105(2):133-175, 2002.
    \bibitem{OSHP} Oliveira e Silva, T.; Herzog, S.; Pardi, S.: Empirical verification of the even Goldbach conjecture and computation of prime gaps up to $4\cdot10^{18}$, \textit{Math. Comp.}, 83:2033-2060, 2014.
    \bibitem{ram} Ramar\'e, O.: On $\check{\text{S}}\text{nirel}^{\prime} \text{man's}$ constant, \textit{Ann. Scuola Norm. Sup. Pisa Cl. Sci.}, 22(4):645-706, 1995.
    \bibitem{riva} Riesel, H.; Vaughan, R. C.: On sums of primes, \textit{Ark. Mat.}, 21(1):46-74, 1983.
    \bibitem{RS} Rosser, J. B.; Schoenfeld, L.: Approximate formulas for some functions of prime numbers, \textit{Illinois J. Math.}, 6:64-94, 1962.
    \bibitem{sch} Schnirelmann, L.: \"Uber additive Eigenschaften von Zahlen, \textit{Math. Ann.}, 107(1):649-690, 1933.
    \bibitem{tao} Tao, T.: Every odd number greater than $1$ is the sum of at most five primes, \textit{Math. Comp.}, 83:997-1038, 2014.
    \bibitem{ari} Travesa, A.: \textit{Aritm\`etica}, Co{\l}ecci\'o UB, No. 25, Barcelona, 1998.
    \bibitem{vau} Vaughan, R. C.: On the estimation of Schnirelman's constant, \textit{J. Reine Angew. Math.}, 290:93-108, 1977.
    \bibitem{vgn} Vaughan, R. C.: \textit{The Hardy-Littlewood method}, Cambridge Tracts in Mathematics, No. 125, 2a edici\'on, Cambridge University Press, 1997.
    \bibitem{vino} Vinogradov, I. M.: Sur le th\'eor\`eme de Waring, \textit{C. R. Acad. Sci. URSS}, 393-400, 1928.
    \bibitem{vin} Vinogradov, I. M.: Representation of an odd number as a sum of three primes, \textit{Dokl. Akad. Nauk. SSSR}, 15:291-294, 1937.
\end{thebibliography}
% --- Empieza la bibliografía ---

\end{document}

